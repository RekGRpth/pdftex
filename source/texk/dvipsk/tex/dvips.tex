%
%   This document describes how to use dvips.  Macros at top cut and
%   slashed from another document.
%
\newif\ifgeneric\genericfalse
\ifx\ntmanloaded Y\else\input dvipsmac \fi % uh oh, no macros, give our own.
%
%   Finally, the document itself.
%
The \.{dvips} program converts a \TeX\ \.{dvi} file into a PostScript file
for printing or distribution.  Seldom has such a seemingly easy programming
task required so much effort.

% Why use dvips?

\sec{Why Use dvips?}

The \.{dvips} program has a number of features that set it apart from other
PostScript drivers for \TeX.  This rather long section describes the advantages
\^{{\tt dvips}}
\^{PostScript}
of using \.{dvips}, and may be skipped if you are just interested in learning
how to use the program.  Installation is covered in section 14 near the
end of this document.

The \.{dvips} driver generates excellent, standard PostScript, that can be
included in other documents as figures or printed through a variety of
spoolers.  The generated PostScript requires very little printer memory,
so very complex documents with a lot of fonts can easily be printed even
on PostScript printers without much memory, such as the original Apple
LaserWriter.  The PostScript output is also compact, requiring less disk
space to store and making it feasible as a transfer format.

Even those documents that are too complex to print in their entirety
on a particular printer
can be printed, since \.{dvips} will automatically split such documents
\^{memory}
into pieces, reclaiming the printer memory between each piece.

The \.{dvips} program supports graphics in a natural way, allowing PostScript
graphics to be included and automatically scaled and positioned in a variety
of ways.

Printers with resolutions other than 300 dpi are also supported, even if they
have a different resolution in the horizontal and vertical directions.
High resolution output is supported for typesetters, including an option
that compresses the bitmapped fonts so that typesetter virtual memory is
\^{resolution}
not exhausted.  This option also significantly reduces the size of the
PostScript file and decoding in the printer is very fast.

Missing fonts can be automatically generated if \MF\ exists on the system,
or fonts can be converted from \.{gf} to \.{pk} format on demand.
\^{automatic font generation}
If a font cannot be generated, a scaled version of the same font at a
different size can be used instead, although \.{dvips} will complain
loudly about the poor \ae sthetics of the resulting output.

Users will appreciate features such as collated copies and support
for \.{tpic}, \.{psfig}, \.{emtex}, and \.{METAPOST};
\^{{\tt tpic}}
\^{{\tt psfig}}
\^{{\tt emtex}}
\^{{\tt METAPOST}}
system administrators will love the support for multiple printers, each
with their own configuration file, and the ability to pipe the output
directly to a program such as \.{lpr}.
Support for MS-DOS, OS/2, and VMS in addition to UNIX is provided in the
\^{MS-DOS}
\^{UNIX}
\^{VMS}
standard distribution, and porting to other systems is easy.

One of the most important features is the support of virtual fonts, which
add an entirely new level of flexibility to \TeX.  Virtual fonts are used to
give \.{dvips} its excellent PostScript font support, handling all the font
remapping in a natural, portable, elegant, and extensible way.  The \.{dvips}
\^{{\tt afm2tfm}}
driver even comes with its own \.{afm2tfm} program that creates the necessary
virtual fonts and \TeX\ font metric files automatically from the Adobe
font metric files.

Source is provided and freely distributable, so adding a site-specific feature
is possible.  Adding such features is made easier by the highly modular
structure of the program.

There is really no reason to use another driver, and the more people use
\.{dvips}, the less time will be spent fighting with PostScript and the
more time will be available to create beautiful documents.
So if you don't use \.{dvips} on your system, get it today.

\sec{Using dvips}

To use \.{dvips}, simply type
\cmd{localhost> dvips foo}
\noindent
where \.{foo.dvi} is the output of \TeX\ that you want to print.  If \.{dvips}
has been installed correctly, the document should come out of your default
printer.

If you use fonts that have not been used on your system before, they
may be automatically generated; this process can take a few minutes.
The next time that document is printed, these fonts will already exist, so
printing will go much faster.

Many options are available; they are described in a later section.
For a brief summary of available options, just type
\cmd{localhost> dvips}

\sec{Paper Size and Landscape Mode}

Most \TeX\ documents at a particular site are designed to use the
standard paper size (for example, letter size in the United States
or A4 in Europe.)  The \.{dvips} program defaults to these paper
sizes and can be customized for the defaults at each site or on
each printer.

But many documents are designed for other paper sizes.  For instance,
you may want to design a document that has the long edge of the
\^{landscape}
paper horizontal.  This can be useful when typesetting booklets,
brochures, complex tables, or many other documents.  This type of
paper orientation is called landscape orientation (the `normal'
orientation is portrait).
\^{portrait}
Alternatively, a document might be designed for ledger or A3 paper.

Since the intended paper size is a document design decision, and
not a decision that is made at printing time, such information
should be given in the \TeX\ file and not on the \.{dvips}
command line.  For this reason, \.{dvips} supports a \.{papersize}
special.  It is hoped that this special will become standard over
time for \TeX\ previewers and other printer drivers.

The format of the \.{papersize} special is
\cmd{\ttbackslash special\char123papersize=8.5in,11in\char125}
\noindent
where the dimensions given above are for a standard letter sheet.
The first dimension given is the horizontal size of the page, and
the second is the vertical size.  The dimensions supported are
the same as for \TeX; namely, in (inches), cm (centimeters), mm
(millimeters), pt (points), sp (scaled points), bp (big points, the
same as the default PostScript unit), pc (picas), dd (didot points),
and cc (ciceros).

For a landscape document, the \.{papersize} comment would be given as
\cmd{\ttbackslash special\char123papersize=11in,8.5in\char125}
\noindent
An alternate specification of \.{landscape} is to have a special
of the form
\cmd{\ttbackslash special\char123landscape\char125}
\noindent
This is supported for backward compatibility, but it is hoped that
eventually the \.{papersize} comment will dominate.

Of course, using such a command only informs \.{dvips} of the desired
paper size; you must still adjust the \.{hsize} and \.{vsize} in your
\TeX\ document to actually use the full page.

The {\tt papersize} special must occur somewhere on the first page of the
document.

The {\tt dvips} default landscape configuration is presently as though
the paper were rotated ninety degrees counterclockwise; this seems
to be the convention adopted by previewers that the author is
familiar with.  If for some reason you need your landscape orientation
to be rotated clockwise, simply include at the top of your \TeX\ file
or in some included macro file
\cmd{\ttbackslash special\char123! /landplus90 true store\char125}
\noindent
to set this orientation.  Alternatively, you can change the default
value of {\tt landplus90} in the {\tt tex.lpro} file before
compilation, or include a header file that just includes the
definition {\tt /landplus90 true store}.

\sec{Including PostScript Graphics}

Scaling and including PostScript graphics is a breeze---if the PostScript
\^{graphics}
\^{PostScript graphics}
file is correctly formed.  Even if it is not, however, the file can usually
be accommodated with just a little more work.  The most important feature of
a good PostScript file---from the standpoint of including it in another
document---is an accurate bounding box comment.

\sub{The Bounding Box Comment}

Every well-formed PostScript file has a comment describing where on the
\^{bounding box}
page the graphic is located, and how big that graphic is.  This information
is given in terms of the lower left and upper right corners of a box just
enclosing the graphic, and is thus referred to as a bounding box.
These coordinates are given
in PostScript units (there are precisely 72 PostScript units to
the inch) with respect to the lower left corner of the sheet of paper.

To see if a PostScript file has a bounding box comment, just
look at the first few lines of the file.
(PostScript is standard ASCII, so you can use any text editor to do this.)
If within the first few dozen lines there is a line of the form
\cmd{\%\%BoundingBox:\ 0 1 2 3}
\noindent
(with any numbers), chances are very good that the file is Encapsulated
PostScript and will work easily with \.{dvips}.  If the file contains
instead a line like
\cmd{\%\%BoundingBox:\ (atend)}
\noindent
the file is still probably Encapsulated PostScript, but the bounding
box (that \.{dvips} needs to position the graphic) is at the end of the
file and should be moved to the position of the line above.  This can be
done with that same text editor, or with a simple Perl script.

If the document lacks a bounding box altogether, one can easily be added.
Simply print the file.  Now, take a ruler, and make the following measurements.
All measurements should be in PostScript units, so measure it in inches and
multiply by 72.  Alternatively, the {\tt bbfig} program distributed with
{\tt dvips} in the {\tt contrib} directory can be used to automate
this process.

From the left edge of the paper to the leftmost mark on the paper is
{\it llx}, the first number.  From the bottom edge of the paper to the
bottommost mark on the paper is {\it lly}, the second number.  From
the left edge of the paper to the rightmost mark on the paper is
{\it urx}, the third number.  The fourth and final number, {\it ury}, is
the distance from the bottom of the page to the uppermost mark on the
paper.

Now, add a comment of the following form as the second line of the document.
(The first line should already be a line starting with the two
characters `{\tt \%!}'; if it is not, the file probably isn't PostScript.)
\cmd{\%\%BoundingBox:\ {\it llx lly urx ury}}
\noindent
Or, if you don't want to modify the file, you can simply write these
numbers down in a convenient place and use them when you import the
graphic.

If the document does not have such a bounding box, or if the bounding box
is given at the end of the document, please complain to the authors of the
software package that generated the file; without such a line, including
PostScript graphics can be tedious.

\sub{Using the EPSF Macros}

Now you are ready to include the graphic into a \TeX\ file.  Simply add to
\^{{\tt epsf} macros}
the top of your \TeX\ file a line like
\cmd{\ttbackslash input epsf}
\noindent
(or, if your document is in La\TeX\ or Sli\TeX, add the \.{epsf} style
option, as was done to the following line).
\cmd{\ttbackslash documentstyle[12pt,epsf]\char123article\char125}
\noindent
This only needs to be done once, no matter how many figures you plan to
include.  Now, at the point you want to include the file, enter a line
such as
\cmd{\ttbackslash epsffile\char123foo.ps\char125}
\noindent
If you are using La\TeX\ or Sli\TeX, you may need to add a
\.{\ttbackslash leavevmode} command immediately before the
\^{{\tt leavevmode}}
\.{\ttbackslash epsffile} command to get certain environments to work
correctly.  If your file did not (or does not currently) have a bounding
box comment, you should supply those numbers you wrote down as in the
following example:
\cmd{\ttbackslash epsffile[100 100 500 500]\char123foo.ps\char125}
\noindent
(in the same order they would have been in a normal bounding box comment).
Now, save your changes and run \TeX\ and \.{dvips}; the output should
have your graphic positioned at precisely the point you indicated, with the
proper amount of space reserved.

The effect of the \.{\ttbackslash epsffile} macro is to typeset the figure
as a \TeX\ \.{\ttbackslash vbox} at the point of the page that the command
is executed.  By default, the graphic will have its
`natural' width (namely, the width of its bounding box).
The \TeX\ box will have depth zero and a `natural' height.
The graphic will be scaled by any \.{dvi} magnification in effect at the
\^{magnification}
time.

Any PostScript graphics included by any method in this document (except
\^{{\tt bop-hook}}
\.{bop-hook} and its ilk) are scaled by the current \.{dvi} magnification.
For graphics included with \.{\ttbackslash epsffile} where the size is given
in \TeX\
dimensions, this scaling will produce the correct, or expected, results.
For compatibility with old PostScript drivers, it is possible to turn
\^{{\tt magscale}}
this scaling off with the following \TeX\ command:
\cmd{\ttbackslash special\char123! /magscale false def\char125}
\noindent
Use of this command is not recommended because it will make the
\.{\ttbackslash epsffile} graphics the wrong size if global
magnification is used
in a \.{dvi} document, and it will cause any PostScript graphics to
appear improperly scaled and out of position if a \.{dvi} to \.{dvi}
program is used to scale or otherwise modify the document.

You can enlarge or reduce the figure by putting
\cmd{\ttbackslash epsfxsize=<dimen>}
\noindent
right before the call to \.{\ttbackslash epsffile}.
Then the width of the \TeX\ box will be \.{<dimen>} and its
height will be scaled proportionately.  Alternatively you can force the
\^{{\tt epsfxsize}}
vertical size to a particular size with
\cmd{\ttbackslash epsfysize=<dimen>}
\noindent
in which case the height will be set and the width will be scaled
proportionally.  If you set both, the aspect ratio of the included
graphic will be distorted but both size specifications will be
honored.

By default, clipping is disabled for included EPSF images.  This is
because clipping to the bounding box dimensions often cuts off a
small portion of the figure, due to slightly inaccurate bounding
box arguments.  The problem might be subtle; lines around the boundary
of the image might be half their intended width, or the tops or
bottoms of some text annotations might be sliced off.  If you want to
turn clipping on, just use the command
\cmd{\ttbackslash epsfclipon}
\noindent
and to turn clipping back off, use
\cmd{\ttbackslash epsfclipoff}

A more general facility for sizing is available by defining the
\.{\ttbackslash epsfsize} macro.   You can redefine this macro
\^{{\tt epsfsize}}
to do almost anything.  This \TeX\ macro is passed two parameters
by \.{\ttbackslash epsffile}.  The first parameter is the natural
horizontal size of
the PostScript graphic, and the second parameter is the natural vertical
size.  This macro is responsible for returning the desired horizontal size of
the graph (the same as assigning \.{\ttbackslash epsfxsize} above).

In the definitions given below, only the body is given; it should be inserted
in
\cmd{\ttbackslash def\ttbackslash epsfsize\#1\#2\char123{\it body}\/\char125}
\noindent
Some common definitions are:

{\options
\.{\ttbackslash epsfxsize}
This definition (the default) enables the default features listed above,
by setting \.{\ttbackslash epsfxsize} to the same value it had before
the macro was called.

\.{0pt}
This definition forces natural sizes for all graphics by setting the width to
zero, which turns on horizontal scaling.

\.{\#1}
This forces natural sizes too, by returning the first parameter only
(the natural width) and setting the width to it.

\.{\ttbackslash hsize}
This forces all graphics to be scaled so they are as wide as the current
horizontal size.  (In La\TeX, use \.{\ttbackslash textwidth} instead
of \.{\ttbackslash hsize}.)

\.{0.5\#1}
This scales all figures to half of their natural size.

\.{\ttbackslash ifdim\#1>\ttbackslash hsize\ttbackslash hsize\ttbackslash
else\#1\ttbackslash fi}
This keeps graphics at their natural size, unless the width would be wider
than the current \.{\ttbackslash hsize}, in which case the graphic is
scaled down to \.{\ttbackslash hsize}.\par}

If you want \TeX\ to report the size of the figure as a message
on your terminal when it processes each figure, give the command
\cmd{\ttbackslash epsfverbosetrue}

\sub{Header Files}

Often in order to get a particular graphic file to work, a certain header
\^{header}
file might need to be sent first.  Sometimes this is even desirable, since
the size of the header macros can dominate the size of certain PostScript
graphics files.  The \.{dvips} program provides support for this with the
\.{header=} special command.  For instance, to ensure that \.{foo.ps}
gets downloaded as part of the header material, the following command
should be added to the \TeX\ file:
\cmd{\ttbackslash special\char123header=foo.ps\char125}
\noindent
The dictionary stack will be at the \.{userdict} level when these header
files are included.

For these and all other header files (including the headers required
by \.{dvips} itself and any downloaded fonts), the printer VM budget is
debited by some value.  If the header file has, in its first 1024 bytes,
a line of the form
\cmd{\%\%VMusage: {\it min} {\it max}}
\noindent
then the maximum value is used.  If it doesn't, then the total size of
the header file in bytes is used as an approximation of the memory
requirements.

\sub{Literal PostScript}

For simple graphics, or just for experimentation, literal PostScript
\^{literal PostScript}
graphics can be included.  Simply use a special command that starts with
a double quote ({\tt"}).
For instance, the following (simple) graphic:
\vskip\baselineskip
\vbox to 100bp{\vss % a bp is the same as a PostScript unit
\special{" newpath 0 0 moveto 100 100 lineto 394 0 lineto
closepath gsave 0.8 setgray fill grestore stroke}}

\noindent
was created by typing:
\cmd{\ttbackslash vbox to 100bp\char123\ttbackslash vss
 \% a bp is the same as a PostScript unit\\
\ttbackslash special\char123" newpath 0 0 moveto 100 100 lineto 394 0 lineto\\
closepath gsave 0.8 setgray fill grestore stroke\char125\char125}
\noindent
(Note that you are responsible for leaving space for such literal graphics.)
Literal graphics are discouraged because of their nonportability.

\sub{Literal Headers}

Similarly, you can define your own macros for use in such literal graphics
through the use of literal macros.  Literal macros are defined just like
literal graphics, only you begin the special with an exclamation point
instead of a double quote.  These literal macros are included as part of the
header material in a special dictionary called \.{SDict}.  This dictionary
\^{{\tt SDict}}
is the first one on the PostScript dictionary stack when any PostScript
graphic is included, whether by literal inclusion or through the
\.{\ttbackslash epsffile} macros.

\sub{Other Graphics Support}

There are other ways to include graphics with \.{dvips}.  One is to use an
\^{obsolete}
existing package, such as \.{emtex}, \.{psfig}, \.{tpic}, or \.{METAPOST},
all supported by \.{dvips}.

Other facilities are available for historical reasons, but their use is
discouraged, in hope that some `sane' form of PostScript inclusion shall
become standard.  Note that the main advantage of the
\.{\ttbackslash epsffile} macros
is that they can be adapted for whatever form of special eventually becomes
standard, and thus only minor modifications to that one file need to be
made, rather than revising an entire library of \TeX\ documents.

Most of these specials use a flexible key and value scheme:
\cmd{\ttbackslash special\char123psfile=filename.ps[key=value]*\char125}
\noindent
This will download the PostScript file called \.{filename.ps}
such that the current point will be the origin of the PostScript
\^{{\tt psfile}}
coordinate system.  The optional key/value assignments allow you
to specify transformations on the PostScript.

The possible keys are:\vskip\baselineskip

\centerline{\vbox{\halign{\tt#\hfil\quad&#\hfil\cr
hoffset	&The horizontal offset (default 0)\cr
voffset	&The vertical offset (default 0)\cr
hsize	&The horizontal clipping size (default 612)\cr
vsize	&The vertical clipping size (default 792)\cr
hscale	&The horizontal scaling factor (default 100)\cr
vscale	&The vertical scaling factor (default 100)\cr
angle	&The rotation (default 0)\cr
clip    &Enable clipping to the bounding box\cr}}}

The dimension parameters are all given in PostScript units.
The \.{hscale} and \.{vscale} are given in non-dimensioned percentage
units, and the rotation value is specified in degrees.
Thus
\cmd{\ttbackslash special\char123psfile=foo.ps hoffset=72 hscale=90
vscale=90\char125}
\noindent
will shift the graphics produced by file \.{foo.ps} right by one inch
and will draw it at 0.9 times normal size.
Offsets are given relative to the point of the special command, and are
unaffected by scaling or rotation.  Rotation is counterclockwise about the
origin.  The order of operations is to rotate the figure,
scale it, then offset it.

For compatibility with older PostScript drivers, it is possible to change
the units that \.{hscale} and \.{vscale} are given in.  This can be
done by redefining \.{@scaleunit}
\^{{\tt scaleunit}}
in \.{SDict}
\^{{\tt SDict}}
by a \TeX\ command such as
\cmd{\ttbackslash special\char123! /@scaleunit 1 def\char125}
\noindent
The \.{@scaleunit} variable, which is by default 100, is what \.{hscale}
and \.{vscale} are divided by to yield an absolute scale factor.

All of the methods for including graphics we have described
so far enclose the graphic in
a PostScript save/restore pair, guaranteeing that the figure will have
no effect on the rest of the document.
Another type of special command allows literal PostScript instructions to be
inserted without
enclosing them in this protective shield; users of this feature
are supposed to understand what they are doing (and they shouldn't change
the PostScript graphics state unless they are willing to take the
consequences).  This command can take many forms, because it has had a
tortuous history; any of the following will work:
\cmd{\ttbackslash special\char123ps:{\it text}\char125\\
\ttbackslash special\char123ps::{\it text}\char125\\
\ttbackslash special\char123ps::[begin]{\it text}\char125\\
\ttbackslash special\char123ps::[end]{\it text}\char125}
\noindent
(with longer forms taking precedence over shorter forms, when they are used).
Note that {\tt ps::} and {\tt ps::[end]} do not do any positioning, so
they can be used to continue PostScript literals started with {\tt ps:}
or {\tt ps::[begin]}.  There is also the command
\cmd{\ttbackslash special\char123ps:\ plotfile filename\char125}
\noindent
which will copy the commands from \.{filename}
verbatim into the output (but omitting lines that begin with \%).
An example of the proper use of literal specials can be found in the file
\.{rotate.tex} which makes it easy to typeset text turned 90 degrees.

To finish off this section, the following examples are presented without
explanation:
{\vskip0pt\parskip=0pt\begverb{`\$}
\def\rotninety{\special{ps:currentpoint currentpoint translate 90
rotate neg exch neg exch translate}}\font\huge=cmbx10 at 14.4truept
\setbox0=\hbox to0pt{\huge A\hss}\vskip16truept\centerline{\copy0
\special{ps:gsave}\rotninety\copy0\rotninety\copy0\rotninety
\box0\special{ps:grestore}}\vskip16truept$endverb}
\smallskip
{\def\rotninety{\special{ps:currentpoint currentpoint translate 90
rotate neg exch neg exch translate}}\font\huge=cmbx10 at 14.4truept
\setbox0=\hbox to0pt{\huge A\hss}\vskip16truept\centerline{\copy0
\special{ps:gsave}\rotninety\copy0\rotninety\copy0\rotninety
\box0\special{ps:grestore}}\vskip16truept}
\smallskip
{\vskip0pt\parskip=0pt\begverb{`\$}
\vbox to 2truein{\special{ps:gsave 0.3 setgray}\hrule height 2in
width\hsize\vskip-2in\special{ps:grestore}\font\big=cminch\big
\vss\special{ps:gsave 1 setgray}\vbox to 0pt{\vskip2pt
\line{\hss\hskip4pt NEAT\hss}\vss}\special{ps:0 setgray}%
\hbox{\raise2pt\line{\hss NEAT\hss}\special{ps:grestore}}\vss}$endverb}
\smallskip
\vbox to 2truein{\special{ps:gsave 0.3 setgray}\hrule height 2in
width\hsize\vskip-2in\special{ps:grestore}\font\big=cminch\big
\vss\special{ps:gsave 1 setgray}\vbox to 0pt{\vskip2pt
\line{\hss\hskip4pt NEAT\hss}\vss}\special{ps:0 setgray}%
\hbox{\raise2pt\line{\hss NEAT\hss}\special{ps:grestore}}\vss}

Some caveats are in order when using the above forms.  Make sure that
each \.{gsave} on a page is matched with a \.{grestore} on the same
page.  Do not use \.{save} or \.{restore}.  Use of these macros can
interact with the PostScript generated by \.{dvips} if care is not
taken; try to understand what the above macros are doing before
writing your own.  The \.{\ttbackslash rotninety} macro especially has
a useful trick that appears again and again.

\sub{Dynamic Creation of PostScript Graphics Files}

PostScript is an excellent page description language---but it does
tend to be rather verbose.  Compressing PostScript graphics files can
often reduce them by more than a factor of five.  For this reason,
if the filename parameter to one of the graphics inclusion techniques
starts with a backtick ({\tt`}), the filename is instead interpreted
as a command to execute that will send the actual file to standard
output.  Thus,
\cmd{\ttbackslash special\char123psfile="`zcat foo.ps.Z"\char125}
\noindent
will include the uncompressed version of \.{foo.ps}.  Since such a
command is not accessible to \TeX, if you use this facility with
the \.{EPSF} macros, you need to supply the bounding box parameter
yourself, as in
\cmd{\ttbackslash epsffile[72 72 540 720]\char123
   "`zcat screendump.ps.Z"\char125}
\noindent
to include \.{screendump.ps}.  Of course, the commands to be executed can
be anything, including using a file conversion utility such as \.{tek2ps}
or whatever is appropriate.

This extension is not portable to other \.{dvi2ps} programs.  Yet.

\sec{Using PostScript Fonts}

Thanks to Donald~E.~Knuth,
\^{Knuth, Donald E.}
the \.{dvips} driver now supports PostScript fonts
\^{PostScript fonts}
through the virtual font capability.
PostScript fonts are (or should be) accompanied by a font metric file
such as \.{Times-Roman.afm},
which describes characteristics of a font called Times-Roman.
\^{{\tt afm}}
To use such fonts with \TeX, we need \.{tfm} files that contain similar
\^{{\tt tfm}}
information. These can be generated from \.{afm} files
by running the program \.{afm2tfm}, supplied with \.{dvips}.
\^{{\tt afm2tfm}}
This program also creates virtual fonts with which you can use normal
plain \TeX\ conventions.

Note that non-resident downloaded PostScript fonts tend to use a
great deal of printer virtual memory.  PostScript printers typically
have between 130,000 and 400,000 bytes of available virtual memory;
each downloaded font will require approximately 30,000 bytes of
that.  For many applications, bitmapped fonts work much better,
even at typesetter resolutions (with the \.{-Z} option.)

Even resident PostScript fonts can take a fair amount of memory,
but less with this release of \.{dvips} than previously.  Also,
bitmapped fonts tend to image faster than PostScript fonts.

A few Type 1 fonts (such as Utopia, Charter, and Courier) have been
contributed by vendors to the X distribution, and are freely
available. You can get \TeX{} distributions for them from
{\tt ftp.cs.umb.edu} in {\tt pub/tex}, and from the CTAN hosts in
{\tt tex-archive/fonts}.

Your Unix system may have come with additional PostScript fonts.  If so,
you can make them available to Dvips by copying the files or making
symbolic links with the appropriate filenames, and running
{\tt afm2tfm} to make TFM and VF files so the fonts
will be available in the same encoding as the fonts distributed with
dvips. Also check {\tt psfonts.map} to be sure the fonts are listed
there.

Here are the typical locations for vendor-supplied fonts:
$$\vtop{\halign{#\hfil\quad&\tt #\hfil\cr
DEC Ultrix& /usr/lib/DPS/outline/decwin\cr
DEC OSF/1& /usr/lib/X11/fonts/Type1Adobe\cr
NeXT& /NextLibrary/Fonts/outline\cr
SGI IRIX& /usr/lib/DPS/outline/base\cr
Sun Solaris 2.3& /usr/openwin/lib/X11/fonts/Type1/outline\cr}}$$

\sub{The afm2tfm Program}

The \.{afm2tfm} program can convert an Adobe \.{afm} file into a `raw'
\TeX\ \.{tfm} file with the command
\cmd{localhost> afm2tfm Times-Roman rptmr}
\noindent
(You should run this from in a directory where \.{Times-Roman.afm} resides.)
The output file \.{rptmr.tfm} is `raw' because it does no character remapping;
it simply converts the character information on a one-to-one basis to \TeX\
characters {\it with the same code}. The name \.{rptmr} stands for
Resident PostScript Times Roman; section~6 below explains more about
a proposed scheme for font names.

In the following examples, we will use the font Times-Roman to
illustrate the conversion process.  For the standard 35 LaserWriter
fonts, however, it is highly recommended that you use the supplied
\.{tfm} and \.{vf} files that come with \.{dvips} (usually in a
file called \.{dvipslib.tar.Z}), as these files contain some additional
changes that make them work better with \TeX\ than they otherwise
would.

PostScript fonts have a different encoding scheme from that of plain
\TeX. Although both schemes are based on ASCII, special characters such as
ligatures and accents are handled quite differently. Therefore we obtain
\^{ligature}
\^{accents}
\^{virtual fonts}
best results by using a `virtual font' interface, which makes \TeX\ act
as if the PostScript font had a standard \TeX\ encoding. Such a virtual
font can be obtained, for example, by the command
\cmd{localhost> afm2tfm Times-Roman -v ptmr rptmr}
\noindent
This produces two outputs, namely the `virtual property list' file
\.{ptmr.vpl}, and the \TeX\ font metric file \.{rptmr.tfm}.
The latter file describes an `actual font' on which the virtual font is based.

To use the font in \TeX, you should first run
\cmd{localhost> vptovf ptmr.vpl ptmr.vf ptmr.tfm}
\noindent
and then install the virtual font file \.{ptmr.vf} in the virtual font
directory (by default, \.{/usr/lib/tex/fonts/vf}) and install \.{ptmr.tfm}
and \.{rptmr.tfm} in the directory for \TeX\ font metrics (by default,
\.{/usr/lib/tex/fonts/tfm}).  (This probably has already been done for you for
the most common PostScript fonts.)
You can also make more complex virtual fonts by editing
\.{ptmr.vpl} before running \.{vptovf}; such editing might add the uppercase
Greek characters in the standard \TeX\ positions, for instance.
Once this has been done, you're all set. You can use
code like this in \TeX\ henceforth:
\cmd{\ttbackslash font\ttbackslash myfont=ptmr at 10pt\\
\ttbackslash myfont Hello, I am being typeset in Times-Roman.}

Note that there are two fonts, one actual (`rptmr',
which is analogous to a raw piece of hardware) and one virtual
(`ptmr', which has typesetting know-how added). You could also say
\cmd{\ttbackslash font\ttbackslash TR=rptmr at 10pt}
\noindent
and typeset directly with that, but then you would have no ligatures
\^{ligature}
or kerning, and you would have to use Adobe
\^{kerning}
character positions for special letters like \AE. The virtual font
called \.{ptmr} not only has ligatures and kerning, and most of the
standard accent conventions of \TeX, it also has a few additional
features not present in the Computer Modern fonts. For example, it
includes all the Adobe characters (such as the Polish ogonek and the
French guillemots).  The only things you lose from ordinary \TeX\ text
fonts are the dotless j (which can be hacked into the VPL file with
literal PostScript specials if you have the patience) and uppercase
Greek letters (which just don't exist unless you buy them separately).
Experts may refer to Donald E.~Knuth article in {\it TUGboat} v. 11,
no. 1, Apr. 1990, pp. 13--23.  ``Virtual Fonts: More Fun for Grand
Wizards.''

When \.{dvips} goes to use a font, it first checks to see if it is one of
the fonts listed in a file called \.{psfonts.map}.  If it is mentioned in
\^{{\tt psfonts.map}}
that file, then it is assumed that the font is a resident PostScript font
and can be found with the PostScript \.{findfont} operator.  This file
resides by default in {\tt /usr/lib/tex/ps}, and consists of a single font
per line.  Note that only the raw PostScript font names should be listed
in this file; the \.{vf} fonts should not be, since they are automatically
mapped to the raw PostScript fonts by the virtual font machinery.
The supplied \.{psfonts.map} file defines the most common PostScript fonts.

As much as possible, the PostScript fonts follow plain \TeX\ conventions
for accents.  The two exceptions to this are the Hungarian umlaut (which
is at position {\tt 0x7D} in {\tt cmr10}, but position {\tt 0xCD} in
{\tt ptmr}) and the dot accent (at positions {\tt 0x5F} and {\tt 0xC7},
respectively).  In order to use these accents with PostScript fonts or in
math mode when {\tt \char92 textfont0} is a PostScript font, you will need
to use the following definitions.  Note that these definitions will not work
with normal \TeX\ fonts for the relevant accents; note also that these
definitions are already part of the distributed \.{psfonts.sty}.
In addition, the \.{\char92 AA} that is used to typeset the \AA\ character
must be modified as shown.

{\vskip0pt\parskip=0pt\leftskip=20pt\begverb{`\$}
\def\H#1{{\accent"CD #1}}\def\.#1{{\accent"C7 #1}}
\def\dot{\mathaccent"70C7 }
\newdimen\aadimen
\def\AA{\leavevmode\setbox0\hbox{h}\aadimen\ht0
  \advance\aadimen-1ex\setbox0\hbox{A}\rlap{\raise.67\aadimen
  \hbox to \wd0{\hss\char'27\hss}}A}$endverb}

These PostScript fonts can be scaled to any size.  Go wild!
Note, however, that using PostScript fonts does
use up a great deal of the printer's virtual memory and it does take time.
\^{memory}
You may find downloading the Computer Modern bitmapped fonts to be
faster than using the built-in PostScript fonts.

\sub{Changing a Font's Encoding}

The \.{afm2tfm} program also allows you to specify a different encoding
for a PostScript font.  This should only be done by wizards.
This can be done at two levels.

You can specify a different output encoding with \.{-o}. This only
applies when you are building a virtual font, and it tells \.{afm2tfm}
to attempt to remap the font to match the output encoding as closely
as possible.  In such an output encoding, you can also specify
ligature pairs and kerning information that will be used in addition
to the information in the \.{afm} file.  This will be the most common
re-encoding required, since the only thing that changes is the \.{vf}
file (and its associated \.{tfm} file) and since most everything you
would want to do can be done with this method.

You can also specify a different PostScript encoding with \.{-p}.  This
option affects the generation of both the raw \.{tfm} file and the
virtual \.{vf} and \.{tfm} files, and it also requires that the
encoding file be available to be downloaded as part of every
PostScript document produced that uses this font.  But this is the
only way to access characters in a PostScript font that are neither
encoded in the \.{afm} file nor built from other characters
(constructed characters.)  For instance, \.{Times-Roman} contains the
extra characters \.{registered} and \.{thorn} (among others) that
can only be accessed through such a PostScript reencoding.  Any ligature
or kern information specified in the PostScript encoding is ignored by
\.{afm2tfm}.

The format of the encoding files is simple---it is precisely the same
format that is used to define an encoding vector in PostScript!  For
this reason, the same file can be used as a PostScript or \TeX\
encoding file for \.{afm2tfm} as well as downloaded to the printer as
part of a document that uses a reencoded font.

The specific format that \.{afm2tfm} accepts is one of the following
form:
{\vskip0pt\parskip=0pt\begverb{`\\}
% comments are mostly ignored; more on this later
/MyEncoding [ /Alpha /Beta /Gamma /Delta ...
     /A /B ... /Z % exactly 256 entries, each with a / at the front
     /wfooaccent /xfooaccent /yfooaccent /zfooaccent ] def
\endverb}

Comments, which start with a percent sign and continue until the
end of the current line, are mostly ignored.  The first `word' of
the file must start with a forward slash (a PostScript literal
name) and is used as the name of the encoding.  The next word must
be an open bracket.  Following that must be precisely 256 character
names; use {\tt /.notdef} for any that you do not want to define.
Finally, there must be a close bracket.  The final token is usually
{\tt def}, but this is not enforced.  Note that all names are
case sensitive.

Any ligature or kern information is given in the comments.  As each
comment is encountered, it is examined.  If the first word after the
percent sign is {\tt LIGKERN}, exactly, then the entire rest of the
line is parsed for ligature and kern information.  This ligature and
kern information is given in groups of words, each group of which must
be terminated by a semicolon (with a space before and after it, unless
it occurs on the end of a line.)

In these {\tt LIGKERN} statements, three types of information may be
specified.  These three types are ligature pairs, kerns to remove or
ignore, and the character value of this font's boundary character.
Which of the types the particular set of words corresponds to is
automatically determined by the allowable syntax.

Throughout the {\tt LIGKERN} section, the boundary character is
specified as {\tt ||}.
To set the boundary character value, a command such as
{\tt || = 39 ;} must be used.

To indicate a kern to remove, give the names of the two characters
(without the leading slash) separated by {\tt \char123\char125}, as in
{\tt one \char123\char125\ one ;}.  This is similar to the way you
might use {\tt \char123\char125} in a \TeX\ file to turn off ligatures
or kerns at a particular location.  Either or both of the character
names can be given as {\tt *}, which is a wild card matching any
character; thus, all kerns can be removed with {\tt *
\char123\char125\ * ;}.

To specify a ligature, specify the names of the pair of characters,
followed by the ligature `operation' (as in \MF), followed by the
replacing character name.  Either (but not both) of the first two
characters can be {\tt ||} to indicate a word boundary.
Normally the `operation' is {\tt =:}
meaning that both characters are removed and replaced by the
third character, but by adding {\tt |} characters on either side of
the {\tt =:}, you can specify which of the two leading characters to
retain.  In addition, by suffixing the ligature operation with one
or two {\tt >} signs, you can indicate that the ligature scanning
operation should skip that many characters before proceeding.  This
works just like in \MF.  A typical ligature might be specified with
{\tt ff i =:{} ffi ;}.  A more convoluted ligature is
{\tt one one |=:|>> exclam ;} which indicates that every pair of
adjacent {\tt 1}'s should be separated by an exclamation point,
and then two of the resulting characters should be skipped over
before continuing searching for ligatures and kerns.  You cannot
give more {\tt >}'s in an ligature operation as you did {\tt |},
so there are a total of eight possible ligature operations:
\cmd{=: |=: |=:> =:| =:|> |=:| |=:|> |=:|>>}
\noindent
The default set of ligatures and kerns built in to \.{afm2tfm} can
be specified with:
{\vskip0pt\parskip=0pt\begverb{`\\}
% LIGKERN space l =: lslash ; space L =: Lslash ;
% LIGKERN question quoteleft =: questiondown ;
% LIGKERN exclam quoteleft =: exclamdown ;
% LIGKERN hyphen hyphen =: endash ; endash hyphen =: emdash ;
% LIGKERN quoteleft quoteleft =: quotedblleft ;
% LIGKERN quoteright quoteright =: quotedblright ;
% LIGKERN space {} * ; * {} space ; zero {} * ; * {} zero ;
% LIGKERN one {} * ; * {} one ; two {} * ; * {} two ;
% LIGKERN three {} * ; * {} three ; four {} * ; * {} four ;
% LIGKERN five {} * ; * {} five ; six {} * ; * {} six ;
% LIGKERN seven {} * ; * {} seven ; eight {} * ; * {} eight ;
% LIGKERN nine {} * ; * {} nine ;
\endverb}

\sub{Special Effects}

Special effects are also obtainable, with commands such as
\cmd{localhost> afm2tfm Times-Roman -s .167 -v ptmro rptmro}
\noindent
which create \.{ptmro.vpl} and \.{rptmro.tfm}.
To use this, proceed as above but put the line
\cmd{rptmro Times-Roman ".167 SlantFont"}
\noindent
into \.{psfonts.map}.  Then \.{rptmro} (our name for an obliqued Times)
will act as if it were a
\^{{\tt psfonts.map}}
resident font, although it is actually constructed from Times-Roman
by PostScript hackery.   (This oblique version of Times-Roman is obtained
by slanting everything 1/6 to the right.)
Similarly, you can get an expanded font by
\cmd{localhost> afm2tfm Times-Roman -e 1.2 -v ptmrre rptmrre}
\noindent
and by recording the pseudo-resident font
\cmd{rptmrre Times-Roman "1.2 ExtendFont"}
\noindent
in \.{psfonts.map}.

You can also create a small caps font with a command such as
\cmd{localhost> afm2tfm Times-Roman -V ptmrc rptmr}
\noindent
This is done strictly with a virtual font, however.  In addition,
the font on which the small caps font is based (in this case
{\tt rptmr} may already be created and installed, in which case no
additional {\tt psfonts.map} entry is needed.
In any case, you must give the appropriate name of the font that
is not small caps as the base name (last parameter) to {\tt afm2tfm}.
For instance, if you create a slanted small caps font, you must
give the base name of the raw slanted font as that last parameter,
not the base name of the unslanted font.

By default, the {\tt -V} option uses a font scaled to 80\% for
lower case.  If you specify the {\tt -c} option, you can change
this scaling.

If you change the PostScript encoding of a font, you must specify the
input file as a header file, as well as give a reencoding
command.  For instance, let us say we are using Times-Roman
reencoded according to the encoding {\tt MyEncoding} (stored
in the file {\tt myenc.enc}) as {\tt rptmrx}.  In this case,
our {\tt psfonts.map} entry would look like
\cmd{rptmrx Times-Roman "MyEncoding ReEncodeFont" <myenc.enc}

The \.{afm2tfm} program prints out the precise line you need to
add to \.{psfonts.map} to use that font, assuming the font is
resident in the printer; if the font is not resident, you must
add the header command to download the font yourself.  Note that
each identical line only needs to be specified once in the
{\tt psfonts.map} file, even though many different fonts (small
caps variants, or ones with different output encodings) may be
based on it.

The command line switches to \.{afm2tfm} are:

{\options
\.{-e {\it ratio}}
All characters are stretched horizontally by the stated ratio;
\^{expanded fonts}
if it is less than 1.0, you get a condensed font.

\.{-c {\it scale}}
If this option is given when creating a small caps font
(with {\tt -V}), then the scaling for the `lower'
case will be changed from the default 0.8 to the fraction
given here.

\.{-O}
This option forces all character designations in the resultant
{\tt vpl} file be given as octal values; this is useful for
symbol or other special-purpose fonts where character names such
as `A' have no meaning.

\.{-p {\it file}}
This specifies a file to use for the PostScript encoding of the
font.  Note that this file must also be mentioned as a header
file for the font in {\tt psfonts.map}, and that ligature
and kern information in this file is ignored.

\.{-s {\it slant}}
All characters are slanted to the right by the stated slant;
\^{slanted fonts}
if it is negative, the letters slope to the left (or they might be
upright if you start with an italic font).

\.{-t {\it file}}
This specifies a file to use for the target \TeX\ encoding of the
font.  Ligature and kern information may also be specified in this
file; the file is not needed once the \.{vf} file has been created.

\.{-T {\it file}}
This option specifies that {\it file} is to be used for both
the PostScript and target \TeX\ encodings of the font.

\.{-u}
This option indicates that \.{afm2tfm} should use only those
characters that are required by the output encoding, and no
others.  Normally, \.{afm2tfm}
tries to include both characters that fit the output encoding
and any additional characters that might exist in the font.
This option forbids those additional characters from being
added.

\.{-v {\it file}}
Generate a virtual property list \.{vpl} file as well as a \.{tfm} file.

\.{-V {\it file}}
Same as \.{-v}, but the virtual font generated is a small caps font obtained
by scaling uppercase letters by 0.8 to typeset lowercase. This font
handles accented letters and retains proper kerning.\par}

\sub{Non-Resident PostScript Fonts}

If you want to use a non-printer-resident PostScript font for which you have
a \.{pfb} or \.{pfa} file (an Adobe Type 1 font program),
you can make it act like
\^{{\tt pfb}}
a resident font by putting a `{\tt <}' sign
and the name of the \.{pfb} or \.{pfa} file just after the font name in the
\.{psfonts.map}
\^{{\tt psfonts.map}}
file entry. For example,
\cmd{rpstrn StoneInformal <StoneInformal.pfb}
\noindent
will cause \.{dvips} to include \.{StoneInformal.pfb} in your document as if
it were a header file, whenever the pseudo-resident font StoneInformal is
used in a document.   Similarly, you can generate transformed fonts and
include lines like
\cmd{rpstrc StoneInformal <StoneInformal.pfb ".8 ExtendFont"}
\noindent
in \.{psfonts.map}, in which case \.{StoneInformal.pfb} will be loaded
whenever StoneInformal-Condensed is used. (Each header file is loaded
at most once per PostScript file. The \.{pfb} files should be
installed in the \.{dvips} header directory [usually {\tt
/usr/lib/tex/ps}] with the other header files.)

If you are using a {\tt pfb} file that has different PostScript
encodings, you would need to
multiple header files for that font in {\tt psfonts.map}.  If, for instance,
{\tt StoneInformal} was both non-resident and you wanted to reencode
it in PostScript with {\tt MyEncoding} stored in {\tt myenc.enc}, a line
such as
\vskip\baselineskip
\centerline{\tt rpstrnx StoneInformal "MyEncoding ReEncodeFont" <myenc.enc
<StoneInformal.pfb}

When using such files, \.{dvips} is smart enough to unpack the standard
binary \.{pfb} format into ASCII so there is no need to perform this
conversion yourself.  In addition, it will scan the font to determine
its memory usage, as it would for any header file.

\sub{Font Aliases}
Some systems don't handle files with long names well---MS-DOS is
\^{MS-DOS}
a notable example.  For this reason, \.{dvips} will accept an alias for such
fonts.  Such an alias should be the first word in the \.{psfonts.map}
line.  For instance, if we wanted the name \.{rptmr} to be used for the
raw \.{Times-Roman} since our computer can't handle long names or,
alternatively, we want to follow the standard naming conventions, we would use
the following line in our
\.{psfonts.map} file:
\cmd{rptmr Times-Roman}
\noindent
The \.{tfm} file must have the name \.{rptmr.tfm}.
\^{{\tt tfm}}

The distribution file \.{adobe} contains a list of the short names that
should be used for most Adobe fonts currently available.  Please reference
this file when installing a new font and use the standard name.

The parsing of the \.{psfonts.map}
\^{{\tt psfonts.map}}
file should be explained to eliminate all confusion.  If a line is
empty or begins with a space, asterisk, semicolon, or hash mark, it is
ignored.  Each remaining line is separated into words, where words are
separated by spaces or tabs.  If a word begins with a double quote, it
extends until the next double quote or the end of the line.  If a word
starts with a less than character, it is treated as a font header file
(or a downloaded PostScript font).  There can be more than one such
header for a given font.  If a word starts with a double quote, it is
a special instruction for generating that font.  Otherwise it is a
name.  The first such name is always the name \TeX\ uses for the font
and is also the name of the raw \.{tfm} file.  If there is another
name word, that name is used as the PostScript name; if there is only
one name word, it is used for both the \TeX\ name and the PostScript
name.

Note that \.{dvips} no longer registers the full PostScript name if an
alias is given, so the single line
\cmd{rptmr Times-Roman}
\noindent
would only allow \.{dvips} to find the \.{rptmr} font and not the
\.{Times-Roman} font.

\sec{Font Naming Conventions}

{\def\journal#1{{\sl #1}}
\def\table{\vbox\bgroup
\vskip\parskip\halign\bgroup\strut\indent\tt ##\hfil\quad
    &\vtop{%
       \advance\hsize by -\parindent % The \indent in the first column.
       \advance\hsize by -1em        % The \quad.
       % One table has two-letter abbreviations.
       \advance\hsize by -2\fontdimen7\tentt
       \rm\noindent ##}\hfil
    \cr}
\def\setuphsize{%
       \advance\hsize by -\parindent % The \indent in the first column.
       \divide\hsize by 2            % double column
       \advance\hsize by -1.5em      % The 3 \quads divided by 2.
       % One table has two-letter abbreviations.
       \advance\hsize by -2\fontdimen7\tentt}
\def\dtable{\vbox\bgroup\vskip\parskip\halign\bgroup\strut
    \indent##&\tt ##\hfil\quad
    &\vtop{\setuphsize\rm\noindent ##}\hfil&\quad\tt ##\hfil\quad
    &\vtop{\setuphsize\rm\noindent ##}\hfil\cr}
\def\endtable{%
  \egroup\egroup % End the \halign.
  \smallskip % I don't know what kind of space you want, but it needs
             %  something here.
}
\def\entry#1#2{#1&#2\cr}
\def\mitem{\item{$\bullet$}}
\def\\{\hfil\break}

This section of the manual has been written by Karl Berry and
specifies a standard for naming fonts for \TeX.  This standard has
been adopted in \.{dvips}, and it is recommended that it be followed
where possible.

As more typeface families become available for use with \TeX, the need
for a consistent, rational naming scheme for the font filenames
concomitantly grows. Some (electronic) discussion has gone into the
following proposal; I felt it was appropriate now to bring it before a
wider community.  In some respects, it follows and simplifies
Mittelbach and Sch\"opf's article in \journal{TUGboat}, volume~11,
number~2 (June 1990).

Here are some facts about fonts that went into the hopper when creating
this proposal:

\mitem \TeX\ runs on virtually all computers, under almost as many operating
systems, all with their own idea of how files should be named.  Any
proposal regarding filenames, therefore, must cater to the lowest common
denominator.  That seems to be eight characters in length, not counting any
extension, and with case being insignificant.  Characters other
than letters and numerals are probably unusable.

\mitem Most typefaces are offered by several vendors.  The version
offered by vendor~A is not compatible with that of vendor~B.

\mitem Typefaces typically come in different weights (hairline to extra
heavy), different expansions (ultra condensed to wide), and an
open-ended range of variants (italic, sans serif, typewriter,
shadow,~$\ldots$).  No accepted standards exist for any of these
qualities, nor are any standards ever likely to gain acceptance.

\mitem The Computer Modern typeface family preserves traditional typesetting
practice in at least one important respect: different sizes of the same
font are not scaled linearly.  This is in contrast to most commercial fonts
available.

Here is how I propose to divide up the eight characters:
\cmd{FTTWVEDD}
\noindent
where

\mitem \.{F} represents the foundry that produced the font, and is omitted
if there isn't one.

\mitem \.{TT} represents the typeface name.

\mitem \.W represents the weight.

\mitem \.V represents the variant, and is omitted if both it and the
expansion are ``normal''.

\mitem \.E represents the expansion, and is omitted if it is ``normal''.

\mitem \.{DD} represents the design size, and is omitted if the font is
linearly scaled from a single {\tt tfm} file.

See the section on virtual fonts (towards the end) for an exception to
the above.

The weight, variant, and expansion are probably all best taken from the
original source of the typeface, instead of trying to relate them to
some external standard.

Before giving the lists of abbreviations, let me point out two problems,
to neither of which I have a good solution.  1)~Assuming that only the
English letters are used, two letters is enough for only 676 typeface
families (even assuming we want to use all possible combinations, which
is doubtful).  There are many more than 676 typeface families in the
world.  2)~Fonts with design sizes over 100$\,$pt are not common, but
neither are they unheard of.

On to the specifics of the lists.  If you adopt this proposal at your
own installation, and find that you have fonts with some property I
missed, please write to me (see the end of the article for various
addresses), so I can update the lists.  You can get the
most up-to-date version of these lists electronically, by anonymous ftp
from the host \.{ftp.cs.umb.edu}.  I will also send them to you by
electronic mail, if necessary.

I give the letters in lowercase, which is preferred on systems where
case is significant.  Most lists are in alphabetical order by the
abbreviations.

\sub{Foundry}

This is the current list of foundries.

\table
a&Autologic\cr
b&Bitstream\cr
c&Agfa-Compugraphic\cr
g&Free Software Foundation ({\tt g} for GNU)\cr
h&Bigelow \& Holmes (with apologies to Chuck)\cr
i&International Typeface Corporation\cr
p&Adobe ({\tt p} for PostScript)\cr
r&reserved for use with virtual fonts; see below\cr
s&Sun\cr
\endtable

\sub{Typeface Families}

The list of typefaces is:

\dtable
&ad&Adobe Garamond&go&Goudy Oldstyle\cr
&ag&Avant Garde&gs&Gill Sans\cr
&ao&Antique Olive&jo&Joanna\cr
&at&American Typewriter&lc&Lucida\cr
&bb&Bembo&lt&Lutetia\cr
&bd&Bodoni&nc&New Century Schoolbook\cr
&bg&Benguiat&op&Optima\cr
&bk&Bookman&pl&Palatino\cr
&bl&Balloon&pp&Perpetua\cr
&bv&Baskerville&rw&Rockwell\cr
&bw&Broadway&st&Stone\cr
&cb&Cooper Black&sy&Symbol\cr
&cl&Cloister&tm&Times\cr
&cr&Courier&un&Univers\cr
&cn&Century&uy&University\cr
&cs&Century Schoolbook&zc&Zapf Chancery\cr
&hv&Helvetica&zd&Zapf Dingbats\cr
&gm&Garamond&&\cr
\endtable

\sub{Weight}

This is a list of the possible weights,
roughly in order from lightest to heaviest.

\dtable
&a&hairline&d&demi\cr
&t&thin&s&semi\cr
&i&extra light&b&bold\cr
&l&light&x&extra bold\cr
&k&book&h&heavy\cr
&r&regular&c&black\cr
&m&medium&u&ultra\cr
\endtable

\sub{Variants}

The variants are:

\dtable
&a&alternate&n&informal\cr
&b&bright&o&oblique (i.e., slanted)\cr
&c&small caps&r&normal (roman or sans)\cr
&e&engraved&s&sans serif\cr
&g&grooved (as in the IBM logo)&t&typewriter\cr
&h&shadow&u&unslanted italic\cr
&i&(text) italic&x&expert\cr
&l&outline&&\cr
\endtable

If the variant is {\tt r}, and the expansion is also normal, both the
variant and the expansion are omitted.  When the normal version of the
typeface is sans serif (e.g., Helvetica), \.{r} should be used,
not \.{s}.  Use \.{s} only when the typeface family
has both serif and sans serif variants.  The ``alternate'' variant
(\.{a}) is used by some Adobe fonts that have spiffy swashes and
additional ligatures.
The ``expert'' variant (\.{x}) is also used by some Adobe
fonts with oldstyle figures and small caps.

Some fonts have multiple variants; Stone Informal Italic, for example.
The only reasonable approach to these is to list all the letters for all
the variants, choosing one to end with that is not also an expansion
letter.  Of course, it is possible that the name will become too
long if you do this, but $\ldots$ well, I'm open to suggestions.  It's
also possible the name will be ambiguous, if some new letter is used for
expansions in the future.  You can avoid this problem by adding the
expansion \.{r} (if it doesn't make the name too long), and
never using \.{r} for the last variant.

\sub{Expansion}

This is a list of the possible expansions, in order from narrowest to
widest.

\dtable
&o&extra condensed&x&extended (by hand)\cr
&c&condensed (by hand)&e&expanded (automatic)\cr
&n&narrow (automatic)&w&wide\cr
&r&\multispan3{regular, normal, medium (usually omitted)\hfil}\cr
\endtable

Expansion of fonts is sometimes done automatically (as in PostScript
{\bf scale}), and sometimes done by humans.  I chose `narrow' and
`expanded' to imply the former, and `condensed' and `extended' to imply
the latter, as I believe this reflects common usage.

\sub{Naming Virtual Fonts}

In concert with releasing \TeX~3.0 and \MF~2.0, Don Knuth wrote two new
utility programs: \.{VFtoVP} and \.{VPtoVF}, which convert
to and from ``virtual'' fonts.  Virtual fonts provide a general
interface between the writers of \TeX\ macros and font suppliers.
In general, therefore, it is impossible to come up with a general scheme
for naming virtual fonts, since each virtual font is an individual
creation, possibly bringing together many unrelated fonts.

Nevertheless, one common case is to use virtual fonts to map \TeX's
default accent and other character code conventions onto a
vendor-supplied font.  For example, \.{dvips}
does this for fonts given in the PostScript ``standard encoding''.
In this case, each font consists of a ``virtual'' \.{tfm} file, which is
what \TeX\ uses, a ``raw'' \.{tfm} file, which corresponds to the actual
device font, and a \.{vf} file, which describes the relationship between the
two.

This adds another dimension to the space of font names, namely,
``virtualness'' (or rather, ``rawness'', since it is the virtual \.{tfm}
files that the users want to see).  But we have already used up all
eight characters in the font names.

The best solution I have been able to think of is this: prepend
\.{r} to the raw \.{tfm} files; the virtual \.{tfm} files should be
named with the usual foundry prefix.  For example, the virtual
Times-Roman tfm file is named \.{ptmr}, as usual; the raw Times-Roman
\.{tfm} file is named \.{rptmr}.  To prevent intolerable confusion, I
promise never to give a foundry the letter \.{r}.

This scheme will work only as long as the virtualized fonts do not have
design sizes; if they do, another foundry letter will have to be
allocated, it seems to me.

A pox upon the houses of those who decided on fixed-length filenames!

\sub{Examples}

In closing, I will give two examples.  First, the fonts in the Univers
typeface family were assigned numbers by its designer, Adrien Frutiger. 
(You can see the scheme on, for example, page~29 of {\sl The Art of
Typo.icon.ography}, by Martin Solomon.)  Naturally, we want to give
them names.

\vskip\baselineskip\vbox{%
\halign to\hsize{\hskip20pt\tt #\hfil\space&#\hfil\quad\tabskip=0pt plus 1in
&\tt #\hfil\space\tabskip=0pt&#\hfil\cr
unl&45 (light)&unmro&59 (medium extra condensed)\cr
unli&46 (light italic)&undrx&63 (demibold extended)\cr
unlrc&47 (light condensed)&und&65 (demibold)\cr
unlic&48 (light condensed italic)&undi&66 (demibold italic)\cr
unlro&49 (light extra condensed)&undrc&67 (demibold condensed)\cr
unmrx&53 (medium extended)&undic&68 (demibold condensed italic)\cr
unm&55 (medium)&unbrx&73 (bold extended)\cr
unmi&56 (medium italic)&unb&75 (bold)\cr
unmrc&57 (medium condensed)&unbi&76 (bold italic)\cr
unmic&58 (medium condensed italic)&unxrx&83 (extra bold extended)\cr}}

Second, here are names for the standard PostScript fonts and their variants:
Fonts marked by an asterisk do not require using virtual fonts; the raw
fonts can be used directly because no remapping is necessary; every
character is encoded.
\vskip\baselineskip\vbox
{\font\small=cmr8\def\mark{${}^*$}\def\smallcaps{{\small(Small Caps)}}
\hyphenation{Post-Script}
\halign to\hsize{\hskip20pt\tt #\hfil\space&#\hfil\quad\tabskip=0pt plus 1in
&\tt #\hfil\space\tabskip=0pt&#\hfil\cr
pagk  &AvantGarde-Book&pncri &NewCenturySchlbk-Italic\cr
pagkc &AvantGarde-Book {\smallcaps}&pncr  &NewCenturySchlbk\cr
pagko &AvantGarde-BookOblique&pncrc &NewCenturySchlbk {\smallcaps}\cr
pagd  &AvantGarde-Demi&pplb  &Palatino-Bold\cr
pagdo &AvantGarde-DemiOblique&pplbi &Palatino-BoldItalic\cr
pbkd  &Bookman-Demi&pplbu &Palatino-BoldUnslanted\cr
pbkdi &Bookman-DemiItalic&pplrrn&Palatino-Narrow\cr
pbkl  &Bookman-Light&pplrre&Palatino-Expanded\cr
pbkli &Bookman-LightItalic&pplri &Palatino-Italic\cr
pbklc &Bookman-Light {\smallcaps}&pplr  &Palatino\cr
pcrb  &Courier-Bold&pplro &Palatino-Oblique\cr
pcrbo &Courier-BoldOblique&pplru &Palatino-Unslanted\cr
pcrro &Courier-Oblique&pplrc &Palatino {\smallcaps}\cr
pcrr  &Courier&psyr  &Symbol\mark\cr
phvb  &Helvetica-Bold&psyro &Symbol-Oblique\mark\cr
phvbo &Helvetica-BoldOblique&ptmb  &Times-Bold\cr
phvro &Helvetica-Oblique&ptmbi &Times-BoldItalic\cr
phvr  &Helvetica&ptmrrn&Times-Narrow\cr
phvrc &Helvetica {\smallcaps}&ptmrre&Times-Expanded\cr
phvbrn&Helvetica-Narrow-Bold&ptmri &Times-Italic\cr
phvbon&Helvetica-Narrow-BoldOblique&ptmro &Times-Oblique\cr
phvron&Helvetica-Narrow-Oblique&ptmr  &Times-Roman\cr
phvrrn&Helvetica-Narrow&ptmrc &Times-Roman {\smallcaps}\cr
pncb  &NewCenturySchlbk-Bold&pzcmi &ZapfChancery-MediumItalic\cr
pncbi &NewCenturySchlbk-BoldItalic&pzdr  &ZapfDingbats\mark\cr}}

Please contact Karl Berry if you have any comments or additions.  Karl
can be reached at \.{karl@cs.umb.edu}, or at 135 Center Hill Road,
Plymouth, MA~~02360.\par}

\sec{Command Line Options}

The \.{dvips} driver has a plethora of command line options.  Reading
through this section will give a good idea of the capabilities of the
driver.

Many of the parameterless options listed here can be turned off by
immediately suffixing
the option with a zero (0); for instance, to turn off page reversal if it is
turned on by default, use \.{-r0}.  The options that can be turned off in
this way are \.a, \.f, \.k, \.i, \.m, \.q, \.r, \.s, \.E, \.F, \.K, \.M, \.N,
\.U, and \.Z.

This is a handy summary of the options; it is printed out when you run
\.{dvips} with no arguments.
{\vskip0pt\parskip=0pt\begverb{`\\}
     Usage: dvips [options] filename[.dvi]
 a*  Conserve memory, not time      y # Multiply by dvi magnification
 b # Page copies, for posters e.g.  A   Print only odd (TeX) pages
 c # Uncollated copies              B   Print only even (TeX) pages
 d # Debugging                      C # Collated copies
 e # Maxdrift value                 D # Resolution
 f*  Run as filter                  E*  Try to create EPSF
 h f Add header file                F*  Send control-D at end
 i*  Separate file per section      K*  Pull comments from inclusions
 k*  Print crop marks               M*  Don't make fonts
 l # Last page                      N*  No structured comments
 m*  Manual feed                    O c Set/change paper offset
 n # Maximum number of pages        P s Load config.$s
 o f Output file                    R   Run securely
 p # First page                     S # Max section size in pages
 q*  Run quietly                    T c Specify desired page size
 r*  Reverse order of pages         U*  Disable string param trick
 s*  Enclose output in save/restore V*  Send downloadable PS fonts as PK
 t s Paper format                   X # Horizontal resolution
 x # Override dvi magnification     Y # Vertical resolution
                                    Z*  Compress bitmap fonts
     # = number   f = file   s = string  * = suffix, `0' to turn off
     c = comma-separated dimension pair (e.g., 3.2in,-32.1cm)
\endverb}

{\options
\def\o#1 #2:{{\tt #1} {\it #2}:}
\def\O#1:{{\tt #1}:}
\O-a:  Conserve memory by making three passes over the \.{dvi} file
instead of two and only loading those characters actually used.
Generally only useful on machines with a very limited amount of
memory, like some PCs.

\o-b num: Generate {\it num} copies of each page, but duplicating the
page body rather than using the {\tt\#numcopies} option.  This can be
useful in conjunction with a header file setting {\tt\char92bop-hook}
to do color separations or other neat tricks.

\o-c num: Generate {\it num} copies of every page, by using
PostScript's \.{\#copies} feature.
\^{copies}
Default is 1. (For collated copies, see the \.{-C} option below.)

\o-d num:  Set the debug flags.  This is intended only for emergencies
or for unusual fact-finding expeditions; it will work only if
\^{debug}
\.{dvips} has been compiled with the \.{DEBUG} option.
The source file \.{debug.h} indicates what the values of
{\it num} can be, or see section 15 of this manual.
Use a value of $-1$ for maximum output.

\o-e num:
Make sure that each character is placed at most this many pixels from its
`true' resolution-independent
position on the page. The default value of this parameter
is resolution dependent (it is the number
of entries in the list [100, 200, 300, 400, 500, 600, 800, 1000, 1200, 1600,
2000, 2400, 2800, 3200, $\ldots\,$] that are less than or equal to the
\^{maxdrift}
resolution in dots per inch). Allowing individual
characters to `drift' from their correctly rounded positions
by a few pixels, while regaining the true position at the beginning of
each new word, improves the spacing of letters in words.

\O-f: Run as a filter.
\^{filter}
Read the \.{dvi} file from standard input and write the PostScript to
standard output.  The standard input must be seekable, so it cannot
be a pipe.  If you must use a pipe, write a shell script that copies
the pipe output to a temporary file and then points \.{dvips} at this file.
This option also disables the automatic reading of the \.{PRINTER}
environment variable, and turns off the automatic sending of control D
if it was turned on with the \.{-F} option or in the configuration file;
use \.{-F} after this option if you want both.

\o-h name:
Prepend file {\it name}
\^{header}
as an additional header file. (However, if the name is simply `\.-',
suppress all header files from the output.)  This header file
gets added to the PostScript \.{userdict}.

\O-i: Make each section be a separate file.  Under certain circumstances,
\.{dvips} will split the document up into `sections' to be processed
independently; this is most often done for memory reasons.  Using this
option tells \.{dvips} to place each section into a separate file; the
new file names are created replacing the suffix of the supplied output
file name by a three-digit sequence number.
This option is most often used in
conjunction with the \.{-S} option which sets the maximum section length
in pages.  For instance, some phototypesetters cannot print more than
ten or so consecutive pages before running out of steam; these options
can be used to automatically split a book into ten-page sections, each
to its own file.

\O-k:  Print crop marks.  This option increases the paper size
(which should be specified, either with a paper size special or
with the {\tt -T} option) by a half inch in each dimension.  It
translates each page by a quarter inch and draws cross-style
crop marks.  It is mostly useful with typesetters that can set
the page size automatically.

\o-l num:
\^{pages}
The last page printed will be the first one numbered {\it num.}
Default is the last page in the document.  If the {\it num} is
prefixed by an equals sign, then it (and any argument to the
{\tt -p} option) is treated as a sequence number, rather than
a value to compare with {\tt \char92 count0} values.  Thus,
using {\tt -l =9} will end with the ninth page of the document,
no matter what the pages are actually numbered.

\O-m:  Specify manual feed for printer.
\^{manual feed}

\o-n num:
At most {\it num} pages will be printed. Default is 100000.
\^{pages}

\o-o name:  The output will be sent to file {\it name.}
\^{output}
If no file name is given, the default name is \.{{\it file}.ps} where
the \.{dvi} file was called \.{{\it file}.dvi};
if this option isn't given, any default in the configuration file is used.
If the first character of the supplied output file name is an
exclamation mark, then
the remainder will be used as an argument to \.{popen}; thus, specifying
\.{!lpr} as the output file will automatically queue the file for printing.
This option also disables the automatic reading of the \.{PRINTER}
environment variable, and turns off the automatic sending of control D
if it was turned on with the \.{-F} option or in the configuration file;
use \.{-F} after this option if you want both.

\o-p num:
\^{pages}
The first page printed will be the first one numbered {\it num.}
Default is the first page in the document.  If the {\it num} is
prefixed by an equals sign, then it (and any argument to the {\tt -l}
option) is treated as a sequence number, rather than a value to
compare with {\tt \char92 count0} values.  Thus, using {\tt -p =3}
will start with the third page of the document, no matter what the
pages are actually numbered.  Another form of page selection is
available by using {\tt -pp} followed by a comma-separated list of
pages or page-ranges, where the page ranges are colon-separated pairs
of numbers.  Thus, you can print pages 3--10, 21, and 73--92 with the
option {\tt -pp 3:10,21,73:92}.

\O-q: Run in quiet mode.
Don't chatter about pages converted, etc.; report nothing but errors to
\^{quiet}
standard error.

\O-r:
Stack pages in reverse order.  Normally, page 1 will be printed first.
\^{reverse}

\O-s:
Causes the entire global output to be enclosed in a save/restore pair.
This causes the file to not be truly conformant, and is thus not recommended,
but is useful if you are driving the printer directly and don't care too
much about the portability of the output.

\o-t papertype:
\^{paper type}
\^{mode}
\^{letter}
\^{legal}
\^{ledger}
\^{a4}
\^{a3}
\^{landscape}
This sets the paper type to {\it papertype}.  The {\it papertype} should
be defined in one of the configuration files, along with the appropriate
code to select it.  See the documentation for \.@ in the configuration
file option descriptions.  You can also specify
\.{-t landscape}, which rotates a document by 90 degrees.
To rotate a document whose size is not letter, you can use the 
\.{-t} option twice, once for the page size, and once for \.{landscape}.
The upper left corner of each page in
the \.{dvi} file is placed one inch from the left and one inch from the top.
Use of this option is highly dependent on the configuration file.
Note that executing the {\tt letter} or {\tt a4} or other PostScript
operators cause the document to be nonconforming and can cause it not
to print on certain printers, so the default paper size should not
execute such an operator if at all possible.

\o-x num:
Set the magnification ratio to {\it num}/1000. Overrides the magnification
\^{magnification}
specified in the \.{dvi} file.  Must be between 10 and 100000.  It is
recommended that you use standard magstep values (1095, 1200, 1440, 1728,
2074, 2488, 2986, and so on) to help reduce the total number of \.{PK}
files generated.

\O-A:
This option prints only the odd pages.  This option uses the \TeX\
page numbering rather than the sequence page numbers.

\O-B:
This option prints only the even pages.  This option uses the \TeX\
page numbering rather than the sequence page numbers.

\o-C num:
Create {\it num}
\^{copies}
copies, but collated (by replicating the data in the PostScript file).
Slower than the \.{-c} option, but easier on the hands, and faster than
resubmitting the same PostScript file multiple times.

\o-D num:
Set the resolution in dpi (dots per inch) to {\it num.}
\^{resolution}
This affects the choice of bitmap fonts that are loaded and also the positioning
of letters in resident PostScript fonts. Must be between 10 and 10000.
This affects both the horizontal and vertical resolution.  If a high resolution
(something greater than 400 dpi, say) is selected, the \.{-Z} flag should
probably also be used.

\O-E:
Makes \.{dvips} attempt to generate an EPSF file with a tight bounding
box.  This only works on one-page files, and it only looks at marks made
by characters and rules, not by any included graphics.  In addition, it
gets the glyph metrics from the {\tt tfm} file, so characters that
lie outside their enclosing {\tt tfm} box may confuse it.  In addition,
the bounding box might be a bit too loose if the character glyph has
significant left or right side bearings.  Nonetheless, this option works
well for creating small EPSF files for equations or tables or the like.
(Note, of course, that {\tt dvips} output is resolution dependent and
thus does not make very good EPSF files, especially if the images are
to be scaled; use these EPSF files with a great deal of care.)

\O-F:
Causes Control-D (ASCII code 4) to be appended as the very last character
of the PostScript file.  This is useful when \.{dvips}
\^{EOF}
\^{control-D}
is driving the printer directly instead of working through a spooler,
as is common on extremely small systems.  Otherwise, it is not recommended.

\O-K:
This option causes comments in included PostScript graphics, font files,
and headers to be removed.  This is sometimes necessary to get around bugs
in spoolers or PostScript post-processing programs.  Specifically, the
\.{\%\%Page} comments, when left in, often cause difficulties.
Use of this flag can cause some included graphics to fail, since the
PostScript header macros from some software packages read portions of
the input stream line by line, searching for a particular comment.
This option has been turned on by default because PostScript previewers
and spoolers still have problems with the structuring conventions.

\O-M:
Turns off the automatic font generation facility.  If any fonts are
missing, commands to generate the fonts are appended to the file
{\tt missfont.log} in the current directory; this file can then be
executed and deleted to create the missing fonts.

\O-N:
Turns off structured comments; this might be necessary on some systems
\^{structured comments}
that try to interpret PostScript comments in weird ways, or on some
PostScript printers.  Old versions of TranScript in particular cannot
handle modern Encapsulated PostScript.

\o-O offset:
Move the origin by a certain amount.  The {\it offset} is a comma-separated
pair of dimensions, such as \.{.1in,-.3cm} (in the same syntax used in
the \.{papersize} special).   The origin of the page is shifted from the
default position (of one inch down, one inch to the right from the upper
left corner of the paper) by this amount.

\o-P printername:
Sets up the output for the appropriate printer.  This is implemented
by reading in \.{config.{\it printername}}, which can then set the output pipe
(as in, \.{o !lpr -Pprintername}) as well as the font paths and any other
\^{{\tt config.ps}}
defaults for that printer only.  It is recommended that all standard
defaults go in the one master {\tt config.ps}
file and only things that vary
printer to printer go in the \.{config.{\it printername}}
files.  Note that {\tt config.ps}
is read before \.{config.{\it printername}}.
In addition, another file called \.{\tilde/.dvipsrc}
is searched for immediately after \.{config.ps};
this file is intended for user defaults.  If no \.{-P} command is
given, the environment variable \.{PRINTER} is checked.  If that
variable exists, and a corresponding configuration
file exists, that configuration file is read in.

\o-S num:
Set the maximum number of pages in each `section'.  This option is most
commonly used with the \.{-i} option; see that documentation above for more
information.

\O-T offset:
Set the paper size to the given pair of dimensions.  This option takes
its arguments in the same style as {\tt -O}.  It overrides any paper
size special in the {\tt dvi} file.

\O-U:
Disable a PostScript virtual memory saving optimization that stores the
character metric information in the same string that is used to store
the bitmap information.  This is only necessary when driving the Xerox
4045 PostScript interpreter.  It is caused by a bug in that interpreter
that results in `garbage' on the bottom of each character.  Not
recommended unless you must drive this printer.

\O-V:
Download non-resident PostScript fonts as bitmaps.  This requires use
of {\tt mtpk} or {\tt pstopk} or some combination of the two in order
to generate the required bitmap fonts; neither of these programs are
supplied with {\tt dvips}.

\o-X num:
Set the horizontal resolution in dots per inch to {\it num.}
\^{resolution}

\o-Y num:
Set the vertical resolution in dots per inch to {\it num.}
\^{resolution}

\O-Z:
Causes bitmapped fonts to be compressed before they are downloaded,
\^{compression}
thereby reducing the size of the PostScript font-downloading information.
Especially useful at high resolutions or when very large fonts are
used.  Will slow down printing somewhat, especially on early 68000-based
PostScript printers.\par}

\sec{Configuration File Searching}

The \.{dvips} program has a system of loading configuration files such that
\^{{\tt config.ps}}
certain parameters can be set globally across the system, others can be set
on a
per-printer basis, and yet others can be set by the user.  When \.{dvips}
starts up, first the global \.{config.ps} file is searched for and loaded.
This file is looked for along the path for configuration files, which is by
default \.{.:/usr/lib/tex/ps}.
After this master configuration file is loaded, a file by the name of
\.{.dvipsrc} is loaded from the current user's home directory, if such
a file exists.  This file is loaded in exactly the same way as the global
configuration file, and it can override any options set in the global
file.

Then the command line is read and parsed.  If the \.{-P} option is
encountered, at that point in the command line a configuration file
for that printer is read in.  Thus, the printer configuration file can
override anything in the global or user configuration file, and it can
also override anything seen in the command line up to the point that the
\.{-P} option was encountered.

After the command line has been completely scanned, if there was no
\.{-P} option selected, and also the \.{-o} and \.{-f} command line
options were not used, a \.{PRINTER} environment variable is
searched for.  If this variable exists, and a configuration file for
the printer mentioned in it exists, this configuration file is
loaded last of all.

Note that because the printer-specific configuration files are read
after the user's configuration file, the user's \.{.dvipsrc}
cannot override things in
the printer configuration files.  On the other hand, the configuration path
usually includes the current directory, and can be set to include the
user's home directory (or any other directory of the user), so the user can
always provide personalized printer-specific configuration files that will
be found before the system global ones.

If your printer uses a different resolution than 300 dpi,
make sure that you
have given a \MF\ mode as well as a resolution in the printer
configuration file.  Also make sure that \MF\ knows about the mode,
by entering it into your local \.{mode\_def} file (typically
\.{waits.mf}; \.{amiga.mf} on the Amiga, \.{next.mf} on the NeXT)
and recreating the \.{plain.base} file for \MF, including the
\.{mode\_def} file.  (Another good mode definition file is
{\tt modes.mf} by Karl Berry, which is available from
{\tt ftp.cs.umb.edu} in {\tt pub/tex/modes.mf}.)
The most common problem in generating fonts
with \MF\ is that
\^{{\tt mode\_def}}
\^{{\tt plain.base}}
\^{\MF}
this file with the mode definitions is not included when creating
the \.{plain.base} file.

\sec{Configuration File Options}

Most of the configuration file options are similar to the command line
options, but there are a few new ones.

Again, many may be turned off by suffixing the letter with a zero (0).
These options are \.a, \.f, \.q, \.r, \.I, \.K, \.N, \.U, and \.Z.

Within a configuration file, any empty line or line starting with a space,
asterisk, equal sign, or a pound sign is ignored.

{\options
\def\o#1 #2:{{\tt #1} {\it #2\/}:}
\def\O#1:{{\tt #1}:}
\o @ name hsize vsize: This option is used to set the paper size defaults
and options for the particular printer this configuration file
describes.  There are three formats for this option.  If the option is
specified on a line by itself, with no parameters, it instructs
\.{dvips} to discard all other paper size information (possibly from
another configuration file) and start fresh.  If three parameters
are given, as above, with the first parameter being a name and the second
and third being a dimension (as in 8.5in or 3.2cc, just like in the
\.{papersize} special), then the option is interpreted as starting a new
paper size description, where {\it name} is the name and {\it hsize} and
{\it vsize} describe the horizontal and vertical size of the sheet of
paper, respectively.  If both {\it hsize} and {\it vsize} are equal to
zero (although you must still specify units!\null) then any page size will
match it.  If the \.@ character is immediately followed by a
\.+ sign, then the remainder of the line (after skipping any leading
blanks) is treated as PostScript code to send to the printer to select
that particular paper size.  After all that, if the first character of
the line is an exclamation point, then the line is put in the initial
comments section of the final output file; else, it is put in the setup
section of the output file.  For instance, a subset of the paper size
information supplied in the default \.{config.ps} looks like
{\parindent=40pt\cmd{@ letterSize 8.5in 11in\\
@ letter 8.5in 11in\\
@+ \%\%BeginPaperSize: Letter\\
@+ letter\\
@+ \%\%EndPaperSize\\
@ legal 8.5in 14in\\
@+ ! \%\%DocumentPaperSizes: Legal\\
@+ \%\%BeginPaperSize: Legal\\
@+ legal\\
@+ \%\%EndPaperSize}}
\noindent
Note that you can even include structured comments in the configuration
file!  When \.{dvips} has a paper format name given on the command line,
it looks for a match by the {\it name}; when it has a \.{papersize}
special, it looks for a match by dimensions.  The first match found (in
the order the paper size information is found in the configuration file)
is used.  If nothing matches, a warning is printed and the first paper
size given is used, so the first paper size should always be the default.
The dimensions must match
within a quarter of an inch.  Landscape mode for all of the paper sizes
are automatically supported.  If your printer has a command to set a
special paper size, then give dimensions of \.{0in 0in}; the PostScript
code that sets the paper size can refer to the dimensions the user
requested as \.{\ttbackslash hsize} and \.{\ttbackslash vsize};
these will be macros defined in
the PostScript that return the requested size in default PostScript
units.  Note that virtually all of the PostScript commands you use
here are device dependent and degrade the portability of the file; that
is why the default first paper size entry should not send any PostScript
commands down (although a structured comment or two would be okay).
Also, some printers want {\tt BeginPaperSize} comments and paper size
setting commands; others (such as the NeXT) want {\tt PaperSize} comments
and they will handle setting the paper size.  There is no solution I
could find that works for both (except maybe specifying both).
See the supplied {\tt config.ps} file for a more realistic
example.

\O a:  Conserve memory by making three passes over the \.{dvi} file
instead of two and only loading those characters actually used.
Generally only useful on machines with a very limited amount of
memory, like some PCs.

\o b num: Generate {\it num} copies of each page, but duplicating the
page body rather than using PostScript's {\tt\#copies}.  This can be
useful in conjunction with a header file setting {\tt\char92bop-hook}
to do color separations or other neat tricks.

\o e num:
Set the maximum drift parameter to {\it num} dots (pixels) as explained above.
\^{maxdrift}

\O f:  Run as a filter by default.
\^{filter}

\o h name:
Add {\it name} as a PostScript header file to be downloaded at the beginning.
\^{header}

\o i num:
Make each section be a separate file, and set the maximum number of pages
in a given file to {\it num}.  Under certain circumstances,
\.{dvips} will split the document into `sections' to be processed
independently; this is most often done for memory reasons.  Using this
option tells \.{dvips} to place each section into a separate file; the
new file names are created by replacing the suffix of the supplied output
file name with a three-digit sequence number.
This is essentially a combination of the
command line options \.{-i} and \.{-S}; see the documentation for these
options for more information.

\o m num:  The value {\it num} is the virtual memory available for fonts
and strings in the printer.
\^{memory}
Default is 180000.  This value must be accurate if memory conservation and
document splitting is to work correctly.  To determine this value, send the
following file to the printer:
{\parindent=40pt\cmd{\%! Hey, we're PostScript\\
/Times-Roman findfont 30 scalefont setfont 144 432 moveto\\
vmstatus exch sub 40 string cvs show pop showpage}}
\noindent
Note that the number returned by this file is the total memory free;
it is often a good idea to tell \.{dvips} that the printer has somewhat
less memory.  This is because many programs download permanent macros
that can reduce the memory in the printer.  In general, a memory size
of about \.{300000} is plenty, since \.{dvips} can automatically split
a document if required.  It is unfortunate that PostScript printers with
much less virtual memory still exist.
Some systems or printers can dynamically increase the memory available
to a PostScript interpreter, in which case this file might return a
ridiculously low number; the NeXT computer is such a machine.  For these
systems, a value of one million works well.

\o o name:
\^{output}
The default output file is set to {\it name}.  As above, it can be a pipe.
Useful in printer-specific configuration files to redirect the output to
a particular printer queue.

\o p name:
The file to examine for PostScript font aliases is {\it name}.  It
defaults to {\tt psfonts.map}.  This option allows different printers
to use different resident fonts.  If the name starts with a `{\tt +}'
character, then the rest of the name (after any leading spaces) is used
as an additional map file; thus, it is possible to have local map files
pointed to by local configuration files that append to the global
map file.

\O q:  Run in quiet mode by default.
\^{quiet}

\O r:  Reverse the order of pages by default.
\^{reverse}

\O s:
Enclose the entire document in a global save/restore pair by default.
Not recommended, but useful in some environments; this breaks the
conformance of the document to the Adobe PostScript structuring
conventions.

\o D num:
Set the vertical and horizontal resolution to {\it num}
\^{resolution}
dots per inch.  Useful in printer-specific configuration files.

\o E command:
Execute the system command listed, for example as a UNIX shell command.
Execution takes place immediately, while the configuration file is
being read. This option can be used to insert the current date into a
header file, for instance, as explained at the end
of section~13.  Possibly dangerous; in many
installations it may be disabled, in which case a warning message will
be printed if the option is used.

\o H path:
The (colon-separated) path to search for PostScript header 
\^{header}
files is {\it path}.

\O I:
Ignore the {\tt PRINTER} environment variable.

\O K:
Filter comments out of included PostScript files; see the description
above for more information.

\o M mode: Set {\it mode}
as the \MF\ mode to be used when generating fonts.  This is
\^{{\tt MakeTeXPK}}
\^{{\MF}}
passed along to \.{MakeTeXPK} and overrides mode derivation from the
base resolution.

\O N:  Disable PostScript comments by default.

\o O offset:
Move the origin by a certain amount.  The {\it offset} is a comma-separated
pair of dimensions, such as \.{.1in,-.3cm} (in the same syntax as used in
the \.{papersize} special).   The origin of the page is shifted from the
default position (of one inch down, one inch to the right from the upper
left corner of the paper) by this amount.
This is useful for a printer that consistently offsets output pages by a
certain amount. You can use the file {\tt testpage.tex} to determine
the correct value for your printer. Be sure to do several runs with the
same {\tt O} value---some printers vary widely from run to run.

\o P path:  The (colon-separated) path to search for bitmap \.{pk}
font files is
{\it path}.  The \.{TEXPKS} environment variable will override this.
\^{{\tt TEXPKS}}
\^{{\tt pk}}
If a \.{\%} character is found in {\it path},
the following substitutions will be made, and then a search will
be made for the resulting filenames.
A \.{\%f} is replaced by the font name.
A \.{\%b} is replaced by the output device horizontal resolution dots per inch.
A \.{\%d} is replaced by the font size in dots per inch.
A \.{\%p} is replaced by the font family; this is always \.{pk}.
A \.{\%m} is replaced by the font mode; this is the mode given
in the \.{M} option.
Note that, for each path element, if it contains a percent sign, you
must give the full file name, including path, rather than just the
directory name; a path element such as {\tt /fonts/\%b} will try to
open {\tt /fonts/300} when looking for {\tt cmr10.329pk}, for
instance, and this may not be what is intended; {\tt /fonts/\%b/\%f.\%dpk}
is needed.  If a path element does not contain a percent sign, there
is no need to specify the entire file name (because you can't, unless
you list all possible specific font names!).
  
\o R num num $\ldots\,$:
Sets up a list of default resolutions to search for \.{pk} fonts, if the
\^{{\tt pk}}
requested size is not available.  The output will then scale the font
found using PostScript scaling to the requested size.  Note that the
resultant output will be ugly, and thus a warning is issued.  To turn
this off, use a line with just the \.{R} and no numbers.

\o S path:  The path to search for special illustrations
(Encapsulated PostScript files or psfiles) is {\it path}.
\^{{\tt TEXINPUTS}}
The \.{TEXINPUTS} environment variable will override this.

\o T path: The path to search for the \.{tfm} files is
{\it path}.  The \.{TEXFONTS} environment variable will override this.
\^{{\tt TEXFONTS}}
This path is used for resident fonts and fonts that can't otherwise be found.
It's usually best to make it identical to the path used by \TeX.

\O U:  Turns off a memory-saving optimization; this is necessary for the
Xerox 4045 printer, but not recommended otherwise.  See the description
above for more information.

\o V path:  The path to search for virtual font \.{vf} files is
\^{{\tt vf}}
\^{virtual fonts}
{\it path}.
This may be device-dependent if you use virtual fonts to simulate
actual fonts on different devices.

\o W string:
Sends {\it string} to stderr, if a parameter is given; otherwise it cancels
another previous  message.
This is useful in the default configuration file if you want to require
the user to specify a printer, for instance, or if you want to notify
the user that the resultant output has special characteristics.

\o X num:
\^{resolution}
Set the horizontal resolution to {\it num} dots per inch.

\o Y num:
\^{resolution}
Set the vertical resolution to {\it num} dots per inch.

\O Z:  Compress all downloaded fonts by default, as above.\par}

\sec{Automatic Font Generation}

One major problem with \TeX\ and the Computer Modern fonts is the huge
amount of disk space a full set of high resolution fonts can take.  The
\.{dvips} program solves this problem by creating fonts on demand, so only
those fonts that are actually used are stored on disk.  At a typical site,
less than one-fifth of the full set of Computer Modern fonts are used over
a long period, so this saves a great deal of disk space.

Furthermore, the addition of dynamic font generation allows fonts to be
used at any size, including typesetter resolutions and extremely huge
banner sizes.  Nothing special needs to be done; the fonts will be
automatically created and installed as needed.

The downside is that it does take a certain amount of time to create a new
font if it has never been used before.  But once a font is created, it
will exist on disk, and the next time that document is printed it will print
very quickly.

It is the \.{MakeTeXPK} shell script that is responsible for making these
\^{{\tt MakeTeXPK}}
fonts.  The \.{MakeTeXPK} script supplied invokes \MF\ to create the font and
then copies the resultant \.{pk} file to a world-writable font cache area.

\.{MakeTeXPK} can be customized to do other things to get the font.  For
instance, if you are installing \.{dvips} to replace (or run alongside)
an existing PostScript driver, and that driver demands \.{gf} fonts, you
can easily modify \.{MakeTeXPK} to invoke \.{gftopk} to convert the
\.{gf} files to \.{pk} files for \.{dvips}.  This provides the same space
savings listed above.

Because \.{dvips} (and thus \.{MakeTeXPK}) is run by a wide variety of users,
there must be a system-wide place to put the cached font files.  In order
for everyone to be able to supply fonts, the directory must be world
writable.  If your system administrator considers this a security hole,
\.{MakeTeXPK} can write to \.{/tmp/pk} or some such directory, and
periodically the cached fonts can be moved to a more general system area.
Note that the cache directory must exist on the \.{pk} file search path
in order for \.{MakeTeXPK} to work.

\sec{Path Interpretation}

The \.{dvips} program needs to read a wide variety of files from a large
set of directories.  It uses a set of paths to do this.  The actual
paths are listed in the next section; this section describes how the
paths are interpreted.

All path variables are names of directories (path elements),
separated by colons.
Each path element can be either the literal name of a directory or one
of the \.{\tilde} forms common under UNIX.  If a path element is a single tilde,
\^{UNIX}
it is replaced by the contents of the environment variable \.{HOME}, which
\^{{\tt HOME}}
is normally set to the user's home directory.  If the path element is
a tilde followed by anything, the part after the tilde is interpreted as
a user name, and his home directory is fetched from the system password
file and used as the real path element.

Where environment variables can override paths, an additional path
element form is allowed.  If a path element is the empty string, it
is replaced with the system defaults.  Thus, to
search the user's home directory, followed by the system
default paths, assuming the current shell is \.{csh},
the following command would be used:
\cmd{setenv TEXINPUTS \tilde:}
\noindent
This is a path of two elements.  The first is the user's home directory.
The second path element is the empty string, indicating that
the system defaults should be searched.

The `system defaults' as defined here means the strings set in the
\.{Makefile} before compilation, rather than any defaults set in
\.{config.ps} or printer-specific configuration files.  This is to
prevent path blowup, where more and more directories are added to the
path by each level of configuration file.

\sec{Environment Variables}

The \.{dvips} program reads a certain set of environment variables to
configure its operation.  The path variables are read after all
configuration files are read, so they override values in the configuration
files.  (The \.{TEXCONFIG} variable, of course, is read before the
configuration files.)  The rest are read as needed.

Note that all defaults supplied here are just as supplied in the
provided {\tt Makefile}; they will almost certainly have been changed
during installation at your particular site.

\descenv HOME,{\rm no default}
  This environment variable is automatically set by the shell and is
used to replace any occurrences of \.{\tilde} in a path.

\descenv MAKETEXPK,{MakeTeXPK \%n \%d \%b \%m}
This environment variable sets the command to be executed to create
a missing font.  A \%n is replaced by the base name of the font to
be created (such as \.{cmr10}); a \%d is replaced by the resultant
horizontal resolution of the font; a \%b is replaced by the
horizontal resolution at which \.{dvips} is currently generating
output, a \%o is replaced with the current \MF\ mode, if any,
or {\tt default} if none is known,
and any \%m is replaced by a string that \MF\ can use as
the right hand side of an assignment to \.{mag} to create the
desired font at the proper resolution.  If a mode for \MF\ is set in
a configuration file and no \%o is specified in the command,
the mode is automatically appended to the command
before execution.  Note that these substitutions are different than
the ones performed on PK paths.

\descenv DVIPSHEADERS,{.:/usr/lib/tex/ps}
  This environment variable determines where to search for header
files such as {\tt tex.pro}, font files, arguments to the
{\tt -h} option, and such files.

\descenv PRINTER,{\rm no default}
  This environment variable is read to determine which default printer
configuration file to read in.  Note that it is the responsibility of
the configuration file to send output to the proper print queue, if such
functionality is desired.

\descenv TEXFONTS,{/LocalLibrary/Fonts/TeXFonts/tfm:/usr/lib/tex/fonts/tfm}
  This is where \.{tfm} files are searched for.  A \.{tfm} file only
\^{{\tt tfm}}
needs to be loaded if the font is a resident (PostScript) font or if
for some reason no \.{pk} file could be found.

\descenv TEXPKS,{/LocalLibrary/Fonts/TeXFonts/pk:/usr/lib/tex/fonts/pk}
This environment variable is a path on which to search for \.{pk} fonts.
Certain substitutions are performed if a percent sign is found anywhere
\^{{\tt pk}}
in the path.  See the description of the {\tt P} configuration file
option for more information.

\descenv TEXINPUTS,{.:..:/usr/lib/tex/inputs}
  This environment variable is used to find PostScript figures when they
are included.

\descenv TEXCONFIG,{.:/usr/lib/tex/ps}
  This environment variable sets the directories to search for configuration
files, including the system-wide one.  Using this single environment variable
and the appropriate configuration files, it is possible to set up the program
for any environment.  (The other path environment variables can thus be
redundant.)

\descenv VFFONTS,{.:/usr/lib/tex/fonts/vf}
  This environment variable sets where \.{dvips} looks for virtual fonts.
A correct virtual font path is essential if PostScript fonts are to be
used.

\sec{Other Bells And Whistles}

For special effects, if any of the macros
\.{bop-hook}, \.{eop-hook}, \.{start-hook}, or \.{end-hook}
\^{{\tt bop-hook}}
\^{{\tt eop-hook}}
\^{{\tt start-hook}}
\^{{\tt end-hook}}
are defined in the PostScript \.{userdict}, they will be executed at the
beginning of a page, end of a page, start of the document, and end of
a document, respectively.
When these macros are executed, the default PostScript coordinate system
and origin
is in effect.  Such macros can be defined in headers added by the
\.{-h} option or the \.{header=}
special, and might be useful for writing, for instance, DRAFT across the
entire page, or, with the aid of a shell script, dating the document.
These macros are executed outside of the save/restore context of the
individual pages, so it is possible for them to accumulate information,
but if a document must be divided into sections because of memory
constraints, such added information will be lost across section breaks.

The two arguments to {\tt bop-hook} are the \TeX\ page number and
the sequence number of the page
in the file; the first page gets zero, the second one, etc.
The arguments to {\tt start-hook} are hsize, vsize, mag, hdpi, vdpi,
and the name of the \.{dvi} input file. The procedures
must leave these parameters on the stack.
The other hooks are not (currently)
given parameters, although this may change in the future.

As an example of what can be done, the following special will write
a light DRAFT across each page in the document:
{\vskip0pt\parskip=0pt\leftskip=20pt\begverb{`\$}
\special{!userdict begin /bop-hook{gsave 200 30 translate
65 rotate /Times-Roman findfont 216 scalefont setfont
0 0 moveto 0.7 setgray (DRAFT) show grestore}def end}$endverb}

Note that using \.{bop-hook} or \.{eop-hook} in any way that preserves
information across pages will break compliance with the Adobe document
structuring conventions, so if you use any such tricks, it is recommended
that you also use the \.{-N} option to turn off structured comments.

Several of the above tricks can be used nicely together, and it is not
necessary that a `printer configuration file' be used only to set
printer defaults.  For instance, you might have a file \.{config.dated}
that contains just the two lines
{\vskip0pt\parskip=0pt\leftskip=20pt\begverb{`\$}
E echo /bop-hook \{save /Times-Roman findfont 7 scalefont setfont \
                  72 756 moveto \(`date`\) show restore\} def >.date
h .date$endverb}
\noindent(with no newline following \.{setfont}); then the command-line option
\.{-Pdated} to \.{dvips} will print current date and time on the top
of each page of output.
Note that multiple \.{-P} options can be used.

\ifgeneric\else
\sec{MS-DOS and OS/2}
The MS-DOS and OS/2 versions of \.{dvips} differs from UNIX in the 
following ways.
\^{MS-DOS}
\^{OS/2}
\item{$\bullet$} Path separators are \.{;} not \.{:}.
\item{$\bullet$} Directory separators are \.{\ttbackslash} not \.{/}.
\item{$\bullet$} The user's initialization file is \.{dvips.ini}
  not \.{.dvipsrc}.
\item{$\bullet$} Printer configuration files are called 
  \.{{\it printername}.cfg}, not \.{config.{\it printername}}.
\item{$\bullet$} Pipes to printers are not supported by MS-DOS.  
  Output must go to a file.  OS/2 supports pipes.
\item{$\bullet$} \.{MakeTeXPk} is a batch or command file.  
  Since MS-DOS has insufficient memory to run both \.{dvips} and \MF\ 
  at the same time, this batch or command file will typically write out 
  a set of commands for running \MF\ later.  
  The \.{maketexp.bat} supplied writes out an \.{mfjob} file for em\TeX.
  OS/2 has more memory so the \.{maketexp.cmd} supplied writes out an 
  \.{mfjob} file for em\TeX\ and calls \.{mfjob}.
\item{$\bullet$} \.{dvidrv} from em\TeX\ can be used
  to automatically make fonts under MS-DOS as follows:
{\vskip0pt\parskip=0pt\leftskip=20pt\begverb{`\$}
    dvidrv dvips file.dvi $endverb}
  \.{dvidrv} sets an option \.{-pj=mfjobfile} for \.{dvips}, where 
  \.{mfjobfile} is the name of a temporary \.{mfjob} file.
  If there are missing fonts, \.{dvips} will write this \.{mfjob} file
  and then ask:
{\vskip0pt\parskip=0pt\leftskip=20pt\begverb{`\$}
    Exit to make missing fonts now (y/n)? $endverb}
  If you answer yes, \.{dvips} exits with errorlevel 8 which tells 
  \.{dvidrv} to call \.{mfjob} to make the fonts, and then to call 
  \.{dvips} again.
  For this to work, \.{dvidrv}, \.{dvips}, \.{mfjob} and \.{mf} 
  must be located in the \.{PATH}, and the environment variables for 
  \.{mfjob} and \.{mf} set correctly.  
  A font mode must be set with the '\.{M}' option in \.{config.ps}.
  If the \.{-pj} option is set, dvips will not call \.{MakeTeXPk.bat}.
\item{$\bullet$} Limited em\TeX\ specials.  The following ones are supported:
{\vskip0pt\parskip=0pt\leftskip=20pt\begverb{`\$}
\special{em:message xxx}
\special{em:point n}
\special{em:line a[h|v|p],b[h|v|p] [,width]}
\special{em:linewidth width}
\special{em:moveto}
\special{em:lineto}
\special{em:graph xxx[,width[,height]]} $endverb}
\item{}
  The line cut parameters \.{[h|v|p]} of the
  \.{\ttbackslash special{\char123}em:line ...{\char125}} command are ignored.
  Lines are ended with a rounded cap.
  A maximum of 1613 different point numbers are permitted on each page.
  The \.{\ttbackslash special{\char123}em:graph xxx{\char125}} 
  supports \.{PCX}, \.{MSP1}, \.{MSP2} and \.{BMP} files.
  The graph file may be scaled by giving an optional 
  width and height (expressed in the same way as \TeX\ dimensions).
  An example:
{\vskip0pt\parskip=0pt\leftskip=20pt\begverb{`\$}
  \special{em:graph scrdump.pcx,100mm,75mm} $endverb}

The program \.{dvips} can use em\TeX\ font libraries created with the 
em\TeX\ \.{fontlib /2} option.
If a \.{pxl} font is found within a font library, \.{dvips} will complain,
and then ignore the \.{pxl} font.

The font library names and directory names can be specified with 
this configuration file option.

{\options
\def\o#1 #2:{{\tt #1} {\it #2}:}
\def\O#1:{{\tt #1}:}
\o L path:  The list of \.{fli} font libraries to search for bitmap \.{pk}
  font files is {\it path}.  
\^{{\tt fli}}
  Fonts are sought first in these libraries and then as single files.
  Font libraries can be created with em\TeX 's \.{fontlib}; the usual
  extension is \.{fli}.  Directories as well as file names can be 
  entered here, the files will be searched for in all these directories.
  A directory name must have trailing directory separator
  (\.{/} for UNIX, \.{\ttbackslash} for MS-DOS and OS/2). \par }
\fi

\sec{Installation}

If \.{dvips} has not already been installed on your system, the
\^{installation}
following steps are all that are needed.

First update the \.{Makefile}---in particular, the paths.  Everything
concerning \.{dvips} can be adjusted in the \.{Makefile}.  Make sure
you set key parameters such as the default resolution, and make sure
that the path given for packed pixel files is correct.

Next, check the file name definitions in \.{MakeTeXPK}.  If you don't
\^{{\tt MakeTeXPK}}
have \MF\ installed, you cannot use \.{MakeTeXPK} to automatically
generate the fonts; you can, however, modify it to generate \.{pk}
fonts from \.{gf} fonts if you don't have a full set of \.{pk} fonts
but do have a set of \.{gf} fonts.  If you don't have that, you should
probably not install \.{MakeTeXPK} at all; this will disable automatic
font generation.

Now, check the configuration parameters in \.{config.ps}.  You should
also update the default resolution here.  This file is the
system-wide configuration file that will be automatically installed.
If you are unsure how much memory your PostScript printer has, print the
following file:
\cmd{\%! Hey, we're PostScript\\
/Times-Roman findfont 30 scalefont setfont 144 432 moveto\\
vmstatus exch sub 40 string cvs show pop showpage}
\noindent
Note that the number returned by this file is the total memory free;
it is often a good idea to tell \.{dvips} that the printer has somewhat
less memory.  This is because many programs download permanent macros
that can reduce the memory in the printer.  In general, a memory size
of about \.{300000} is plenty, since \.{dvips} can automatically split
a document if required.  It is unfortunate that PostScript printers with
much less virtual memory still exist.
Some systems or printers can dynamically increase the memory available
to a PostScript interpreter; for these systems, a value of one million
works well.

Next, run \.{make}.  Everything should compile smoothly.  You may need
to adjust the compiler options in the \.{Makefile} if something goes
amiss.

Once everything is compiled, run \.{make} \.{install}.  After this is done,
you may want to create a configuration file for each PostScript printer
at your site.

If the font caching is considered a security hole, make the `cache'
directory be something like \.{/tmp/pks}, and \.{cron} a job to move the
good \.{pk} files into the real directory.  Or simply disable this
feature by not installing \.{MakeTeXPK}.

Don't forget to install the new \.{vf} files and \.{tfm} files.  Note
that the \.{tfm} files distributed with earlier (pre-5.471) versions
of \.{dvips}, and all versions of other PostScript drivers, are different.

A test program called \.{test.tex} is provided, so you can easily check
that the most important parts of \.{dvips} have been installed properly.

\sec{Diagnosing Problems}

You've gone through all the trouble of installing \.{dvips}, carefully
read all the instructions in this manual, and still can't get something
to work.  This is all too common, and is usually caused by some broken
PostScript application out there.  The following sections provide some
helpful hints if you find yourself in such a situation.

In all cases, you should attempt to find the smallest file that causes
the problem.  This will not only make debugging easier, it will also
reduce the number of possible interactions among different parts of the
system.

\sub{Debug Options}

The \.{-d} flag to \.{dvips} is very useful for helping to track down
certain errors.  The parameter to this flag is an integer that tells
what errors are currently being tracked.  To track a certain class of
debug messages, simply provide the appropriate number given below;
if you wish to track multiple classes, sum the numbers of the classes
you wish to track.  The classes are:

$$\vbox{\halign{\hfil #\quad&#\hfil\cr
1&specials\cr2&paths\cr4&fonts\cr8&pages\cr16&headers\cr
32&font compression\cr64&files\cr
128&memory\cr}}$$

\sub{No Output At All}

If you are not getting any output at all, even from the simplest
one-character file (for instance, \.{\char92~\char92 bye}),
then something is very wrong.  Practically any file sent to a
PostScript laser printer should generate some output, at the very
least a page detailing what error occurred, if any.  Talk to your
system administrator about downloading a PostScript error handler.
(Adobe distributes a good one called \.{ehandler.ps}.)

It is possible, especially if you are using non-Adobe PostScript,
that your PostScript interpreter is broken.  Even then it should
generate an error message.  I've tried to work around as many bugs
as possible in common non-Adobe PostScript interpreters, but I'm
sure I've missed a few.

If \.{dvips} gives any strange error messages, or compilation on your
machine generated a lot of warnings, perhaps the \.{dvips} program
itself is broken.  Carefully check the types in \.{dvips.h} and
the declarations in the \.{Makefile}, and try using the debug options
to determine where the error occurred.

It is possible your spooler is broken and is misinterpreting the
structured comments.  Try the \.{-N} flag to turn off
structured comments and see what happens.

\sub{Output Too Small or Inverted}

If some documents come out inverted or too small, your spooler is not
supplying an end of job indicator at the end of each file.  (This happens
a lot on small machines that don't have spoolers.)  You can
force \.{dvips} to do this with the \.{-F} flag, but note that this
generates files with a binary character (control-D) in them.  You can
also try using the \.{-s} flag to enclose the entire job in a save/restore
pair.

\sub{Error Messages From Printer}

If your printer returns error messages, the error message gives very
good information on what might be going wrong.  One of the most common
error messages is \.{bop undefined}.  This is caused by old versions
of Transcript and other spoolers that do not properly parse the setup
section of the PostScript.  To fix this, turn off structured comments
with the \.{-N} option, but make sure you get your spooling software
updated.  You might also try turning off comments on included files
with the \.{-K} option; many spoolers cannot deal with nested
documents.

Another error message is \.{VM exhausted}.  (Some printers indicate
this error by locking up; others quietly reset.)  This is caused by telling
\.{dvips} that the printer has more memory than it actually does, and
then printing a complicated document.  To fix this, try lowering the
parameter to \.{m} in the configuration file; use the debug option
to make sure you adjust the correct file.

Other errors may indicate that the graphics you are trying to include
don't nest properly in other PostScript documents, or any of a number of
other possibilities.  Try the output on a QMS PS-810 or other Adobe
PostScript printer; it might be a problem with the printer itself.

\sub{400 DPI Is Used Instead Of 300 DPI}

This common error is caused by not editing the \.{config.ps} file to
reflect the correct resolution for your site.  You can use the debug flags
(\.{-d64}) to see what files are actually being read.

\sub{Long Documents Fail To Print}

This is usually caused by incorrectly specifying the amount of memory
the printer has in \.{config.ps}; see the description above.

\sub{Including Graphics Fails}

The reasons why graphics inclusions fail are too numerous to mention.
The most common problem is an incorrect bounding box; read the section
on bounding boxes and check your PostScript file.  Complain very loudly
to whoever wrote the software that generated the file if the bounding
box is indeed incorrect.

Another possible problem is that the figure you are trying to include
does not nest properly; there are certain rules PostScript applications
should follow when generating files to be included.  The \.{dvips}
program includes work-arounds for such errors in Adobe Illustrator and
other programs, but there are certainly applications that haven't
been tested.

One possible thing to try is the \.{-K} flag, to strip the comments from
an included figure.  This might be necessary if the PostScript spooling
software does not read the structuring comments correctly.  Use of this
flag will break graphics from some applications, though, since some
applications read the PostScript file from the input stream looking for
a particular comment.

Any application which generates graphics output containing raw binary
(not hex) will probably fail with \.{dvips}.

\sub{Can't Find Font Files}

If \.{dvips} complains that it cannot find certain font files, it is
possible that the paths haven't been set up correctly for your system.
Use the debug flags to determine precisely what fonts are being looked
for and make sure these match where the fonts are located on your system.

\sub{Can't Generate Fonts}

This happens a lot if either \.{MakeTeXPK} hasn't been properly edited
and installed, or if the local installation of \MF\ isn't correct.
If \.{MakeTeXPK} isn't properly edited or isn't installed, an error
such as \.{MakeTeXPK not found} will be printed on the console.  The
fix is to talk to the person who installed \.{dvips} and have them fix
this.

If \MF\ isn't found when \.{MakeTeXPK} is running, make sure it is installed
on your system.  The person who installed \TeX\ should be able to install
\MF\ easily.

If \MF\ runs but generates fonts that are too large (and prints out the
name of each character as well as just a character number), then your
\MF\ base file probably hasn't been made properly.  To make a proper
\.{plain.base}, assuming the local mode definitions are contained in
\.{local.mf} (on the NeXT, \.{next.mf}; on the Amiga, \.{amiga.mf}),
type the following command (assuming \.{csh} under UNIX):
\cmd{localhost> inimf "plain; input local; dump"}
\noindent
Now, copy the \.{plain.base} file from the current directory to where
the base files are stored on your system.

Note that a preloaded \.{cmbase.base} should never be used when creating
fonts, and a program such as \.{cmmf} should never exist on the system.
The macros defined in \.{cmbase} will break fonts that do not use
\.{cmbase}; such fonts include the La\TeX\ fonts.  Loading the \.{cmbase}
macros when they are needed is done automatically and takes less than a
second---an insignificant fraction of the total run time of \MF\ for a
font, especially when the possibility of generating incorrect fonts is
taken into account.  If you create the La\TeX\ font {\tt circle10},
for instance, with the \.{cmbase} macros loaded, the characters will
have incorrect widths.

\sec{Using Color with dvips}

This new feature of \.{dvips} is somewhat experimental so your
experiences and comments are welcome.  Initially added by Jim Hafner,
IBM Research, {\tt hafner@almaden.ibm.com}, the color support has
gone through many changes by Tomas Rokicki.  Besides the changes to the
source code itself, there are additional
\TeX\ macro files: \.{colordvi.tex} and \.{blackdvi.tex}.  There are also
\.{.sty} versions of these files that can be used with La\TeX\ and
other similar macro packages.  This feature adds one-pass multi-color
printing of \TeX\ documents on any color PostScript device.

In this section we describe the use of color from the document
preparer's point of view and then add some simple instructions on
installation for the system administrator.

\sub{The Macro Files}

All the color macro commands are defined in \.{colordvi.tex} (or
\.{colordvi.sty}).  To access these macros simply add to the top of your
\TeX\ file the command
\cmd{\ttbackslash input colordvi}
\noindent
or, if your document uses style files like La\TeX, add the \.{colordvi}
style option as in 
\cmd{\ttbackslash documentstyle[12pt,colordvi]\char123article\char125}

There are basically two kinds of color macros, ones for local color
changes to, say, a few words or even one symbol and one for global
color changes.  Note that all the color names use a mixed case scheme.
There are 68 predefined colors, with names taken primarily from the
Crayola crayon box of 64 colors, and one pair of macros for the user
to set his own color pattern.  More on this extra feature later. You
can browse the file \.{colordvi.tex} for a list of the predefined colors.
The comments in this file also show a rough correspondence between the
crayon names and PANTONEs.

A local color command is in the form
\cmd{\ttbackslash ColorName\char123this will print in color\char125}
\noindent
Here \.{ColorName} is the name of a predefined color. As this example
shows, this type of command takes one argument which is the text that
is to print in the selected color.  This can be used for nested color
changes since it restores the original color state when it completes.
For example, suppose you were writing in green and want to switch
temporarily to red, then blue, back to red and restore green.  Here is
one way that you can do this:
\cmd{This text is green but here we are \ttbackslash Red\char123switching 
to red, \\
\ttbackslash Blue\char123nesting blue\char125, recovering the 
red\char125\ and back to \\
original green.}
\noindent
In principle there is no limit to the nesting level, but it is not
advisable to nest too deep lest you lose track of the color history.
 
The global color command has the form
\cmd{\ttbackslash textColorName}
\noindent 
This macro takes no arguments and immediately changes the default
color from that point on to the specified color. This of course can be
overriden globally by another such command or locally by local color
commands. For example, expanding on the example above, we might have
\cmd{\ttbackslash textGreen \\
This text is green but here we are \ttbackslash Red\char123switching 
to red, \\
\ttbackslash Blue\char123nesting blue\char125, recovering the 
red\char125\ and back to \\
original green.\\
\ttbackslash textCyan \\
The text from here on will be cyan unless \\
\ttbackslash Yellow\char123locally changed to yellow\char125. Now we
are back to cyan.} 
 
The color commands will even work in math mode and across math mode
boundaries.  This means that if you have a color before going into
math mode, the mathematics will be set in that color as well.  More
importantly however, in alignment environments like \.{\ttbackslash
halign}, \.{tabular} or \.{eqnarray}, local color commands cannot
extend beyond the alignment characters.
 
Because local color commands respect only some environment and
delimiter changes besides their own, care must be taken in setting
their scope.  It is best not to have then stretch too far.
 
At the present time there are no macros for color environments in
La\TeX\ which might have a larger range.  This is primarily to keep
the \TeX\ and La\TeX\ use compatible. 

\sub{User Definable Colors}
 
There are two ways for the user to specify colors not already defined.
For local changes, there is the command \.{\ttbackslash Color} which
takes two arguments.  The first argument is a quadruple of numbers
between zero and one and specifies the intensity of cyan, magenta,
yellow and black (CMYK) in that order.  The second argument is the
text that should appear in the given color.  For example, suppose you
want the words ``this color is pretty'' to appear in a color which is
50\% cyan, 85\% magenta, 40\% yellow and 20\% black.  You would use
the command
\cmd{\ttbackslash Color\char123{.5 .85 .4 .2}\char125\char123 this
color is pretty\char125}

For global color changes, there is a command \.{\ttbackslash
textColor} which takes one argument, the CMYK quadruple of relative
color intensities.  For example, if you want the default color to be
as above, then the command
\cmd{\ttbackslash textColor\char123{.5 .85 .4 .2}\char125 \\
The text from now on will be this pretty color} 
\noindent
will do the trick.
 
Making a global color change in the midst of nested local colors is
highly discouraged.  Consequently, \.{dvips} will give you warning
message and do its best to recover by discarding the current color
history.

\sub{Subtleties in Using Color}

These color macros are defined by use of specialized \.{\ttbackslash
special} keywords.  As such, they are put in the \.{.dvi} file only as
explicit message strings to the driver.  The (unpleasant)
result is that certain unprotected regions of the text can have
unwanted color side effects.  For example, if a color region is split
by \TeX\ across a page boundary, then the footers of the current page
(e.g., the page number) and the headers of the next page can inherit
that color.  To avoid this effect globally, users should make sure
that these special regions of the text are defined with their own
local color commands.  For example in \TeX, to protect the header and
footer, use 
\cmd{\ttbackslash headline\char123\ttbackslash Black\char123 My
Header\char125\char125 \\
\ttbackslash footline\char123\ttbackslash Black\char123 \ttbackslash
hss\ttbackslash tenrm\ttbackslash folio\ttbackslash hss\char125\char125} 

This warning also applies to figures and other insertions, so be
careful!

Of course, in La\TeX, this is much more difficult to do because of the
complexity of the macros that control these regions.  This is
unfortunate, but is somehow inevitable because \TeX\ and La\TeX\ were
not written with color in mind.

Even when writing your own macros, much care must be taken.  The
color macros that `colorize' a portion of the text work by prefixing
the text with a special command to turn the color on and postfixing it
with a different special command to restore the original color.
It is often useful to insure that \TeX\ is in horizontal mode before
the first special command is issued; this can be done by prefixing the
color command with {\tt\char92 leavevmode}.

\sub{Printing in Black/White, after Colorizing}

If you have a \TeX\ or La\TeX\ document written with color macros and
you want to print it in black and white there are two options.  On all
(good) PostScript devices, printing a color file will print in
corresponding grey-levels.  This is useful since in this way you can
get a rough idea of where the colors are changing without using
expensive color printing devices.  The second option is to replace the
call to input \.{colordvi.tex} with \.{blackdvi.tex} (and similarly for the
\.{.sty} files).  So in the above example, replacing the word
\.{colordvi} with \.{blackdvi} suffices.  This file defines the color macros
as no-ops, and so will produce normal black/white printing.  By this
simple mechanism, the user can switch to all black/white printing
without having to ferret out the color commands.  Also, some device
drivers, particularly non-PostScript ones like screen previewers, will
simply ignore the color commands and so print in normal black/white.
Hopefully, in the future screen previewers for color displays will be
compatible with some form of color support.

\sub{Configuring dvips for Color Devices}

To configure dvips for a particular color device you need to fine tune
the color parameters to match your device's color rendition.  To do
this, you will need a PANTONE chart for your device.  The header file
\.{color.lpro} shows a (rough) correspondence between the Crayola
crayon names and the PANTONE numbers and also defines default CMYK
values for each of the colors.  Note that these colors must be defined
in CMYK terms and not RGB as \.{dvips} outputs PostScript color
commands in CMYK.  This header file also defines (if they are not
known to the interpreter) the PostScript commands \.{setcmykcolor} and
\.{currentcmykcolor} in terms of a RGB equivalent so if your device
only understands RGB, there should be no problem.

The parameters set in this file were determined by comparing the
PANTONE chart of a Tektronics PHASER printer with the actual Crayola
Crayons.  Because these were defined for a particular device, the
actual color rendition on your device may be very different.  There
are two ways to adjust this.  One is to use the PANTONE chart for your
device to rewrite \.{color.lpro} prior to compilation and
installation.  A better alternative, which supports multiple devices,
is to add a header file option in the configuration file for each
device that defines, in \.{userdict}, the color parameters for those
colors that need redefining.

For example, if you need to change the parameters defining
\.{Goldenrod} (approximately PANTONE 109) for your device
\.{mycolordev}, do the following.  In the PANTONE chart for your
device, find the CMYK values for PANTONE 109.  Let's say they are
\.{\char123\ 0 0.10 0.75 0.03 \char125}. Then create a header file
named \.{mycolordev.pro} with the commands
\cmd{userdict begin \\
/Goldenrod \char123\ 0 0.10 0.75 0.03 setcmykcolor\char125\ bind def}
\noindent
Finally, in \.{config.mycolordev} add the line 
\cmd{h mycolordev.pro}
\noindent
This will then define \.{Goldenrod} in your device's CMYK values in
\.{userdict} which is checked before defining it in \.{TeXdict} by
\.{color.pro}.

This mechanism, together with additions to \.{colordvi.tex} and
\.{blackdvi.tex} (and the \.{.sty} files), can also be used to predefine
other colors for your users.

\sub{Protecting Regions From Spurious Colors}

Because color is defined via \TeX's {\tt\ttbackslash special} command,
it cannot be sensitive to the output routine or certain regions of the
page like the header and footer.  Consequently, these regions need to
be protected from spurious color changes (particularly when local
colors spread across page breaks).

Users need to be aware of the possibility of certain regions getting
unwanted or unpredicted colors.  Headers and footers are most
worrisome so style designers who want to use color should keep this in
mind.

One particular region of text that gets spurious color effects is
labels in list environments.  For instance, because of the way list
items are defined in standard La\TeX, the bullet for items that
start with a different color also gets drawn in that color.

To give the user a simple mechanism to solve this problem (and other
unforeseen effects of this type) one other special macro is
automatically defined.  This macro is called {\tt\ttbackslash
globalColor}.  It is actually a {\it local\/} color macro and so takes
a single argument.  But the color effect it produces is always the
same as that set by the {\it last\/} {\tt\ttbackslash textColor} or
{\tt\ttbackslash textColorName} command.  In effect, when a
{\tt\ttbackslash textColorName} command is called, {\tt\ttbackslash
globalColor} gets a new definition equivalent to the local
{\tt\ttbackslash ColorName} macro.  For example, when the default is
black, {\tt\ttbackslash globalColor=\ttbackslash Black} and when
{\tt\ttbackslash textGreen} appears, {\tt\ttbackslash
globalColor=\ttbackslash Green}.  This special macro can then be used
to protect sensitive regions of the text.

For example, in La\TeX\ files, one might make sure that the header and
footers have {\tt\ttbackslash globalColor} wrapping their contents.
In this way, they will inherit the current active root (unnested)
color state.

\sub{Color Support Details}

To support color, \.{dvips} recognizes a certain set of specials.
These specials all start with the keyword \.{color} or the keyword
\.{background}.

We will describe \.{background} first, since it is the simplest.
The \.{background} keyword must be followed by a color specification.
That color specification is used as a fill color for the background.
The last \.{background} special on a page is the one that gets issued,
and it gets issued at the very beginning of the page, before any
text or specials are sent.  (This is possible because the prescan
phase of \.{dvips} notices all of the color specials so that the
appropriate information can be written out during the second phase.)

Ahh, but what is a color specification?  It is one of three things.
First, it might be a PostScript procedure as defined in a PostScript header
file.  The \.{color.pro} file defines 64 of these, including
\.{Maroon}.  This PostScript procedure must set the current color to be
some value; in this case, \.{Maroon} is defined as
\.{0 0.87 0.68 0.32 setcmykcolor}.

The second possibility is the name of a color model (initially, one of
\.{rgb}, \.{hsb}, \.{cmyk}, or \.{gray}) followed by the appropriate
number of parameters.  When \.{dvips} encounters such a macro, it sends
out the parameters first, followed by the string created by prefixing
\.{TeXcolor} to the color model.  Thus, the color specification
\.{rgb 0.3 0.4 0.5} would generate the PostScript code \.{0.3 0.4 0.5
TeXrgbcolor}.  Note that the case of zero arguments is disallowed, as
that is handled by the single keyword case above (where no changes to
the name are made before it is sent to the PostScript file.)

The third and final type of color specification is a double quote followed
by any sequence of PostScript.  The double quote is stripped from the
output.  For instance, the color specification \.{"AggiePattern setpattern}
will set the `color' to the Aggie logo pattern (assuming such exists.)

So those are the three types of color specifications.  The same type of
specifications are used by both the {\tt background} special and the
{\tt color} special.  The {\tt color} special itself has three forms.
The first is just {\tt color} followed by a color specification.  In
this case, the current global color is set to that color; the color stack
must be empty when such a command is executed.

The second form is {\tt color push} followed by a color specification.
This saves the current color on the color stack and sets the color
to be that given by the color specification.  This is the most common
way to set a color.

The final version of the {\tt color} special is just {\tt color pop},
with no color specification; this says to pop the color last pushed
on the color stack from the color stack and set the current color to
be that color.

The {\tt dvips} program correctly handles these color specials
across pages, even when the pages are repeated or reversed.

These color specials can be used for things such as patterns or
screens as well as simple colors.  However, note that in the
PostScript, only one `color specification' can be active at a
time.  For instance, at the beginning of a page, only the
bottommost entry on the color stack is sent; also, when a color
is `popped', all that is done is that the color specification
from the previous stack entry is sent.  No \.{gsave} or \.{grestore}
is used.  This means that you cannot easily mix usage of the 
{\tt color} specials for screens and colors, just one or the other.
This may be addressed in the future by adding support for
different `categories' of color-like state.

\bye
