% interface=english modes=letter,screen output=pdftex
% vim: tw=79

% $Id: pdftex-t.tex,v 1.621 2005/08/25 15:41:41 oneiros Exp $

% The number of lines on the titlepage depends on the kind of
% \PDF\ code that \PDFTEX\ generates.

\setvariables[pdftex][titlepagelines=62]

%D We use a multi purpose style (using modes) that enable us
%D to generate an A4, letter and screen version.
%D
%D Default A4 size manual:
%D
%D texexec --result pdftex-a.pdf pdftex-t
%D
%D Letter size manual:
%D
%D texexec --mode=letter --result=pdftex-l.pdf pdftex-t
%D
%D Booklet (given that A4 document is available):
%D
%D texexec --pdfarrange --paper=a5a4 --print=up --addempty=1,2 --result=pdftex-b.pdf pdftex-a
%D
%D Screen version
%D
%D texexec --mode=screen pdftex-t

%D This is the \PDFTEX\ manual, so it makes sense to force \PDF\ output here.

\setupoutput
  [pdftex]

%D ConTeXt defaults to 1.5; we want to be backwards compatible (and we don't
%D use any features from 1.4++ anyway, do we?)
\pdfminorversion=3

%D First we define ourselves some abbreviations, if only to force
%D consistency and to save typing. We use real small capitals, not pseudo
%D ones.

\setupsynonyms
  [abbreviation]
  [textstyle=smallcaps,
   style=smallcaps,
   location=left,
   width=broad,
   sample=\POSTSCRIPT]

\setupcapitals
  [title=no]

%***********************************************************************

\def\rcsscan $#1 #2,v #3 #4/#5/#6 #7${
  \def\rcsrevision{#3}
  \def\rcsyear{#4}
  \def\rcsmonth{#5}
  \def\rcsday{{\count0=#6\relax\the\count0}}
  \def\rcsmonthname{\ifcase\rcsmonth ERROR\or
    January\or February\or March\or April\or May\or June\or
    July\or August\or September\or October\or November\or December\else ERROR\fi}
}

\rcsscan $Id: pdftex-t.tex,v 1.621 2005/08/25 15:41:41 oneiros Exp $

\def\currentpdftex{1.30.2}

%***********************************************************************

\def\eTeX{{$\varepsilon$}-\kern-.125em\TeX}
\def\eg{e.\,g.}
\def\Eg{E.\,g.}

\abbreviation [AFM]        {afm}        {Adobe Font Metrics}
%\abbreviation [AMIGA]      {Amiga}      {Amiga hardware platform}
%\abbreviation [AMIWEB]     {AmiWeb2c}   {\AMIGA\ distribution}
\abbreviation [ASCII]      {ascii}      {American Standard Code for Information Interchange}
\abbreviation [CMACTEX]    {CMac\TeX}   {\MAC\ \WEBC\ distribution}
\abbreviation [CONTEXT]    {\ConTeXt}   {general purpose macro package}
\abbreviation [CTAN]       {CTAN}       {global \TEX\ archive server}
%\abbreviation [DJGPP]      {djgpp}      {DJ Delorie's \GNU\ Programming Platform}
\abbreviation [DVI]        {dvi}        {native \TEX\ Device Independent file format}
\abbreviation [ENCTEX]     {enc\TeX}    {enc\TeX\ extension to \TEX}
\abbreviation [EPSTOPDF]   {epstopdf}   {\EPS\ to \PDF\ conversion tool}
\abbreviation [EPS]        {eps}        {Encapsulated PostScript}
\abbreviation [ETEX]       {\eTeX}      {an extension to \TEX}
\abbreviation [EXIF]       {exif}       {Exchangeable Image File format (JPEG file variant)}
\abbreviation [FPTEX]      {fp\TeX}     {\WIN\ \WEBC\ distribution}
\abbreviation [GHOSTSCRIPT] {Ghostscript} {\PS\ and \PDF\ language interpreter}
\abbreviation [GNU]        {gnu}        {GNU's Not Unix}
\abbreviation [HZ]         {hz}         {Hermann Zapf optimization}
\abbreviation [JPEG]       {jpeg}       {Joint Photographic Expert Group}
\abbreviation [LATEX]      {\LaTeX}     {general purpose macro package}
\abbreviation [MAC]        {Macintosh}  {Macintosh hardware platform}
\abbreviation [MACOSX]     {Mac\,OS\,X} {Macintosh operating system version 10}
\abbreviation [MDFIVE]     {md5}        {MD5 message-digest algorithm}
\abbreviation [METAFONT]   {\MetaFont}  {graphic programming environment, bitmap output}
\abbreviation [METAPOST]   {\MetaPost}  {graphic programming environment, vector output}
\abbreviation [MIKTEX]     {MikTeX}     {\WIN\ distribution}
\abbreviation [MLTEX]      {ML\TeX}     {ML\TeX\ extension to \TEX}
\abbreviation [MPTOPDF]    {mptopdf}    {\METAPOST\ to \PDF\ conversion tool}
\abbreviation [MSDOS]      {MSDos}      {Microsoft DOS platform (Intel)}
\abbreviation [PDFETEX]    {pdfe\TeX}   {\ETEX\ extension producing \PDF\ output}
\abbreviation [PDFLATEX]   {pdf\LaTeX}  {\TEX\ extension producing \PDF\ output (\LATEX\ format loaded)}
\abbreviation [PDFTEX]     {pdf\TeX}    {\TEX\ extension producing \PDF\ output}
\abbreviation [PDF]        {pdf}        {Portable Document Format}
\abbreviation [PERL]       {Perl}       {Perl programming environment}
\abbreviation [PFA]        {PFA}        {Adobe PostScript Font format (ASCII)}
\abbreviation [PFB]        {PFB}        {Adobe PostScript Font format (Binary)}
\abbreviation [PGC]        {pgc}        {\PDF\ Glyph Container}
\abbreviation [PK]         {pk}         {Packed bitmap font}
\abbreviation [PNG]        {png}        {Portable Network Graphics}
\abbreviation [PS]         {ps}         {PostScript}
\abbreviation [POSTSCRIPT] {PostScript} {PostScript}
\abbreviation [PSTOPDF]    {PStoPDF}    {PostScript to \PDF\ converter (on top of GhostScript)}
\abbreviation [RGB]        {rgb}        {Red Green Blue color specification}
\abbreviation [TCX]        {tcx}        {\TEX\ Character Translation}
\abbreviation [TDS]        {tds}        {\TEX\ Directory Standard}
\abbreviation [TETEX]      {te\TeX}     {\TEX\ distribution for \UNIX\ (based on \WEBC)}
\abbreviation [TEXEXEC]    {\TeX exec}  {\CONTEXT\ command line interface}
\abbreviation [TEXINFO]    {Texinfo}    {generate typeset documentation from info pages}
\abbreviation [TEXUTIL]    {\TeX util}  {\CONTEXT\ utility tool}
\abbreviation [TEX]        {\TeX}       {typographic language and program}
\abbreviation [TEXLIVE]    {\TeX\ Live} {\TeX-Live distribution (multiple platform)}
\abbreviation [TFM]        {tfm}        {\TEX\ Font Metrics}
\abbreviation [TIF]        {tiff}       {Tagged Interchange File format}
\abbreviation [TUG]        {tug}        {\TEX\ Users Group}
\abbreviation [UNIX]       {Unix}       {Unix platform}
\abbreviation [URL]        {url}        {Uniform Resource Locator}
\abbreviation [WEBC]       {Web2c}      {official multi||platform \WEB\ environment}
\abbreviation [WEB]        {web}        {literate programming environment}
\abbreviation [WIN]        {Win32}      {Microsoft Windows platform}
\abbreviation [ZIP]        {zip}        {compressed file format}

%D It makes sense to predefine the name of the author of \PDFTEX, doesn't it?

\def\THANH{H\`an Th\^e\llap{\raise 0.5ex\hbox{\'{}}} Th\`anh}

%D Because they are subjected to changes and spoil the visual appearance of
%D the \TEX\ source, \URL's are defined here.

\useURL [ptex_org]      [http://www.pdftex.org] % links to ptex_examples
\useURL [ptex_ctan]     [ctan:systems/pdftex]
%\useURL [ptex_devel]    [http://pdftex.sarovar.org/current/]
\useURL [ptex_sarovar]  [http://sarovar.org/projects/pdftex/]
%\useURL [ptex_examples] [http://www.tug.org/applications/pdftex/]
\useURL [tex_showcase]  [http://www.tug.org/texshowcase]

\useURL [tetex]         [http://www.tug.org/teTeX/]
\useURL [texlive]       [http://www.tug.org/tex-live/]
\useURL [ctan_systems]  [ctan:systems]
\useURL [win32]         [ctan:systems/win32]
%\useURL [amiga]         [ctan:systems/amiga/amiweb2c/]

\useURL [fabrice]    [mailto:popineau@supelec.fr]
\useURL [thomas]     [mailto:te@dbs.uni-hannover.de]
\useURL [andreas]    [mailto:andreas.scherer@pobox.com]
\useURL [christian]  [mailto:cschenk@berlin.snafu.de]
\useURL [context]    [http://www.pragma-ade.com]

% where the bug reports should go:
\useURL [ptex_bugs]  [mailto:pdftex@tug.org]                                  [] [pdftex@tug.org]
% anything else pdftex related:
\useURL [ptex_list]  [mailto:pdftex@tug.org]                                  [] [pdftex@tug.org]

\useURL [sebastian]  [mailto:s.rahtz@elsevier.co.uk]                          [] [s.rahtz@elsevier.co.uk]
\useURL [sebastian]  [mailto:sebastian.rahtz@computing-services.oxford.ac.uk] [] [sebastian.rahtz@computing-services.oxford.ac.uk]
\useURL [thanh]      [mailto:thanh@informatics.muni.cz]                       [] [thanh@informatics.muni.cz]
\useURL [thanh]      [mailto:hanthethanh@myrealbox.com]                       [] [hanthethanh@myrealbox.com]
\useURL [martin]     [mailto:martin@pdftex.org]                               [] [martin@pdftex.org]
\useURL [hans]       [mailto:pragma@wxs.nl]                                   [] [pragma@wxs.nl]
\useURL [hartmut]    [mailto:hartmut_henkel@gmx.de]                           [] [hartmut\_henkel@gmx.de]
\useURL [pawel]      [mailto:p.jackowski@gust.org.pl]                         [] [p.jackowski@gust.org.pl]

%D The primitive definitions are specified a bit fuzzy using the next set of
%D commands. Some day I'll write some proper macros to deal with this.

\def\Sugar    #1{\unskip\unskip\unskip\kern.25em{#1}\kern.25em\ignorespaces}
\def\Something#1{\Sugar{\mathematics{\langle\hbox{#1}\rangle}}}
\def\Lbrace     {\Sugar{\tttf\leftargument}}
\def\Rbrace     {\Sugar{\tttf\rightargument}}
\def\Or         {\Sugar{\mathematics{\vert}}}
\def\Optional #1{\Sugar{\mathematics{[\hbox{#1}]}}}
\def\Means      {\Sugar{\mathematics{\rightarrow}}}
\def\Tex      #1{\Sugar{\type{#1}}}
\def\Literal  #1{\Sugar{\type{#1}}}
\def\Syntax   #1{\strut\kern-.25em{#1}\kern-.25em}
\def\Next       {\crlf\hbox to 2em{}\nobreak}
\def\Whatever #1{\Sugar{\mathematics{(\hbox{#1})}}}

% hyphenates

\def\Something#1{\Sugar{\mathematics{\langle}\prewordbreak{#1}\prewordbreak\mathematics{\rangle}}}

% introduced
\def\introduced#1{The primitive has been introduced in \PDFTEX\ #1.}

%D We typeset the \URL's in a monospaced font.

\setupurl
  [style=type]

%D The basic layout is pretty simple. Because we want non european readers to
%D read the whole text as well, a letter size based alternative is offered
%D too. Mode switching is taken care of at the command line when running
%D \TEXEXEC.

\startmode[letter]

  \setuppapersize
    [letter][letter]

\stopmode

\setuplayout
  [topspace=1.5cm,
   backspace=2.5cm,
   leftmargin=2.5cm,
   width=middle,
   footer=1.5cm,
   header=1.5cm,
   height=middle]

%D For the moment we use the description mechanism to typeset keywords etc.
%D Well, I should have used capitals.

\definedescription
  [description]
  [location=serried,
   width=broad]

\definedescription
  [PathDescription]
  [location=left,
   sample=TEXPSHEADERS,
   width=broad,
   headstyle=type]

\definehead
  [pdftexprimitive]
  [subsubsection]

\setuphead
  [pdftexprimitive]
  [style=,
   before=\blank,
   after=\blank,
   numbercommand=\SubSub]

%D This is typically a document where we want to save pages,
%D so we don't start any matter (part) on a new page.

\setupsectionblock [frontpart] [page=]
\setupsectionblock [bodypart]  [page=]
\setupsectionblock [backpart]  [page=]

%D The \type{\SubSub} command is rather simple and generates a triangle.
%D This makes the primitives stand out a bit more.

\def\SubSub#1{\mathematics{\blacktriangleright}}

%D But, for non Lucida users, the next one is more safe:

\def\SubSub#1{\goforwardcharacter}

%D I could have said:
%D
%D \starttyping
%D \setupsection[section-5][previousnumber=no,bodypartconversion=empty]
%D \setuplabeltext[subsubsection=\mathematics{\blacktriangleright}]
%D \stoptyping
%D
%D but this is less clear.

%D We use white space, but not too much.

\setupwhitespace
  [medium]

\setupblank
  [medium]

%D Verbatim things get a small margin.

\setuptyping
  [margin=standard]

\definetype
  [esctype]
\setuptype
  [esctype]
  [option=commands,escape=@]

\definetyping
  [esctyping]
\setuptyping
  [esctyping]
  [margin=standard,option=commands,escape=@]

%D Due to the lots of verbatim we will be a bit more tolerant in breaking
%D paragraphs into lines.

\setuptolerance
  [verytolerant,stretch]

%D We put the chapter and section numbers in the margin and use bold fonts.

\setupheads
  [alternative=margin]

\setuphead
  [section]
  [style=\bfb]

\setuphead
  [subsection]
  [style=\bfa]

%D The small table of contents is limited to section titles and is fully
%D interactive.

\setuplist
  [section]
  [criterium=all,
   interaction=all,
   width=2em,
   aligntitle=yes,
   alternative=a]

%D Ah, this manual is in english!

\mainlanguage
  [en]

%D We use Palatino fonts, because they look so well on the screen. The
%D version generated at \PRAGMA\ uses Optima Nova instead. Both fonts are
%D designed by Hermann Zapf.

\setupfonthandling [hz] [min=25,max=25,step=5]

\setupalign
  [hz,hanging]

\startnotmode[atpragma]

  \setupfontsynonym [Serif]            [handling=quality]
  \setupfontsynonym [SerifBold]        [handling=quality]
  \setupfontsynonym [SerifSlanted]     [handling=quality]
  \setupfontsynonym [SerifItalic]      [handling=quality]
  \setupfontsynonym [SerifBoldSlanted] [handling=quality]
  \setupfontsynonym [SerifBoldItalic]  [handling=quality]

  %setupfontsynonym [Mono]             [handling=quality] % sloooow

  % We use adobe metrics instead of urw metrics because tetex only
  % ships the former. Beware, these metrics differ!

  \usetypescript [adobekb] [\defaultencoding]
  \usetypescript [palatino][\defaultencoding]

  \setupbodyfont
    [palatino,10pt]

  \definefontsynonym[TitleFont][SerifBold]

\stopnotmode

\startmode[atpragma]

  \usetypescriptfile[type-ghz]

  \setupfontsynonym [Sans]            [handling=quality]
  \setupfontsynonym [SansBold]        [handling=quality]
  \setupfontsynonym [SansSlanted]     [handling=quality]
  \setupfontsynonym [SansItalic]      [handling=quality]
  \setupfontsynonym [SansBoldSlanted] [handling=quality]
  \setupfontsynonym [SansBoldItalic]  [handling=quality]

  %setupfontsynonym [Mono]             [handling=quality] % sloooow

  \definetypeface[optima][ss][sans][optima-nova] [default][encoding=\defaultencoding]
  \definetypeface[optima][tt][mono][latin-modern][default][encoding=\defaultencoding,rscale=1.1]

  \setupbodyfont[optima,10pt,ss]

  \definefontsynonym[TitleFont][SansBold]

\stopmode

%D This document is mildly interactive. Yes, Thanh's name will end up ok in
%D the document information data.

\setupinteraction
  [state=start,
   style=normal,
   color=,
   page=yes,
   openaction=firstpage,
   title=the pdfTeX users manual,
   author={H\`an Th\^e Th\`anh, Sebastian Rahtz, Hans Hagen, Hartmut Henkel, Pawe\l\ Jackowski}]

\setupinteractionscreen % still needed?

\startnotmode[screen]

\setupinteractionscreen
  [option=bookmark]

\stopnotmode

%D Some headers and footers will complete the layout.

\setupheadertexts
  [The pdf\TeX\ user manual]

\setupfootertexts
  [pagenumber]

%D For Tobias Burnus, who loves bookmarks, I've enabled them.

\placebookmarks
  [section,subsection,pdftexprimitive]

%D Let's also define a screen layout:

\startmode[screen] \environment pdftex-i \stopmode

%D We auto-cross link the paper and screen version:

\startnotmode[screen]

%\coupledocument
%  [screenversion]
%  [pdftex-s]
%  [section,subsection,subsubsection,pdftexprimitive]
%  [the screen version]

\setuphead
  [section,subsection,subsubsection,pdftexprimitive]
  [file=screenversion]

\setuplist
  [section]
  [alternative=c]

\stopnotmode

%D The text starts here!

\starttext

%D Being lazy, I don't split the file, so paper and screen get
%D mixed in one document.

\startmode[screen] \getbuffer[screen] \stopmode

\startnotmode[screen]

%D But first we typeset the title page. It has a background. This
%D background, showing a piece of \PDF\ code, will be referred to
%D later on.

%D We can use more modern \CONTEXT\ features, but for the moment
%D stick to the old style and methods.

\NormalizeFontWidth
  \TitleFont
  {\hskip.5mm The pdf\TeX\ user manual} % \hskip is fake offset
  \paperheight
  {TitleFont}

\setupbackgrounds
  [page]
  [background={title,joke,names,content}]

\defineoverlay
  [title]
  [\hbox to \paperwidth
     {\hfill
      \TitleFont\setstrut
      \rotate{The pdf\TeX\ user manual}%
      \hskip.5cm}] % \dp\strutbox}]

% \defineoverlay
%   [joke]
%   [\hbox to \paperwidth
%      {\TitleFont\setstrut
%       \dimen0=\overlaywidth
%       \advance\dimen0 by -\ht\strutbox
%       \advance\dimen0 by -\dp\strutbox
%       \advance\dimen0 by -1cm
%       \dimen2=\overlayheight
%       \advance\dimen2 by -1cm
%       \hskip.5cm\expanded{\externalfigure
%          [pdftex-z.pdf]
%          [width=\the\dimen0,height=\the\dimen2]}%
%       \hfill}]

% \defineoverlay
%   [names]
%   [\vbox to \paperheight
%      {\dontcomplain
%       \TitleFont\setstrut
%       \setbox0=\hbox{\TeX}%
%       \hsize\paperwidth
%       \rightskip\ht0
%       \advance\rightskip\dp\strutbox
%       \advance\rightskip\dp\strutbox
%       \bfa \setupinterlinespace
%       \vfill
%       \hfill \THANH \endgraf
%       \hfill Sebastian Rahtz \endgraf
%       \hfill Hans Hagen
%       \hfill Hartmut Henkel
%       \hfill Pawe\l\ Jackowski
%       \vskip 1ex
%       \hfill \currentdate
%       \vskip 2ex}]

\defineoverlay
  [content]
  [\overlaybutton{contents}]

% title page

\definelayout
  [titlepage]
  [backspace=.5cm,
   cutspace=3.5cm,
   topspace=.5cm,
   bottomspace=.5cm,
   header=0pt,
   footer=0pt,
   lines=\getvariable{pdftex}{titlepagelines},
   grid=yes,
   width=middle]

\definecolumnset
  [titlepage]
  [n=2,distance=0.5cm]

\start

    \chardef\fontdigits=2
    \switchtobodyfont[10pt,tt]
    \setupinterlinespace[line=\dimexpr((\paperheight-1cm)/\getvariable{pdftex}{titlepagelines})]
    \setuptyping[margin=,color=]
    \setuplayout[titlepage]

    \startcolumnset[titlepage]
      \typefile{pdftex-t.txt}
    \stopcolumnset

    \page
    \setuplayout

\stop

% \startstandardmakeup
%
% %D The titlepage is generated using the background overlay mechanism. This
% %D saves me the trouble of determining funny skips and alignments. So no text
% %D goes here.
%
% \stopstandardmakeup

\setupbackgrounds
  [page]
  [background=]

%D The inside title page is as follows.

\startstandardmakeup[footerstate=none]

  \dontcomplain

  \setupalign[left]

  \start

    \bfd The pdf\TeX\ user manual

    \bfa \setupinterlinespace

    \vskip4ex

    \THANH\par
    Sebastian Rahtz\par
    Hans Hagen\par
    Hartmut Henkel\par
    Pawe\l\ Jackowski\par

    \vskip3ex

    \rcsmonthname\ \rcsday, \rcsyear\par
    \vskip1ex
    Rev.\ \rcsrevision

  \stop

  \vfill

  \startlines
    The title page of this manual
    represents the plain \TeX\ coded
    text ``Welcome to pdf\TeX !''
  \stoplines

  \blank[2*big] \setuptyping[after=]

  \starttyping
    \pdfoutput=1
    \pdfcompresslevel=0
    \font\tenrm=ptmr8r
    \tenrm
    Welcome to pdf\TeX!
    \bye
  \stoptyping

\stopstandardmakeup

\setuppagenumber[number=1] % added in case of single sided usage

%D So far for non screen mode.

\stopnotmode

%D We start with a small table of contents, typeset in double column format.

\startfrontmatter

\subject[contents]{Contents}

\startcolumns[distance=3em]
  \placelist[section]
\stopcolumns

\stopfrontmatter

%D And here it is: the main piece of text.

\startbodymatter

%***********************************************************************

\section{Introduction}

The main purpose of the \PDFTEX\ project is to create and maintain
an extension of \TEX\ that can produce \PDF\ directly from \TEX\
source files and improve|/|enhance the result of \TEX\ typesetting
with the help of \PDF. When \PDF\ output is not selected, \PDFTEX\
produces normal \DVI\ output, otherwise it generates \PDF\ output
that looks identical to the \DVI\ output. An important aspect of this
project is to investigate alternative justification algorithms (\eg\
a~font expansion algorithm akin to the \HZ\ micro||typography algorithm by
Prof.~Hermann Zapf), optionally making use of multiple master fonts.

\PDFTEX\ is based on the original \TEX\ sources and \WEBC, and has been
successfully compiled on \UNIX, \WIN\ and \MSDOS\ systems. It is under
active development, with new features trickling in. Great care is taken
to keep new \PDFTEX\ versions backward compatible with earlier ones.

For some years there has been a~\quote {moderate} successor to \TEX\
available, called \ETEX. Because mainstream macro packages such as
\LATEX\ have started supporting this welcome extension, \PDFTEX\ also
is available as \PDFETEX.  Although in this document we will speak of
\PDFTEX, we advise users to use \PDFETEX\ when available. That way they
get the best of all worlds and are ready for the future. Starting with
\TEXLIVE\ 2004, that future has arrived: \PDFETEX\ is now the primary
\TEX\ engine.

Other extensions to \PDFTEX\ are \MLTEX\ and \ENCTEX; recent \PDFTEX\
engines have these often included.

\PDFTEX\ is maintained by \THANH, Martin Schr\"oder, Hans Hagen,
Hartmut Henkel, and others. The \PDFTEX\ homepage is \from [ptex_org].
Please send \PDFTEX\ comments and bug reports to the mailing list
\from [ptex_bugs].

We thank all readers who send us corrections and suggestions. We also
wish to express the hope that \PDFTEX\ will be of as much use to you
as it is to us.  Since \PDFTEX\ is still being improved and extended,
we suggest you to keep track of updates.

%***********************************************************************

\subsection{About this manual}

This manual revision (\rcsrevision) tries to keep track with the recent
\PDFTEX\ development up to version \currentpdftex. Main text updates
were done regarding the new configuration scheme, font mapping, and new
or updated primitives.  The primary repository for the manual and its
sources is at \from[ptex_sarovar].  Copies in \PDF\ format can also be
found at the CTAN network in directory \from[ptex_ctan].

Thanks to Karl Berry for proof reading and submitting a~long changes
list. New errors might have slipped in afterwards by the editor.
Please send questions or suggestions by email to \from[ptex_bugs].

%***********************************************************************

\subsection{Legal Notice}

Copyright \copyright\ 1996||2005  \THANH.
Permission is granted to copy, distribute and/or modify this document
under the terms of the GNU Free Documentation License, Version 1.2
or any later version published by the Free Software Foundation;
with no Invariant Sections, no Front-Cover Texts, and no Back-Cover Texts.
A~copy of the license is included in the section entitled ``GNU
Free Documentation License''.

%***********************************************************************

\section{About \PDF}

The cover of this manual lists an almost minimal \PDF\ file generated
by \PDFTEX, with the corresponding source file on the next page. Unless
compression is enabled, such a~\PDF\ file is rather verbose and readable.
The first line specifies the version used; currently \PDFTEX\ produces
level 1.4 output by default. \PDF\ viewers are supposed to silently skip
over all elements they cannot handle.

A~\PDF\ file consist of objects. These objects can be recognized by their
number and keywords:

\starttyping
7 0 obj << /Type /Catalog /Pages 5 0 R >> endobj
\stoptyping

Here \typ{7 0 obj ... endobj} is the object capsule. The first number is
the object number. Later we will see that \PDFTEX\ gives access to this
number. One can for instance create an object by using \type{\pdfobj} after
which \type{\pdflastobj} returns the number. So

\starttyping
\pdfobj{/Type /Catalog /Pages 5 0 R}
\stoptyping

inserts an object into the file, while \type{\pdflastobj} returns the
number \PDFTEX\ assigned to this object. The sequence \type{5 0 R} is
an object reference, a~pointer to another object (no.~5). The second
number (here a~zero) is currently not used in \PDFTEX; it is the version
number of the object. It is for instance used by \PDF\ editors, when they
replace objects by new ones. The version numbers permit a~roll||back. (An
example of a~graphic editor that uses \PDF\ as storage format is the
Adobe Illustrator.)

In general this rather direct way of pushing objects in the files is not very
useful, and only makes sense when implementing, say, fill||in field
support or annotation content reuse. We will come to that later. Unless such
direct objects are part of something larger, they will end up as isolated
entities, not doing any harm but not doing any good either.

When a~viewer opens a~\PDF\ file, it first goes to the end of the file. There
it finds the keyword \type{startxref}, the signal where to look for the so
called \quote {object cross reference table}. This table provides fast access
to the objects that make up the file. The actual starting point of the file
is defined after the \type{trailer}. The \type{/Root} entry points to the
catalog. In this catalog the viewer can find the page list. In our example we
have only one page. The trailer also holds an \type{/Info} entry, which tells
 a~bit more about the document. Just follow the thread:

\startnarrower
\type{/Root}     $\longrightarrow$ object~7 $\longrightarrow$
\type{/Pages}    $\longrightarrow$ object~5 $\longrightarrow$
\type{/Kids}     $\longrightarrow$ object~2 $\longrightarrow$
\type{/Contents} $\longrightarrow$ object~3
\stopnarrower

As soon as we add annotations, a~fancy word for hyperlinks and the like, some
more entries are present in the catalog. We invite users to take a~look at
the \PDF\ code of this file to get an impression of that.

The page content is a~stream of drawing operations. Such a~stream
can be compressed, where the level of compression can be set with
\type {\pdfcompresslevel}. Let's take a~closer look at this stream
in object~3. Often there is a~transformation matrix, six numbers
followed by \type{cm}. As in \POSTSCRIPT, the operator comes after
the operands. Between \type{BT} and \type{ET} comes the text. A~font
is selected by a~\type{Tf} operator, which is given a~resource name
\type{/F..} and the font size. The actual text goes into \type{()}
bracket pairs so that it creates a~\POSTSCRIPT\ string. The numbers
inbetween bracket pairs provide fine glyph positioning (kerning). When
one analyzes a~file produced by a~less sophisticated typesetting engine,
whole sequences of words can be recognized. In \PDF\ files generated by
\PDFTEX\ however, the words comes out rather fragmented, mainly because
a~lot of kerning takes place. \PDF\ viewers in search mode simply ignore the
kerning information in these text streams. When a~document is searched,
the search engine reconstructs the text from these (string) snippets.

This one page example uses an Adobe Times||Roman font. This is one of
the 14 so||called standard fonts that are always present in the viewer
application, and therefore need not be embedded in the \PDF\ file.
However, when we use for instance Computer Modern Roman, we have to
make sure that this font is available, and the best way to do this is
to embed it. Just let your eyes follow the object thread and see how
 a~font is described.  The only thing removed from this example is the
(partially) embedded glyph description file, which for the 14 standard
fonts is not needed.

In this simple file, we don't specify in what way the file should be opened,
for instance full screen or clipped. A~closer look at the page object no.~2
(\typ{/Type /Page}) shows that a~mediabox (\typ{/MediaBox}) is part of the
page description. A~mediabox acts like the (high-resolution) bounding box
in a~\POSTSCRIPT\ file. \PDFTEX\ users can add dictionary stuff to page
objects by the \type{\pdfpageattr} primitive.

Although in most cases macro packages will shield users from these internals,
\PDFTEX\ provides access to many of the entries described here, either
automatically by translating the \TEX\ data structures into \PDF\ ones, or
manually by pushing entries to the catalog, page, info or self created
objects. Those who, after this introduction, feel unsure how to
proceed, are advised to read on but skip \in{section}[primitives]. Before we
come to that section, we will describe how to get started with \PDFTEX.

%***********************************************************************

\section{Getting started}

This section describes the steps needed to get \PDFTEX\ running on
a~system where \PDFTEX\ is not yet installed.  Nowadays virtually all \TEX\
distributions have \PDFTEX\ as a~component, such as \TEXLIVE, \TETEX,
\FPTEX, \MIKTEX, and \CMACTEX.  The ready to run \TEXLIVE\ distribution
comes with \PDFTEX\ versions for many \UNIX, \WIN, and \MACOSX\ systems; more
information can be found at \from[texlive]. \TETEX\ by Thomas Esser is
 a~source distribution with an automated compilation process for \UNIX\
systems; see \from[tetex].  For \WIN\ systems there are also two separate
distributions that contain \PDFTEX, both in \from[win32]: \FPTEX\ by
Fabrice Popineau and \MIKTEX\ by Christian Schenk.  So when you use
any of these distributions, you don't need to bother with the \PDFTEX\
installation procedure in the next sections.

If there is no precompiled binary of \PDFTEX\ for your system, or the
version coming with a~distribution is not the current one and you would
like to try out a~fresh \PDFTEX\ immediately, you will need to build
\PDFTEX\ from sources; read on.  You should already have a~working \TEX\
system, \eg\ \TETEX, into which the freshly compiled \PDFTEX\ will
be integrated.  Note that the installation description in this manual
is \WEBC||specific.

%***********************************************************************

\subsection{Getting sources and binaries}

The latest sources of \PDFTEX\ are currently distributed for compilation
on \UNIX\ systems (including Linux), and \WIN\ systems (Windows 95,
98, NT, 2000, XP).  The primary location where one can fetch the latest
released code is at the developers' homepage \from[ptex_sarovar], where
you also find bug tracking information, and the manual sources. Download
the \PDFTEX\ archive from there.

The \PDFTEX\ sources can also be found at their canonical place in
the CTAN network, \from[ptex_ctan].
Separate \PDFTEX\ binaries for various systems might also be available,
check out the subdirectories below \from[ctan_systems].

%***********************************************************************

\subsection{Compiling}

The compilation is expected to be easy on \UNIX||like systems and
can be described best by example. Assuming that the file \filename
{pdftex.zip} is downloaded to some working directory, \eg\
\filename {\$HOME/pdftex}, on a~\UNIX\ system the following steps are
needed to compile \PDFTEX:

\startesctyping
cd $HOME/pdftex
unzip pdftex-@currentpdftex.zip
cd pdftex-@currentpdftex
./Build
\stopesctyping

The binaries \filename{pdftex} and \filename{pdfetex} are then
built in the subdirectory \filename{build/texk/web2c}. In the same
directory also the corresponding pool files \filename{pdftex.pool}
and \filename{pdfetex.pool} are generated, that are needed for creating
formats.

%***********************************************************************

\subsection{Placing files}

The next step is to put the freshly compiled binaries and pool
files into their proper places within the \TDS\ structure
of the \TEX\ system.  Put the files \filename{pdftex} and
\filename{pdfetex} into the directory (\eg\ for a~typical \TETEX\ system)
\filename{/usr/local/teTeX/bin/i686-pc-linux-gnu}, and the pool files
into \filename{/usr/local/teTeX/share/texmf/web2c}.

Don't forget to do a~\filename{texconfig-sys init} afterwards,
so that all formats are regenerated system-wide with the fresh binaries.

%***********************************************************************

\subsection{Setting search paths}

\WEBC||based programs, including \PDFTEX, use the \WEBC\ run||time
configuration file called \filename {texmf.cnf}.  The location
of this file is the appropriate position within the \TDS\ tree
relative to the place of the \PDFTEX\ binary; on a~\TETEX\ system,
file \filename{texmf.cnf} typically is located either in directory
\filename{texmf/web2c} or \filename{texmf-local/web2c}.  The path to
file \filename{texmf.cnf} can also be set up by the environment variable
\type{TEXMFCNF}.

Next you might need to edit \filename {texmf.cnf} so that \PDFTEX\
can find all necessary files, but the \filename{texmf.cnf} files
coming with the major \TEX\ distributions should already be set up for
normal use. You might check into the file \filename{texmf.cnf} to see
where the various bits and pieces are going.

\PDFTEX\ uses the search path variables shown in \in{table}[tbl:spathvar].

\startbuffer
\starttabulate[|l|l|]
\HL
\NC \bf used for        \NC \bf texmf.cnf       \NC\NR
\HL
\NC output files        \NC \type{TEXMFOUTPUT}  \NC\NR
\NC input files, images \NC \type{TEXINPUTS}    \NC\NR
\NC format files        \NC \type{TEXFORMATS}   \NC\NR
\NC text pool files     \NC \type{TEXPOOL}      \NC\NR
\NC encoding files      \NC \type{ENCFONTS}     \NC\NR
\NC font map files      \NC \type{TEXFONTMAPS}  \NC\NR
\NC \TFM\ files         \NC \type{TFMFONTS}     \NC\NR
\NC virtual fonts       \NC \type{VFFONTS}      \NC\NR
\NC type1 fonts         \NC \type{T1FONTS}      \NC\NR
\NC TrueType fonts      \NC \type{TTFONTS}      \NC\NR
\NC pixel fonts         \NC \type{PKFONTS}      \NC\NR
\HL
\stoptabulate
\stopbuffer

\placetable[here][tbl:spathvar]
  {The \WEBC\ variables.}
  {\getbuffer}

\PathDescription {TEXMFOUTPUT} Normally, \PDFTEX\ puts its output files
in the current directory. If any output file cannot be opened there, it
tries to open it in the directory specified in the environment variable
\type{TEXMFOUTPUT}. There is no default value for that variable. For
example, if you type \type{pdfetex paper} and the current directory is
not writable, if \type{TEXMFOUTPUT} has the value \type{/tmp}, \PDFTEX\
attempts to create \type{/tmp/paper.log} (and \type{/tmp/paper.pdf},
if any output is produced.)

\PathDescription {TEXINPUTS} This variable specifies where \PDFTEX\ finds
its input files. Image files are considered
input files and searched for along this path.

\PathDescription {TEXFORMATS} Search path for format (\type{.fmt}) files.

\PathDescription {TEXPOOL} Search path for pool (\type{.pool}) files.

\PathDescription {ENCFONTS} Search path for encoding (\type{.enc}) files.

\PathDescription {TEXFONTMAPS} Search path for font map (\type{.map}) files.

\PathDescription {TFMFONTS} Search path for font metric (\type{.tfm}) files.

\PathDescription {VFFONTS} Search path for virtual font (\type{.vf})
files.  Virtual fonts are fonts made up of other fonts.
Because \PDFTEX\ produces the
final output code, it must consult those files.

\PathDescription {T1FONTS} Search path for Type~1 font files (\type{.pfa}
and \type{.pfb}).  These outline (vector) fonts are to be preferred over
bitmap \PK\ fonts. In most cases Type~1 fonts are used and this variable
tells \PDFTEX\ where to find them.

\PathDescription {TTFFONTS} Search path for TrueType font (\type{.ttf})
files.  Like Type~1 fonts, TrueType fonts are also outlines.

\PathDescription {PKFONTS} Search path for packed (bitmap) font
(\type{.pk}) files.
Unfortunately bitmap fonts are still displayed poorly by some \PDF\
viewers, so when possible one should use outline fonts. When no outline
is available, \PDFTEX\ tries to locate a~suitable \PK\ font (or invoke
 a~process that generates it).

%***********************************************************************

\subsection[cfg]{The \PDFTEX\ configuration}

One has to keep in mind that, as opposed to \TEX\ with its \DVI\ output,
the \PDFTEX\ program does not require a~separate postprocessing stage
to transform the \TEX\ input into a~\PDF\ file. As a~consequence, all
data needed for building a~ready \PDF\ page must be available
during the \PDFTEX\ run, in particular information on media dimensions
and offsets, graphics files for embedding, and font information (font
files, encodings).

When \TEX\ builds a~page, it places items relative to the top left page
corner (the \DVI\ reference point). Separate \DVI\ postprocessors allow
specifying the paper size (\eg\ \quote {A4} or \quote{letter}), so
that this reference point is moved to the correct position on the paper,
and the text ends up at the right place.

In \PDF, the paper dimensions are part of the page definition, and
\PDFTEX\ therefore requires that they be defined at the beginning of
the \PDFTEX\ run. As with pages described by \POSTSCRIPT, the \PDF\
reference point is in the lower||left corner.

Formerly, these dimensions and other \PDFTEX\ parameters were read
in from a~configuration file named \filename{pdftex.cfg}, which had
a~special (non-\TEX) format, at the start of processing.  Nowadays such
a~file is ignored by \PDFTEX.  Instead, the page dimensions and offsets,
as well as many other parameters, can be set by \PDFTEX\ primitives
during the \PDFTEX\ format building process, so that the settings are
dumped into the fresh format and consequently will be used when \PDFTEX\
is later called with that format. All settings from the format can
still be overridden during a~\PDFTEX\ run by using the same primitives.
This new configuration concept is a~more unified approach, as it avoids
the configuration file with a~special format.

A~list of \PDFTEX\ primitives relevant for setting up the \PDFTEX\ engine
is given in \in{table}[tbl:configparms]. All primitives are described in
detail within later sections. \in{Figure}[in:pdftexconfig] shows a~recent
configuration file (\type{pdftexconfig.tex}) in \TEX\ format, using the
primitives from \in{table}[tbl:configparms], which typically is read
in during the format building process. It enables \PDF\ output, sets paper
dimensions and the default pixel density for \PK\ font inclusion. The default
values are chosen so that \PDFTEX\ often can be used (\eg\ in \type{-ini} mode)
even without setting any parameters.

\startbuffer
\starttabulate[|l|l|l|l|l|]
\HL
\NC \bf internal name              \NC \bf type  \NC\bf default\NC\bf comment\NC\NR
\HL
\NC \type{\pdfoutput}              \NC integer   \NC      0 \NC \DVI                 \NC\NR
\NC \type{\pdfadjustspacing}       \NC integer   \NC      0 \NC off                  \NC\NR
\NC \type{\pdfcompresslevel}       \NC integer   \NC      9 \NC best                 \NC\NR
\NC \type{\pdfdecimaldigits}       \NC integer   \NC      4 \NC max.                 \NC\NR
\NC \type{\pdfimageresolution}     \NC integer   \NC     72 \NC dpi                  \NC\NR
\NC \type{\pdfpkresolution}        \NC integer   \NC      0 \NC 72\,dpi              \NC\NR
\NC \type{\pdfpkmode}              \NC token reg.\NC  empty \NC mode set in \type{mktex.cnf} \NC\NR
\NC \type{\pdfuniqueresname}       \NC integer   \NC      0 \NC                      \NC\NR
\NC \type{\pdfprotrudechars}       \NC integer   \NC      0 \NC                      \NC\NR
\NC \type{\pdfminorversion}        \NC integer   \NC      4 \NC \PDF\ 1.4            \NC\NR
\NC \type{\pdfforcepagebox}        \NC integer   \NC      0 \NC                      \NC\NR
\NC \type{\pdfinclusionerrorlevel} \NC integer   \NC      0 \NC                      \NC\NR
%-----------------------------------------------------------------------
\NC \type{\pdfhorigin}             \NC dimension \NC  1\,in \NC                      \NC\NR
\NC \type{\pdfvorigin}             \NC dimension \NC  1\,in \NC                      \NC\NR
\NC \type{\pdfpagewidth}           \NC dimension \NC  0\,pt \NC                      \NC\NR
\NC \type{\pdfpageheight}          \NC dimension \NC  0\,pt \NC                      \NC\NR
\NC \type{\pdflinkmargin}          \NC dimension \NC  0\,pt \NC                      \NC\NR
\NC \type{\pdfdestmargin}          \NC dimension \NC  0\,pt \NC                      \NC\NR
\NC \type{\pdfthreadmargin}        \NC dimension \NC  0\,pt \NC                      \NC\NR
\NC \type{\pdfmapfile}             \NC text      \NC \filename{pdftex.map} \NC not dumped\NC\NR
\HL
\stoptabulate
\stopbuffer

\placetable[here][tbl:configparms]
  {The set of \PDFTEX\ configuration parameters.}
  {\getbuffer}

\startbuffer
\tx\setupinterlinespace
\startframedtext
\starttyping
% Set pdfTeX parameters for pdf mode (replacing pdftex.cfg file).
% Thomas Esser, 2004. public domain.
\pdfoutput=1
\pdfpagewidth=210 true mm
\pdfpageheight=297 true mm
\pdfpkresolution=600
\endinput
\stoptyping
\stopframedtext
\stopbuffer

\placefigure[here][in:pdftexconfig]
  {A typical configuration file (\filename{pdftexconfig.tex}).}
  {\getbuffer}

Independent of whether such a~configuration file is read or not, the
first action in a~\PDFTEX\ run is that the program reads the global \WEBC\
configuration file (\filename{texmf.cnf}), which is common to all programs
in the \WEB2C\ system. This file mainly defines file search paths, the
memory layout (\eg\ pool and hash size), and other general parameters.

%***********************************************************************

\subsection{Creating format files}

\startbuffer
\tx\setupinterlinespace
\startframedtext
\starttyping
% Thomas Esser, 1998, 2004. public domain.
\ifx\pdfoutput\undefined
\else
  \ifx\pdfoutput\relax
  \else
    \input pdftexconfig
    \pdfoutput=0
  \fi
\fi
\input etex.src
\dump
\endinput
\stoptyping
\stopframedtext
\stopbuffer

\placefigure[here][in:etexini]
  {File \type{etex.ini} for \ETEX\ format with \DVI\ output.}
  {\getbuffer}

\startbuffer
\tx\setupinterlinespace
\startframedtext
\starttyping
\ifx\pdfoutput\undefined
\else
  \ifx\pdfoutput\relax
  \else
    \input pdftexconfig
    \pdfoutput=1
  \fi
\fi
\scrollmode
\input latex.ltx
\endinput
\stoptyping
\stopframedtext
\stopbuffer

\placefigure[here][in:pdflatexini]
  {File \type{pdflatex.ini} for \LATEX\ format with \PDF\ output.}
  {\getbuffer}

Both \PDFTEX\ and \PDFETEX\ engines allow building formats for \DVI\ and
\PDF\ output in the same way as the classical \TEX\ engine does for \DVI.
Format generation is enabled by the \type{-ini} option. The default mode
(\DVI\ or \PDF) can be chosen either on the command line by setting
the option \type{-output-format} to \type{dvi} or \type{pdf}, or by
setting the \type{\pdfoutput} parameter. The format file then inherits
this setting, so that a~later call to \PDFTEX\ with this format starts
in the preselected mode (which still can be overrun then). A~format
file can be read in only by the engine that has generated it; a~format
incompatible with an engine leads to a~fatal error. Often the \PDFTEX\
program is a~mere link to the \PDFETEX\ engine; then also a~\PDFTEX\
call generates an extended format.

It is customary to package the configuration and macro file input
into a~\type{.ini} file.  \Eg, the file \type{etex.ini} in
\in{figure}[in:etexini] is for generating an \ETEX\ format with \DVI\
output (it contains a~few comparisons to be safe also for \TEX\ engines).
 A~similar file \type{pdflatex.ini} can be used for generating a~\LATEX\
format with \PDF\ output; refer to \in{figure}[in:pdflatexini].
One can see how the primitive \type{\pdfoutput} is used to override
the output mode set by file \type{pdftexconfig.tex}. The corresponding
\PDFTEX\ calls for format generation are:

\starttyping
pdfetex -ini *etex.ini
pdftex -ini pdflatex.ini
\stoptyping

These calls produce format files \filename{etex.fmt} and
\filename{pdflatex.fmt}, as the default format file name is taken from
the input file name.  You can overrule this with the \type{-jobname}
option.  The asterisk \type{*} in the first example line tells the
\PDFETEX\ engine to go into extended \type{-ini} mode; otherwise it
stays in non||extended mode. The \PDFTEX\ engine can't be brought into
extended mode at all; it interprets an asterisk \type{*} in front
of a~file name as part of the file name.  So, if you want a~\PDFLATEX\
format with \PDF\ output and \ETEX\ extensions available (format file
\filename{pdfelatex.fmt}), you would need to type \eg:

\starttyping
pdfetex -ini -jobname=pdfelatex *pdflatex.ini
\stoptyping

In \CONTEXT\ the generation depends on the interface used. A~format using the
English user interface is generated with

\starttyping
pdfetex -ini cont-en
\stoptyping

When properly set up, one can also use the \CONTEXT\ command line interface
\TEXEXEC\ to generate one or more formats, like:

\starttyping
texexec --make en
\stoptyping

for an English format, or

\starttyping
texexec --make en de
\stoptyping

for an English and German one. Most users will simply say:

\starttyping
texexec --make --all [--alone]
\stoptyping

and so generate the \TEX\ and \METAPOST\ related formats that \CONTEXT\
needs. Whatever macro package used, the formats should be placed in the \type
{TEXFORMATS} path.

\subsection{Testing the installation}

When everything is set up, you can test the installation. In the distribution
there is a~plain \TEX\ test file \filename {example.tex}. Process this file
by typing:

\starttyping
pdftex example
\stoptyping

If the installation is ok, this run should produce a~file called \filename
{example.pdf}. The file \filename {example.tex} is also a~good place to look
for how to use \PDFTEX's primitives.

%***********************************************************************

\subsection{Common problems}

The most common problem with installations is that \PDFTEX\ complains that
something cannot be found. In such cases make sure that \type{TEXMFCNF} is
set correctly, so \PDFTEX\ can find \filename {texmf.cnf}. The next best
place to look|/|edit is the file \type{texmf.cnf}. When still in deep
trouble, set \type{KPATHSEA_DEBUG=255} before running \PDFTEX\ or run
\PDFTEX\ with option \type{-k 255}. This will cause \PDFTEX\ to write a~lot
of debugging information that can be useful to trace problems. More options
can be found in the \WEBC\ documentation.

Variables in \filename {texmf.cnf} can be overwritten by environment
variables. Here are some of the most common problems you can encounter when
getting started:

\startitemize

\head  \type{I can't read pdftex.pool; bad path?}

       \type{TEXMFCNF} is not set correctly and so \PDFTEX\ cannot find
       \type{texmf.cnf}, or \type{TEXPOOL} in \filename {texmf.cnf}
       doesn't contain a~path to the pool file \filename {pdftex.pool} or
       \type{pdfetex.pool} when you use \PDFETEX.

\head  \type{You have to increase POOLSIZE.}

       \PDFTEX\ cannot find \filename {texmf.cnf}, or the value of \type
       {pool_size} specified in \filename {texmf.cnf} is not large enough
       and must be increased. If \type{pool_size} is not specified in
       \filename {texmf.cnf} then you can add something like

\starttyping
pool_size=500000
\stoptyping

\head  \type{I can't find the format file `pdftex.fmt'!} \crlf
       \type{I can't find the format file `pdflatex.fmt'!}

       The format file is not created (see above how to do that) or
       is not properly placed. Make sure that \type{TEXFORMATS} in
       \filename {texmf.cnf} contains the path to \filename {pdftex.fmt}
       or \filename {pdflatex.fmt}.

\head  \type{---! xx.fmt was written by tex} \crlf
       \type{Fatal format file error; I'm stymied}

       This appears \eg\ if you forgot to regenerate the \type{.fmt}
       files after installing a~new version of the \PDFTEX\ binary and
       \filename {pdftex.pool}. The first line tells by which engine
       the offending format was generated.

\head  \type{TEX.POOL doesn't match; TANGLE me again!} \crlf
       \type{TEX.POOL doesn't match; TANGLE me again (or fix the path).}

       This might appear if you forgot to install the proper \filename
       {pdftex.pool} when installing a~new version of the \PDFTEX\
       binary. \Eg\ under \TETEX\ then run \type{texconfig-sys init}.

\head  \type{! I can't find file `*pdftex.ini'.} \crlf
       \type{<*> *pdftex.ini}

       This typically appears when you try to generate an extended format
       with the \PDFTEX\ engine (it does not know about the special asterisk
       \type{*} notation). Use the \PDFETEX\ engine instead.

\head  \PDFTEX\ cannot find one or more map files (\type{*.map}),
       encoding vectors (\type{*.enc}), virtual fonts, Type~1 fonts,
       TrueType fonts or some image file.

       Make sure that the required file exists and the corresponding variable
       in \filename {texmf.cnf} contains a~path to the file. See above which
       variables \PDFTEX\ needs apart from the ones \TEX\ uses.

       When you have installed new fonts, and your \PDF\ viewer complains
       about missing fonts, you should take a~look at the log file produced
       by \PDFTEX. Missing fonts, map files, encoding vectors as well as
       missing characters (glyphs) are reported there.

\stopitemize

%MS it would be nice if we could get the number of objects directly from TeX,
%MS but currently even the line given at the end of the log "PDF objects out
%MS of" is only a minimum as it doesn't include all the objects from the page
%MS tree and document info...
Normally the page content takes one object. This means that one seldom finds
more than a~few hundred objects in a~simple file. This document for instance
uses about 900 objects. In demanding applications this number can grow quite
rapidly, especially when one uses a~lot of widget annotations, shared
annotations or other shared things. In these situations in \filename
{texmf.cnf} one can enlarge \PDFTEX 's internal object table by adding a~line
in \filename {texmf.cnf}, for instance:

\starttyping
obj_tab_size=400000
\stoptyping

%***********************************************************************

\section{Macro packages supporting \PDFTEX}

As \PDFTEX\ generates the final \PDF\ output without help of
a~postprocessor, macro packages that take care of these \PDF\ features
have to be set up properly.  Typical tasks are handling color,
graphics, hyperlink support, threading, font||inclusion, as well as
page imposition and manipulation.  All these \PDF||specific tasks can be
commanded by \PDFTEX's own primitives (a few also by a~\PDFTEX||specific
\type{\special{pdf: ...}} primitive). Any other \type{\special{}} commands,
like the ones defined for various \DVI\ postprocessors, are simply
ignored by \PDFTEX\ when in \PDF\ output mode; a~warning is given only
for non||empty \type{\special{}} commands.

When a~macro package already written for classical \TEX\ with \DVI\ output
is to be modified for use with \PDFTEX, it is very helpful to get some
insight to what extent \PDFTEX||specific support is needed. This info can
be gathered \eg\ by outputting the various \type{\special} commands
as \type{\message}. Simply type

\starttyping
\pdfoutput=1 \let\special\message
\stoptyping

or, if this leads to confusion,

\starttyping
\pdfoutput=1 \def\special#1{\write16{special: #1}}
\stoptyping

and see what happens. As soon as one \quote {special} message turns up, one
knows for sure that some kind of \PDFTEX\ specific support is needed, and
often the message itself gives a~indication of what is needed.

Currently all mainstream macro packages offer \PDFTEX\ support, with
automatic detection of \PDFTEX\ as engine. So there is normally no need
to turn on \PDFTEX\ support explicitly.

\startitemize

\item  For \LATEX\ users, Sebastian Rahtz' and Heiko Oberdiek's \type
       {hyperref} package has substantial support for \PDFTEX\ and
       provides access to most of its features.  In the simplest and
       most common case, the user merely needs to load \type{hyperref},
       and all cross||references will be converted to \PDF\ hypertext
       links. \PDF\ output is automatically selected, compression is
       turned on, and the page size is set up correctly. Bookmarks are
       created to match the table of contents.

\item  The standard \LATEX\ \type{graphics}, \type{graphicx}, and
       \type{color} packages also have automatic \pdfTeX\ support, which
       allow use of color, text rotation, and graphics inclusion commands.

\item  The \CONTEXT\ macro package by Hans Hagen has very full support
       for \PDFTEX\ in its generalized hypertext features. Support for
       \PDFTEX\ is implemented as a~special driver, and is invoked by
       typing \type{\setupoutput[pdftex]} or feeding \TEXEXEC\ with the
       \type{--pdf} option.

\item  \PDF\ from \TEXINFO\ documents can be created by running \PDFTEX\ on
       the \TEXINFO\ file, instead of \TEX.  Alternatively, run the shell
       command \type{texi2pdf} instead of \type{texi2dvi}.

\item A~small modification of \filename {webmac.tex}, called \filename
       {pdfwebmac.tex}, allows production of hyperlinked \PDF\ versions
       of the program code written in \WEB.

\stopitemize

Some nice samples of \PDFTEX\ output can be found at
\from[ptex_org], \from[context], and \from[tex_showcase].

%***********************************************************************

\section{Setting up fonts}

\PDFTEX\ can work with Type~1 and TrueType fonts, but a~source must be
available for all fonts used in the document, except for the 14 standard
fonts supplied by the \PDF\ reader (Times, Helvetica, Courier, Symbol
and Dingbats).  It is possible to use \METAFONT||generated fonts in
\PDFTEX\ --- but it is strongly recommended not to use these fonts if an
equivalent is available in Type~1 or TrueType format, if only because
bitmap Type~3 fonts render very poorly in (older versions of) Adobe
Reader. Given the
free availability of Type~1 versions of all the Computer Modern fonts,
and the ability to use standard \POSTSCRIPT\ fonts, there is rarely
 a~need to use bitmap fonts in \PDFTEX.

%***********************************************************************

\subsection[mapfile]{Map files}

Font map files provide the connection between \TEX\ \TFM\ font files
and the outline font file names. They contain also information about
re||encoding arrays, partial downloading, and character transformation
parameters (like SlantFont and ExtendFont). Those map files were first
created for \DVI\ postprocessors. But, as \PDFTEX\ in \PDF\ output mode
includes all \PDF\ processing steps, it also needs to know about font
mapping, and therefore reads in one or more map files.  Map files are
not read in when \PDFTEX\ is in \DVI\ mode. Pixel fonts can be used
without being listed in the map file.

By default, \PDFTEX\ reads the map file \filename{pdftex.map}.  In \WEBC,
map files are searched for using the \type{TEXFONTMAPS} config file value
and environment variable.  By default, the current directory and various
system directories are searched.

Within the map file, each font is listed on an individual line.  The syntax of
each line is upward||compatible with \type{dvips} map files and can contain
the following fields (some are optional; explanations follow):

\startnarrower
{\em tfmname basename fontflags special encodingfile fontfile}
\stopnarrower

It is mandatory that {\em tfmname} is the first field. If
 a~{\em basename} is given, it must be the second field. Similarly if
{\em fontflags} is given it must be the third field (if {\em basename}
is present) or the second field (if {\em basename} is left out). It is
possible to mix the positions of {\em special}, {\em encodingfile},
and {\em fontfile}, however the first three fields must be given in
fixed order.

\startdescription {tfmname}

sets the name of the \TFM\ file for a~font --- the name \TEX\ sees.
This name must always
be given.

\stopdescription

\startdescription {basename}

sets the base (\POSTSCRIPT) font name.  The {\em basename} field is
checked against the BaseName entry of fonts coming with embedded \PDF\
files.  If there is a~match, the font will be removed from the embedded
file, and a~local font is opened, which will contain the glyphs from the
embedded file. This collecting mechanism helps keeping the resulting \PDF\
file size small, if many files with similar fonts are to be embedded.
Therefore it is recommended always to set the {\em basename} field.

If a~{\em basename} field is given, also a~{\em fontfile} field must
be there, unless the {\em basename} matches one of the 14 standard font
names; then the {\em fontfile} field is optional. If the {\em fontfile}
name is given, this font will be embedded (depending on flags, see below).
If the {\em fontfile} name for a~standard font is missing, the font will
be quietly left out, which is fine, as \PDF\ viewers will later render
the text with their own versions of the font.

\stopdescription

\startdescription {fontflags}

specify some characteristics of the font. The following description of
these flags is taken, with slight modification, from the \PDF\ Reference
Manual (the section on font descriptor flags). Viewers can adapt their
rendering to these flags, especially when they substitute a~replacements
for not embedded fonts.

\startnarrower

The value of the flags key in a~font descriptor is a~32||bit integer that
contains a~collection of boolean attributes. These attributes are true if the
corresponding bit is set to~1. \in{Table}[flags] specifies the meanings of
the bits, with bit~1 being the least significant. Reserved bits must be set
to zero.

\startbuffer
\starttabulate[|c|l|]
\HL
\NC \bf bit position \NC \bf semantics                               \NC\NR
\HL
\NC 1                \NC Fixed||width font                           \NC\NR
\NC 2                \NC Serif font                                  \NC\NR
\NC 3                \NC Symbolic font                               \NC\NR
\NC 4                \NC Script font                                 \NC\NR
\NC 5                \NC Reserved                                    \NC\NR
\NC 6                \NC Uses the Adobe Standard Roman Character Set \NC\NR
\NC 7                \NC Italic                                      \NC\NR
\NC 8--16            \NC Reserved                                    \NC\NR
\NC 17               \NC All||cap font                               \NC\NR
\NC 18               \NC Small||cap font                             \NC\NR
\NC 19               \NC Force bold at small text sizes              \NC\NR
\NC 20--32           \NC Reserved                                    \NC\NR
\HL
\stoptabulate
\stopbuffer

\placetable
  [here][flags]
  {The meaning of flags in the font descriptor.}
  {\getbuffer}

All characters in a~{\em fixed||width} font have the same width, while
characters in a~proportional font have different widths. Characters in a~{\em
serif font} have short strokes drawn at an angle on the top and bottom of
character stems, while sans serif fonts do not have such strokes. A~{\em
symbolic font} contains symbols rather than letters and numbers. Characters
in a~{\em script font} resemble cursive handwriting. An {\em all||cap} font,
which is typically used for display purposes such as titles or headlines,
contains no lowercase letters. It differs from a~{\em small||cap} font in
that characters in the latter, while also capital letters, have been sized
and their proportions adjusted so that they have the same size and stroke
weight as lowercase characters in the same typeface family.

Bit~6 in the flags field indicates that the font's character set
conforms to the
Adobe Standard Roman Character Set, or a~subset of that, and that it uses
the standard names for those characters.

Finally, bit~19 is used to determine whether or not bold characters are
drawn with extra pixels even at very small text sizes. Typically, when
characters are drawn at small sizes on very low resolution devices such
as display screens, features of bold characters may appear only one pixel
wide. Because this is the minimum feature width on a~pixel||based device,
ordinary non||bold characters also appear with one||pixel wide features,
and thus cannot be distinguished from bold characters. If bit~19 is set,
features of bold characters may be thickened at small text sizes.

\stopnarrower

If the font flags are not given, \PDFTEX\ treats it as being~4,
a~symbolic font. If you do not know the correct value, it is best not to
specify it at all, as specifying a~bad value of font flags may cause
troubles in viewers. On the other hand this option is not absolutely
useless because it provides backward compatibility with older map files
(see the {\em fontfile} description below).

\stopdescription

\startdescription {special}

instructions can be used to manipulate fonts similar to the way
\type{dvips} does. Currently only the keywords \type{SlantFont}
and \type{ExtendFont} are interpreted, other instructions (as
\type{ReEncodeFont} with parameters, see {\em encoding} below) are
just ignored.  The permitted \type{SlantFont} range is $-$1..1;
for \type{ExtendFont} it's $-$2..2.  The block of {\em special}
instruction must be enclosed by double quotes \type{"}.

\stopdescription

\startdescription {encoding}

specifies the name of the file containing the external encoding vector to be
used for the font. The file name may be preceded by a~\type{<}, but the
effect is the same. The format of the encoding vector is identical to that
used by \type{dvips}. If no encoding is specified, the font's built||in
default encoding is used. It may be omitted if you are sure that the font
resource has the correct built||in encoding. In general this option is highly
preferred and is {\em required} when subsetting a~TrueType font.

\stopdescription

\startdescription {fontfile}

sets the name of the font source file. This must be a~Type~1 or TrueType font
file. The font file name can be preceded by one or two special characters,
which says how the font file should be handled.

\startitemize

\item  If the font file name is preceded by a~\type{<} the font file
       will be partially downloaded, meaning that only used glyphs
       (characters) are
       embedded to the font. This is the most common use and is
       {\em strongly recommended} for any font, as it ensures the
       portability and reduces the size of the \PDF\ output. Partial fonts
       are included in such a~way that name and cache clashes are minimized.

\item  If the font file name is preceded by a~double \type{<<}, the font
       file will be included entirely --- all glyphs of the font are
       embedded, including the ones that are not used in the document. Apart
       from causing large size \PDF\ output, this option may cause troubles
       with TrueType fonts, so it is not recommended. It might be useful
       in case the font is atypical and can not be subsetted well by
       \PDFTEX. {\em Beware: some font vendors forbid full font inclusion.}

\item  If nothing precedes the font file name, the font file is read but
       nothing is embedded, only the font parameters are extracted to
       generate the so||called font descriptor, which is used by the \PDF\
       reader to simulate the font if needed. This option is useful only when
       you do not want to embed the font (i.\,e.~to reduce the output size),
       but wish to use the font metrics and let the \PDF\ reader generate
       instances that look close to the used font in case the font resource
       is not installed on the system where the \PDF\ output will be viewed
       or printed. To use this feature the font flags {\em must} be
       specified, and it must have the bit~6 set on, which means that only
       fonts with the Adobe Standard Roman Character Set can be simulated.
       The only exception is the case of a~Symbolic font, which is not very
       useful.

\stopitemize

When one suffers from invalid lookups, for instance when \PDFTEX\ tries
to open a~\type{.pfa} file instead of a~\type{.pfb} one, one can add
the suffix to the filename.  In this respect, \PDFTEX\ completely relies
on the \type{kpathsea} libraries.

\stopdescription

%HE Huh? pgc?
If a~used font is not present in the map files, first \PDFTEX\ will
look for a~source with suffix \type{.pgc}, which is a~so||called \PGC\
source (\PDF\ Glyph Container) \footnote {This is a~text file containing
 a~\PDF\ Type~3 font, created by \METAPOST\ using some utilities by Hans
Hagen. In general \PGC\ files can contain whatever is allowed in a \PDF\ page
description, which may be used to support fonts that are not available
in \METAFONT. \PGC\ fonts are not widely useful, as vector Type~3 fonts
are not displayed very well in older versions of Acrobat Reader, but may
be more useful when better Type~3 font handling is more common.}. If no
\PGC\ source is available, \PDFTEX\ will try to use \PK~fonts as \DVI\
drivers do, creating \PK~fonts on||the||fly if needed.

Lines containing nothing apart from {\em tfmname} stand for scalable Type~3
fonts. For scalable fonts as Type~1, TrueType and scalable Type~3 font, all
the fonts loaded from a~\TFM\ at various sizes will be included only once in
the \PDF\ output. Thus if a~font, let's say \type{csr10}, is described in one
of the map files, then it will be treated as scalable. As a~result the font
source for csr10 will be included only once for \type{csr10}, \type{csr10
at 12pt} etc. So \PDFTEX\ tries to do its best to avoid multiple downloading
of identical font sources. Thus vector \PGC\ fonts should be specified as
scalable Type~3 in map files like:

\starttyping
csr10
\stoptyping

It doesn't hurt much if a~scalable Type~3 font is not given in map files,
except that the font source will be downloaded multiple times for various
sizes, which causes a~much larger \PDF\ output. On the other hand if a~font
in the map files is defined as scalable Type~3 font and its \PGC\ source
is not scalable or not available, \PDFTEX\ will use \PK\ fonts instead; the
\PDF\ output is still valid but some fonts may look ugly because of the
scaled bitmap.

To summarize this rather confusing story, we include a~some example lines.
First we use two fonts from the 14 standard fonts with font||specific
encoding, i.\,e.~no external encoding is given.  In the first line, the
fontfile is missing, so viewers will use their own font.  The ZapfDingbats
font is taken from the given font file.

\starttyping
psyr Symbol
pzdr ZapfDingbats <pzdr.pfb
\stoptyping

Similarly, two standard fonts with an external encoding. The \type{<}
preceding the encoding file name may be left out.

\starttyping
ptmr8r  Times-Roman  <8r.enc
ptmri8r Times-Italic <8r.enc <ptmri8a.pfb
\stoptyping

 A~SlantFont is specified similarly as for \type{dvips}. The \type
{SlantFont} or \type{ExtendFont} entries work only with embedded
font files.

\starttyping
psyro           ".167 SlantFont"         <usyr.pfb
pcrr8rn Courier ".85 ExtendFont" <8r.enc <pcrr8a.pfb
\stoptyping

Partially download a~font without re||encoding:

\starttyping
pgsr8a GillSans <pgsr8a.pfb
\stoptyping

Download a~font entirely without re||encoding:

\starttyping
pgsr8a GillSans <<pgsr8a.pfb
\stoptyping

Partially download a~font with re||encoding:

\starttyping
pgsr8r GillSans <8r.enc <pgsr8a.pfb
\stoptyping

Entirely download a~font with re||encoding:

\starttyping
pgsr8r GillSans <8r.enc <<pgsr8a.pfb
\stoptyping

Sometimes we do not want to include a~font, but need to extract parameters
from the font file and re||encode the font as well. This only works for
fonts with Adobe Standard Encoding. The font flags specify how such
a~font looks like, so \eg\ the Adobe Reader can generate a~similar instance
if the font resource is not available on the target system.

\starttyping
pgsr8r GillSans 32 <8r.enc pgsr8a.pfb
\stoptyping

Do not embed the font --- only extract the font parameters:

\starttyping
pgsr8a GillSans pgsr8a.pfb
\stoptyping

 A~TrueType font can be used in the same way as a~Type~1 font:

\starttyping
verdana8r Verdana <8r.enc <verdana.ttf
\stoptyping

\subsection{TrueType fonts}

As mentioned above, \PDFTEX\ can work with TrueType fonts. Defining
TrueType fonts is similar to Type~1. The only extra thing to do
with TrueType is to create a~\TFM\ file. There is a~program called
\type{ttf2afm} in the \PDFTEX\ distribution which can be used to
extract \AFM\ from TrueType fonts (another conversion program is
\type{ttf2pt1}). Usage of \type{ttf2afm} is simple:

\starttyping
ttf2afm -e <encoding vector> -o <afm outputfile> <ttf input file>
\stoptyping

 A~TrueType file can be recognized by its suffix \filename {ttf}. The optional
{\em encoding} specifies the encoding, which is the same as the encoding
vector used in map files for \PDFTEX\ and \type{dvips}. If the encoding is
not given, all the glyphs of the \AFM\ output will be mapped to \type
{/.notdef}. \type{ttf2afm} writes the output \AFM\ to standard output. If we
need to know which glyphs are available in the font, we can run \type
{ttf2afm} without encoding to get all glyph names. The resulting \AFM\ file
can be used to generate a~\TFM\ one by applying \filename {afm2tfm}.

To use a~new TrueType font the minimal steps may look like below. We suppose
that \filename {test.map} is used.

\starttyping
ttf2afm -e 8r.enc -o times.afm times.ttf
afm2tfm times.afm -T 8r.enc
echo "times TimesNewRomanPSMT <8r.enc <times.ttf" >>test.map
\stoptyping

There are a~few limitations with TrueType fonts in comparison with
Type~1 fonts:

\startitemize[a,packed]
\item  The special effects SlantFont|/|ExtendFont cannot be used.
\item  To subset a~TrueType font, the font must be specified as re||encoded,
       therefore an encoding vector must be given.
\item  TrueType fonts coming with embedded \PDF\ files are kept
       untouched; they are not replaced by local ones.
\stopitemize

%***********************************************************************

\section{Formal syntax specification}

This section formally specifies the \PDFTEX\ specific extensions to
the \TEX\ macro programming language. All primitives are prefixed by
\type {pdf} except for \type{\efcode}, \type{\lpcode}, \type{\rpcode},
\type{\leftmarginkern}, and \type{\rightmarginkern}. The general
definitions and syntax rules follow after the list of primitives.

%%S This is the list of new or extended primitives provided by pdftex.
%%S Don't edit this file, as it is now auto-generated from the
%%S pdfTeX documentation file pdftex-t.tex by script syntaxform.awk.
%%S Used convention for syntax rules is borrowed from `TeXbook naruby'
%%S by Petr Olsak.
%%S $Id: pdftex-t.tex,v 1.621 2005/08/25 15:41:41 oneiros Exp $
%%S NL
%%S integer registers:

\subsubject{Integer registers}

\startpacked

\Syntax {
\Tex {\pdfoutput} \Whatever{integer}
}

\Syntax {
\Tex {\pdfminorversion} \Whatever{integer}
}

%\Syntax {
%\Tex {\pdfoptionpdfminorversion} \Whatever{integer}
%}

\Syntax {
\Tex {\pdfcompresslevel} \Whatever{integer}
}

\Syntax {
\Tex {\pdfdecimaldigits} \Whatever{integer}
}

\Syntax {
\Tex {\pdfimageresolution} \Whatever{integer}
}

\Syntax {
\Tex {\pdfpkresolution} \Whatever{integer}
}

\Syntax {
\Tex {\pdftracingfonts} \Whatever{integer}
}

\Syntax {
\Tex {\pdfuniqueresname} \Whatever{integer}
}

\Syntax {
\Tex {\pdfadjustspacing} \Whatever{integer}
}

\Syntax {
\Tex {\pdfprotrudechars} \Whatever{integer}
}

\Syntax {
\Tex {\efcode} \Something{font} \Something{8-bit number} \Whatever{integer}
}

\Syntax {
\Tex {\lpcode} \Something{font} \Something{8-bit number} \Whatever{integer}
}

\Syntax {
\Tex {\rpcode} \Something{font} \Something{8-bit number} \Whatever{integer}
}

\Syntax {
\Tex {\pdfforcepagebox} \Whatever{integer}
}

\Syntax {
\Tex {\pdfoptionalwaysusepdfpagebox} \Whatever{integer}
}

\Syntax {
\Tex {\pdfinclusionerrorlevel} \Whatever{integer}
}

\Syntax {
\Tex {\pdfoptionpdfinclusionerrorlevel} \Whatever{integer}
}

\Syntax {
\Tex {\pdfimagehicolor} \Whatever{integer}
}

\Syntax {
\Tex {\pdfimageapplygamma} \Whatever{integer}
}

\Syntax {
\Tex {\pdfgamma} \Whatever{integer}
}

\Syntax {
\Tex {\pdfimagegamma} \Whatever{integer}
}

\stoppacked

%%S NL
%%S dimen registers:

\subsubject{Dimen registers}

\startpacked

\Syntax {
\Tex {\pdfhorigin} \Whatever{dimen}
}

\Syntax {
\Tex {\pdfvorigin} \Whatever{dimen}
}

\Syntax {
\Tex {\pdfpagewidth} \Whatever{dimen}
}

\Syntax {
\Tex {\pdfpageheight} \Whatever{dimen}
}

\Syntax {
\Tex {\pdflinkmargin} \Whatever{dimen}
}

\Syntax {
\Tex {\pdfdestmargin} \Whatever{dimen}
}

\Syntax {
\Tex {\pdfthreadmargin} \Whatever{dimen}
}

\stoppacked

%%S NL
%%S token registers:

\subsubject{Token registers}

\startpacked

\Syntax {
\Tex {\pdfpagesattr} \Whatever{tokens}
}

\Syntax {
\Tex {\pdfpageattr} \Whatever{tokens}
}

\Syntax {
\Tex {\pdfpageresources} \Whatever{tokens}
}

\Syntax {
\Tex {\pdfpkmode} \Whatever{tokens}
}

\stoppacked

%%S NL
%%S expandable commands:

\subsubject{Expandable commands}

\startpacked

\Syntax {
\Tex {\pdftexrevision} \Whatever{expandable}
}

\Syntax {
\Tex {\pdftexbanner} \Whatever{expandable}
}

\Syntax {
\Tex {\pdfcreationdate} \Whatever{expandable}
}

\Syntax {
\Tex {\pdfpageref} \Something{page number} \Whatever{expandable}
}

\Syntax {
\Tex {\pdfxformname} \Something{object number} \Whatever{expandable}
}

\Syntax {
\Tex {\pdffontname} \Something{font} \Whatever{expandable}
}

\Syntax {
\Tex {\pdffontobjnum} \Something{font} \Whatever{expandable}
}

\Syntax {
\Tex {\pdffontsize} \Something{font} \Whatever{expandable}
}

\Syntax {
\Tex {\pdfincludechars} \Something{font} \Something{general text} \Whatever{expandable}
}

\Syntax {
\Tex {\leftmarginkern} \Something{box number} \Whatever{expandable}
}

\Syntax {
\Tex {\rightmarginkern} \Something{box number} \Whatever{expandable}
}

\Syntax {
\Tex {\pdfescapestring} \Something{general text} \Whatever{expandable}
}

\Syntax {
\Tex {\pdfescapename} \Something{general text} \Whatever{expandable}
}

\Syntax {
\Tex {\pdfescapehex} \Something{general text} \Whatever{expandable}
}

\Syntax {
\Tex {\pdfunescapehex} \Something{general text} \Whatever{expandable}
}

% Currently it's only experimental, so don't really document it
\iffalse
\Syntax {
\Tex {\pdfstrcmp} \Something{general text} \Something{general text} \Whatever{expandable}
}
\fi

\Syntax {
\Tex {\pdfuniformdeviate} \Something{number} \Whatever{expandable}
}

\Syntax {
\Tex {\pdfnormaldeviate} \Whatever{expandable}
}

\Syntax {
\Tex {\pdfmdfivesum} \Optional{file} \Something{general text} \Whatever{expandable}
}

\Syntax {
\Tex {\pdffilemoddate} \Something{general text} \Whatever{expandable}
}

\Syntax {
\Tex {\pdffilesize} \Something{general text} \Whatever{expandable}
}

\Syntax {
\Tex {\pdffiledump} \Optional{offset \Something{number}} \Optional{length \Something{number}} \Something{general text}
\Whatever{expandable}
}

\stoppacked

%%S NL
%%S read-only integers:

\subsubject{Read||only integers}

\startpacked

\Syntax {
\Tex {\pdftexversion} \Whatever{read||only integer}
}

\Syntax {
\Tex {\pdflastobj} \Whatever{read||only integer}
}

\Syntax {
\Tex {\pdflastxform} \Whatever{read||only integer}
}

\Syntax {
\Tex {\pdflastximage} \Whatever{read||only integer}
}

\Syntax {
\Tex {\pdflastximagepages} \Whatever{read||only integer}
}

\Syntax {
\Tex {\pdflastannot} \Whatever{read||only integer}
}

\Syntax {
\Tex {\pdflastxpos} \Whatever{read||only integer}
}

\Syntax {
\Tex {\pdflastypos} \Whatever{read||only integer}
}

\Syntax {
\Tex {\pdflastdemerits} \Whatever{read||only integer}
}

\Syntax {
\Tex {\pdfelapsedtime} \Whatever{read||only integer}
}

\Syntax {
\Tex {\pdfrandomseed} \Whatever{read||only integer}
}

\Syntax {
\Tex {\pdfshellescape} \Whatever{read||only integer}
}

% No longer available
%\Syntax {
%\Tex {\pdflastvbreakpenalty} \Whatever{read||only integer}
%}

\stoppacked

%%S NL
%%S general commands:

\subsubject{General commands}

\startpacked

\Syntax {
\Tex {\pdfobj} \Something{object type spec} \Whatever{h, v, m}
}

\Syntax {
\Tex {\pdfrefobj} \Something{object number} \Whatever{h, v, m}
}

\Syntax {
\Tex {\pdfxform} \Optional{\Something{xform attr spec}} \Something{box number} \Whatever{h, v, m}
}

\Syntax {
\Tex {\pdfrefxform} \Something{object number} \Whatever{h, v, m}
}

\Syntax {
\Tex {\pdfximage} \Optional{\Something{image attr spec}} %
  \Something{general text} \Whatever{h, v, m}
}

\Syntax {
\Tex {\pdfrefximage} \Something{object number} \Whatever{h, v, m}
}

\Syntax {
\Tex {\pdfannot} \Something{annot type spec} \Whatever{h, v, m}
}

\Syntax {
\Tex {\pdfstartlink} %
  \Optional{\Something{rule spec}} %
  \Optional{\Something{attr spec}} \Something{action spec} \Whatever{h, m}
}

\Syntax {
\Tex {\pdfendlink} \Whatever{h, m}
}

\Syntax {
\Tex {\pdfoutline} \Something{outline spec} \Whatever{h, v, m}
}

\Syntax {
\Tex {\pdfdest} \Something{dest spec} \Whatever{h, v, m}
}

\Syntax {
\Tex {\pdfthread} \Something{thread spec} \Whatever{h, v, m}
}

\Syntax {
\Tex {\pdfstartthread} \Something{thread spec} \Whatever{v, m}
}

\Syntax {
\Tex {\pdfendthread} \Whatever{v, m}
}

\Syntax {
\Tex {\pdfsavepos} \Whatever{h, v, m}
}

\Syntax {
\Tex {\pdfinfo} \Something{general text}
}

\Syntax {
\Tex {\pdfcatalog} \Something{general text} %
  \Optional{\Something{open-action spec}}
}

\Syntax {
\Tex {\pdfnames} \Something{general text}
}

\Syntax {
\Tex {\pdfmapfile} \Something{map spec}
}

\Syntax {
\Tex {\pdfmapline} \Something{map spec}
}

\Syntax {
\Tex {\pdffontattr} \Something{font} \Something{general text}
}

\Syntax {
\Tex {\pdftrailer} \Something{general text}
}

\Syntax {
\Tex {\pdffontexpand} \Something{font} \Something{expand spec}
}

\Syntax {
\Tex {\vadjust} \Optional{\Something{pre spec}} \Something{filler} \Lbrace \Something{vertical mode material} \Rbrace \Whatever{h, m}
}

\Syntax {
\Tex {\pdfliteral} \Optional{\Literal {direct}} \Something{general text} \Whatever{h, v, m}
}

\Syntax {
\Tex {\special} \Something{pdfspecial spec}
}

\Syntax {
\Tex {\pdfresettimer}
}

\Syntax {
\Tex {\pdfsetrandomseed} \Something{number}
}

\Syntax {
\Tex {\pdfnoligatures} \Something{font}
}

\stoppacked

\subsubject{General definitions and syntax rules}

\startpacked

%%S NL
%%S syntax rules:

\Syntax {
\Something{general text} \Means %
  \Lbrace \Something{balanced text} \Rbrace
}

\Syntax {
\Something{attr spec} \Means %
  \Literal {attr} \Something{general text}
}

\Syntax {
\Something{resources spec} \Means %
  \Literal {resources} \Something{general text}
}

\Syntax {
\Something{rule spec} \Means %
  (\Literal {width} \Or \Literal {height} \Or \Literal {depth}) %
  \Something{dimension} \Optional{\Something{rule spec}}
}

%\Syntax {\Something{object type spec} \Means
%  \Optional{\Literal {stream}
%  \Optional{\Something{attr spec}}} \Something{object contents}}

\Syntax {
\Something{object type spec} \Means %
  \Literal{reserveobjnum}
  \Or \Next %
  \Optional{\Literal{useobjnum} \Something{number}} \Next %
  \Optional{\Literal{stream} \Optional{\Something{attr spec}}} %
  \Something{object contents}
}

\Syntax {
\Something{annot type spec} \Means %
  \Literal{reserveobjnum}
  \Or \Next %
  \Optional{\Literal{useobjnum} \Something{number}} %
  \Optional{\Something{rule spec}} %
  \Something{general text}
}

\Syntax {
\Something{object contents} \Means %
  \Something{file spec}
  \Or \Something{general text}
}

\Syntax {
\Something{xform attr spec} \Means %
  \Optional{\Something{attr spec}} \Optional{\Something{resources spec}}
}

\Syntax {
\Something{image attr spec} \Means %
  \Optional{\Something{rule spec}} %
  \Optional{\Something{attr spec}} %
  \Optional{\Something{page spec}} %
  \Optional{\Something{colorspace spec}} %
  \Optional{\Something{pdf box spec}}
}

\Syntax {
\Something{outline spec} \Means %
  \Optional{\Something{attr spec}} %
  \Something{action spec} %
  \Optional{\Literal{count} \Something{number}} %
  \Something{general text}
}

\Syntax {
\Something{action spec} \Means %
  \Literal {user} \Something{user-action spec}
  \Or \Literal {goto} \Something{goto-action spec}
  \Or \Next \Literal {thread} \Something{thread-action spec}
}

\Syntax {
\Something{user-action spec} \Means %
  \Something{general text}
}

%HE Check:
\Syntax {
\Something{goto-action spec} \Means %
  \Something{numid}
  \Or \Next \Optional{\Something{file spec}} \Something{nameid}
  \Or \Next \Optional{\Something{file spec}} \Optional{\Something{page spec}} \Something{general text}
  \Or \Next \Something{file spec} \Something{nameid} \Something{newwindow spec}
  \Or \Next \Something{file spec} \Optional{\Something{page spec}} \Something{general text} \Something{newwindow spec}
}

\Syntax {
\Something{thread-action spec} \Means %
  \Optional{\Something{file spec}} \Something{numid}
  \Or \Optional{\Something{file spec}} \Something{nameid}
}

\Syntax {
\Something{open-action spec} \Means %
  \Literal {openaction} \Something{action spec}
}

\Syntax {
\Something{colorspace spec} \Means %
  \Literal {colorspace} \Something{number}
}

\Syntax{
\Something{pdf box spec} \Means %
  \Literal{mediabox} %
  \Or \Literal{cropbox} %
  \Or \Literal{bleedbox} %
  \Or \Literal{trimbox} %
  \Or \Literal{artbox}
}

\Syntax{
\Something{map spec} \Means %
  \Lbrace \Optional{\Something{map modifier}} %
  \Something{balanced text} \Rbrace
}

\Syntax{
\Something{map modifier} \Means %
  \Literal{+} \Or \Literal{=} \Or \Literal{-}
}

\Syntax {
\Something{numid} \Means %
  \Literal {num} \Something{number}
}

\Syntax {
\Something{nameid} \Means %
  \Literal {name} \Something{general text}
}

\Syntax {
\Something{newwindow spec} \Means %
  \Literal {newwindow} \Or \Literal {nonewwindow}
}

\Syntax {
\Something{dest spec} \Means %
  \Something{numid} \Something{dest type}
  \Or \Something{nameid} \Something{dest type}
}

\Syntax {
\Something{dest type} \Means %
  \Literal {xyz} \Optional{\Literal {zoom} \Something{number}}
  \Or \Literal {fitr} \Something{rule spec}
  \Or \Next \Literal {fitbh}
  \Or \Literal {fitbv}
  \Or \Literal {fitb}
  \Or \Literal {fith}
  \Or \Literal {fitv}
  \Or \Literal {fit}
}

\Syntax {
\Something{thread spec} \Means %
  \Optional{\Something{rule spec}} %
  \Optional{\Something{attr spec}} %
  \Something{id spec}
}

\Syntax {
\Something{id spec} \Means %
  \Something{numid} \Or \Something{nameid}
}

\Syntax {
\Something{file spec} \Means %
  \Literal {file} \Something{general text}
}

\Syntax {
\Something{page spec} \Means %
  \Literal {page} \Something{number}
}

\Syntax {
\Something{expand spec} \Means %
  \Something{stretch} %
  \Something{shrink} %
  \Something{step} %
  \Optional{\Literal{autoexpand}}
}

\Syntax {
\Something{stretch} \Means %
  \Something{number}
}

\Syntax {
\Something{shrink} \Means %
  \Something{number}
}

\Syntax {
\Something{step} \Means %
  \Something{number}
}

\Syntax {
\Something{pre spec} \Means %
  \Literal{pre}
}

\Syntax {
\Something{pdfspecial spec} \Means %
  \Lbrace \Optional{\Something{pdfspecial id} \Optional{\Something{pdfspecial modifier}}} %
  \Something{balanced text} \Rbrace
}

\Syntax {
\Something{pdfspecial id} \Means %
  \Literal{pdf:} \Or \Literal{PDF:}
}

\Syntax {
\Something{pdfspecial modifier} \Means %
  \Literal{direct:}
}

\stoppacked

 A~\Something {general text} is expanded immediately, like \type{\special}
in traditional \TEX, unless explicitly mentioned otherwise.

Some of the object and image related primitives can be prefixed by
\type {\immediate}. More about that in the next sections.

%***********************************************************************

\section[primitives]{\PDFTEX\ primitives}

Here follows a~short description of the primitives added by \PDFTEX\
to the original \TEX\ engine (other extensions by \MLTEX\ and \ENCTEX\
are not listed).  One way to learn more about how to use these new
primitives is to have a~look at the file \filename {samplepdf.tex} in the
\PDFTEX\ distribution.

Note that if the output is \DVI\ then the \PDFTEX\ specific dimension
parameters are not used at all. However some \PDFTEX\ integer parameters
can affect the \DVI\ as well as \PDF\ output (currently \type{\pdfoutput}
and \type{\pdfadjustspacing}).

%A warning to macro writers: The \PDFTEX-team reserves the namespaces
%\type{\pdf*} and \type{\ptex*} for use by \PDFTEX; if you define macros
%whose names start with \type{\pdf} or \type{\ptex}, you risk nameclashes
%with new primitives introduced in future versions of \PDFTEX.

General warning: many of these new primitives, for example
\type{\pdfdest} and \type{\pdfoutline}, write their arguments directly
to the \PDF\ output file (when producing \PDF), as \PDF\ string
constants.  This means that {\em you} (or, more likely, the macros you
write) must escape characters as necessary (namely \type{\}, \type{(},
and \type{)}.  Otherwise, an invalid \PDF\ file may result.  The
\type{hyperref} and \TEXINFO\ packages have code which may serve as
a~starting point for implementing
this, although it will certainly need to be adapted to any particular
situation.

%***********************************************************************

\subsection{Document setup}

\pdftexprimitive {\Syntax {\Tex {\pdfoutput} \Whatever{integer}}}

\bookmark{\type{\pdfoutput}}

This parameter specifies whether the output format should be \DVI\ or
\PDF. A~positive value means \PDF\ output, otherwise (default 0) one gets
\DVI\ output.  This primitive is the only one that must be set to produce
\PDF\ output (unless the commandline option \type{-output-format=pdf}
is used); all other primitives are optional.  This parameter cannot be
specified {\em after} shipping out the first page.  In other words, if
we want \PDF\ output, we have to set \type{\pdfoutput} before \PDFTEX\
ships out the first page.

When \PDFTEX\ starts complaining about specials, one can be rather sure
that a~macro package is not aware of the \PDF\ mode. A~simple way of
making macros aware of \PDFTEX\ in \PDF\ or \DVI\ mode is:

\starttyping
\ifx\pdfoutput\undefined \csname newcount\endcsname\pdfoutput \fi
\ifcase\pdfoutput DVI CODE \else PDF CODE \fi
\stoptyping

Using the \type{ifpdf.sty} file, which works with both \LATEX\ and plain
\TeX, is a~cleaner way of doing this.  Historically, the simple test
\type{\ifx\pdfoutput\undefined} was defined; but nowadays, the \PDFTEX\
engine is used in distributions also for non-\PDF\ formats (\eg\
\LATEX), so \type{\pdfoutput} may be defined even when the output format
is \DVI.

\pdftexprimitive {\Syntax {\Tex {\pdfminorversion} \Whatever{integer}}}

\bookmark{\type{\pdfminorversion}}

This primitive sets the \PDF\ version of the generated file and the latest
allowed \PDF\ version of included \PDF{}s. \Eg, \type{\pdfminorversion=3} tells
\PDFTEX\ to set the \PDF\ version to 1.3 and allows only included \PDF\ files
with versions numbers up to 1.3. The default for \type{\pdfminorversion} is 4,
producing files with \PDF\ version 1.4. If specified, this primitive must
appear before any data is to be written to the generated \PDF\ file, so you
should put it at the very start of your files. The command has been introduced
in \PDFTEX\ 1.30.0 as a~shortened synonym of \type{\pdfoptionpdfminorversion}
command, that is obsolete by now.

\pdftexprimitive {\Syntax {\Tex {\pdfcompresslevel} \Whatever{integer}}}

\bookmark{\type{\pdfcompresslevel}}

This integer parameter specifies the level of stream compression (text,
in||line graphics, and embedded \PNG\ images (only if they are un- and
re||compressed during the embedding process); all done by the \type{zlib}
library). Zero means no compression, 1~means fastest, 9~means best, 2..8
means something in between. A~value outside this range will be adjusted to
the nearest meaningful value. This parameter is read each time \PDFTEX\
starts a~stream. Setting \type{\pdfcompresslevel=0} is great for \PDF\
stream debugging.

\pdftexprimitive {\Syntax {\Tex {\pdfdecimaldigits} \Whatever{integer}}}

\bookmark{\type{\pdfdecimaldigits}}

This integer parameter specifies the numeric accuracy of real coordinates as written
to the \PDF\ file. It gives the maximal number of decimal digits after the
decimal point. Valid values are in range 0..4. A~higher value means
more precise output, but also results in a~larger file size and more
time to display or print. In most cases the optimal value is~2. This
parameter does not influence the precision of numbers used in raw \PDF\ code,
like that used in \type{\pdfliteral} and annotation action specifications;
also multiplication items (\eg\ scaling factors)
are not affected and are always output with best precision.
This parameter is read when \PDFTEX\ writes a~real number to the \PDF\ output.

When including huge \METAPOST\ images using \filename {supp-pdf.tex}, one can
limit the accuracy to two digits by typing: \type{\twodigitMPoutput}.

\pdftexprimitive {\Syntax {\Tex {\pdfhorigin} \Whatever{dimension}}}

\bookmark{\type{\pdfhorigin}}

This parameter can be used to set the horizontal offset the output box
from the top left corner of the page. A~value of 1~inch corresponds to the
normal \TEX\ offset. This parameter is read when \PDFTEX\ starts shipping
out a~page to the \PDF\ output.

For standard purposes, this parameter should always be kept at
1~true inch.  If you want to shift text on the page, use \TEX's
own \type{\hoffset} primitive.  To avoid surprises, after global
magnification has been changed by the \type{\mag} primitive, the
\type{\pdfhorigin} parameter should still be 1~true inch, \eg\
by typing \type{\pdfhorigin=1 true in} after issuing the \type{\mag}
command.  Or, you can preadjust the \type{\pdfhorigin} value before typing
\type{\mag}, so that its value after the \type{\mag} command ends up at
1~true inch again.

\pdftexprimitive {\Syntax {\Tex {\pdfvorigin} \Whatever{dimension}}}

\bookmark{\type{\pdfvorigin}}

This parameter is the vertical companion of \type{\pdfhorigin}, and the
notes above regarding \type{\mag} and true dimensions apply. Also keep
in mind that the \TEX\ coordinate system starts in the top left corner
(downward), while \PDF\ coordinates start at the bottom left corner
(upward).

\pdftexprimitive {\Syntax {\Tex {\pdfpagewidth} \Whatever{dimension}}}

\bookmark{\type{\pdfpagewidth}}

This dimension parameter specifies the page width of the \PDF\ output
(the screen, the paper, etc.). \PDFTEX\ reads this parameter when it
starts shipping out a~page.  After magnification has been changed by
the \type{\mag} primitive, check that this parameter reflects the wished
true page width.

If the value is set to zero, the page width is calculated as
$w_{\hbox{\txx box being shipped out}} + 2 \times (\hbox{horigin} +
\hbox{\type{\hoffset}})$.  When part of the page falls off the paper or
screen, you can be rather sure that this parameter is set wrong.

\pdftexprimitive {\Syntax {\Tex {\pdfpageheight} \Whatever{dimension}}}

\bookmark{\type{\pdfpageheight}}

Similar to the previous item, this dimension parameter specifies the
page height of the \PDF\ output. If set to zero, the page height will
be calculated analogously to the above. After magnification has been
changed by the \type{\mag} primitive, check that this parameter reflects
the wished true page height.

\subsection{The document info and catalog}

\pdftexprimitive {\Syntax {\Tex {\pdfinfo} \Something{general text}}}

\bookmark{\type{\pdfinfo}}

This primitive allows the user to add information to the document info
section; if this information is provided, it can be extracted, \eg\ by the
pdfinfo program, or by the Adobe Reader (version 7.0: menu option {\em File
$\longrightarrow$ Document Properties}). The \Something{general text} is
a~collection of key||value||pairs. The key names are preceded by a~\type{/},
and the values, being strings, are given between parentheses. All keys
are optional. Possible keys are \type{/Author}, \type{/CreationDate}
(defaults to current date including time zone info), \type{/ModDate},
\type{/Creator} (defaults to \type {TeX}), \type{/Producer} (defaults
to \esctype{pdfTeX-@currentpdftex} or \esctype{pdfeTeX-@currentpdftex}),
\type{/Title}, \type{/Subject}, and \type{/Keywords}.

\type{/CreationDate} and \type{/ModDate} are expressed in the form
\type{D:YYYYMMDDhhmmssTZ..}, where \type{YYYY} is the year, \type{MM} is
the month, \type{DD} is the day, hh is the hour, \type{mm} is the minutes,
\type{ss} is the seconds, and \type{TZ..} is an optional string denoting
the time zone. An example of this format is shown below.  For details
please refer to the \PDF\ Reference.

Multiple appearances of \type{\pdfinfo} will be concatenated. In general, if
a~key is given more than once, one may expect that the first appearance will be used.
Be aware however, that this behaviour is viewer dependent. Except expansion,
\PDFTEX\ does not perform any further operations on \Something{general text} provided
by the user.

An example of the use of \type{\pdfinfo} is:

\startesctyping
\pdfinfo
  { /Title        (example.pdf)
    /Creator      (TeX)
    /Producer     (pdfeTeX @currentpdftex)
    /Author       (Tom and Jerry)
    /CreationDate (D:20050428154343+01'00')
    /ModDate      (D:20050428155343+01'00')
    /Subject      (Example)
    /Keywords     (mouse, cat) }
\stopesctyping

\pdftexprimitive {\Syntax {\Tex {\pdfcatalog} \Something{general text}
\Optional{\Something{open-action spec}}}}

\bookmark{\type{\pdfcatalog}}

Similar to the document info section is the document catalog, where keys
are \type{/URI}, which provides the base \URL\ of the document, and \type
{/PageMode}, which determines how the \PDF\ viewer displays the document
on startup. The possibilities for the latter are explained in \in {Table}
[pagemode]:

\startbuffer
\starttabulate[|l|l|]
\HL
\NC \bf value        \NC \bf meaning                            \NC\NR
\HL
\NC \tt /UseNone     \NC neither outline nor thumbnails visible \NC\NR
\NC \tt /UseOutlines \NC outline visible                        \NC\NR
\NC \tt /UseThumbs   \NC thumbnails visible                     \NC\NR
\NC \tt /FullScreen  \NC full||screen mode                      \NC\NR
\HL
\stoptabulate
\stopbuffer

\placetable
  [here][pagemode]
  {Supported \type{/PageMode} values.}
  {\getbuffer}

In full||screen mode, there is no menu bar, window controls, nor any other
window present. The default setting is \type{/UseNone}.

The \Something{openaction} is the action provided when opening the
document and is specified in the same way as internal links, see \in
{section} [linking]. Instead of using this method, one can also write the
open action directly into the catalog.

\pdftexprimitive {\Syntax {\Tex {\pdfnames} \Something{general text}}}

\bookmark{\type{\pdfnames}}

This primitive inserts the \Something{general text} to the \type
{/Names} array. The text must
conform to the specifications as laid down in the \PDF\ Reference Manual,
otherwise the document can be invalid.

\pdftexprimitive {\Syntax {\Tex {\pdftrailer} \Something{general text}}}

\bookmark{\type{\pdftrailer}}

This command puts its argument text verbatim into the file trailer
dictionary. \introduced{1.11a}

%***********************************************************************

\subsection{Fonts}

\pdftexprimitive {\Syntax {\Tex {\pdfpkresolution} \Whatever{integer}}}

\bookmark{\type{\pdfpkresolution}}

This integer parameter specifies the default resolution of embedded \PK\
fonts and is read when \PDFTEX\ downloads a~\PK\ font during finishing
the \PDF\ output. As bitmap fonts are still rendered poorly by some \PDF\
viewers, it is best to use Type~1 fonts when available.

\pdftexprimitive {\Syntax {\Tex {\pdffontexpand}
\Something{font} \Something{stretch} \Something{shrink}
\Something{step} \Optional{\Literal{autoexpand}}}}

\bookmark{\type{\pdffontexpand}}

This extension to \TEX's font definitions controls a~\PDFTEX\ automatism
called {\em font expansion}. We describe this by an example:

\starttyping
\font\somefont=sometfm at 10pt
\pdffontexpand\somefont 30 20 10 autoexpand
\pdfadjustspacing=2
\stoptyping

The \type{30 20 10} means this: \quotation {hey \TEX, when line
breaking is going badly, you may stretch the glyphs in this font as
much as 3\,\% or shrink them by 2\,\%}. Because \PDFTEX\ uses internal
data structures with fixed widths, each additional width also means an
additional font. For practical reasons \PDFTEX\ uses discrete steps,
in this example, 1\,\%. This means that for font \type{sometfm} up to~6
differently scaled alternatives may be used. When no step is specified,
0.5\,\% steps are used.

Roughly spoken, the trick is as follows. Consider a~text typeset in
triple column mode. When \TEX\ cannot break a~line in the appropriate
way, the unbreakable parts of the word will stick into the margin. When
\PDFTEX\ notes this, it will try to scale (shrink) the glyphs in that
line using fixed steps, until the line fits. When lines are too spacy,
the opposite happens: \PDFTEX\ starts scaling (stretching) the glyphs
until the white space gaps is acceptable. This glyph stretching and
shrinking is called {\em font expansion}.

The additional expanded fonts get artificial names by adding the font
expansion value to the tfmname of the base font, \eg\ \type{sometfm+10}
for 1\,\% stretch or \type {sometfm-15} for 1.5\,\% shrink. If the
\type{autoexpand} option is not given, \TFM\ files with these names and
appropriate dimensions must be available. So, each expanded variant
of a~font must have its own \TFM\ file! Expanded \TFM\ names like
\type{sometfm+10} must not be mentioned in the map file (but the \TFM\
name of the base font without expansion must be there). When no \TFM\
file can be found, \PDFTEX\ will try to generate it by executing the
script \type{mktextfm}, where available and supported.

The font expansion is greatly simplified, if the \type{autoexpand}
option is there.  Then no expanded \TFM\ file versions are needed;
instead, \PDFTEX\ generates expanded copies of the unexpanded \TFM\
data structures and keeps them in its memory.

\PDFTEX\ requires only unexpanded Type~1 font files for font expansion,
from which all expanded font versions are internally generated and
included (subsetted) into the \PDF\ output file.  To enable font
expansion, don't forget to set \type{\pdfadjustspacing} to a~value
greater than zero.

The font expansion mechanism is inspired by an optimization first
introduced by Prof.~Hermann Zapf, which in itself goes back to
optimizations used in the early days of typesetting: use different
glyphs to optimize the grayness of a~page. So, there are many, slightly
different~a's, e's, etc. For practical reasons \PDFTEX\ does not use
such huge glyph collections; it uses horizontal scaling instead. This is
sub||optimal, and for many fonts, possibly offensive to the design. But,
when using \PDF, it's not illogical: \PDF\ viewers use so||called
Multiple Master fonts when no fonts are embedded and|/|or can be found
on the target system. Such fonts are designed to adapt their design to
the different scaling parameters. It is up to the user to determine
to what extent mixing slightly remastered fonts can be used without
violating the design. Think of an~O: when geometrically stretched, the
vertical part of the glyph becomes thicker, and looks incompatible with
an unscaled original.  With a~multiple master situation, one can stretch
while keeping this thickness compatible.

\pdftexprimitive {\Syntax {\Tex {\pdfadjustspacing} \Whatever{integer}}}

\bookmark{\type{\pdfadjustspacing}}

This primitive provides a~switch for enabling font expansion. By default,
\type{\pdfadjustspacing} is set to~0; then font expansion is disabled,
so that the \PDFTEX\ output is identical to that from the original \TEX\
engine.

Font expansion can be activated in two modes. When
\type{\pdfadjustspacing} is set to~1, font expansion is applied {\em
after} \TEX's normal paragraph breaking routines have broken the paragraph
into lines. In this case, line breaks are identical to standard \TEX\
behaviour.

When set to~2, the width changes that are the result of stretching and
shrinking are taken into account {\em while} the paragraph is broken into
lines. In this case, line breaks are likely to be different from those of
standard \TEX. In fact, paragraphs may even become longer or shorter.

Both alternatives require a~collection of \TFM\ files that are
related to the \Something{stretch} and \Something{shrink} settings
for the \type{\pdffontexpand} primitive, unless this is given with the
\type{autoexpand} option.

\pdftexprimitive {\Syntax {\Tex {\efcode} \Something{font} \Whatever{integer}}}

\bookmark{\type{\efcode}}

We didn't yet tell the whole story. One can imagine that some glyphs
are visually more sensitive to stretching or shrinking than others. Then
the \type{\efcode} primitive can be used to influence the expandability
of individual glyphs within a~given font, as a~factor to the expansion
setting from the \type{\pdffontexpand} primitive. The syntax is similar
to \type{\sfcode} (but with the \Something{font} required), and it
defaults to~1000, meaning 100\,\% expandability. The given integer value
is clipped to the range 0..1000, corresponding to a~usable expandability
range of 0..100\,\%. Example:

\starttyping
\efcode\somefont`A=800
\efcode\somefont`O=0
\stoptyping

Here an~A may stretch or shrink only by 80\,\% of the current expansion
value for that font, and expansion for the~O is disabled. The actual
expansion is still bound to the steps as defined by \type{\pdffontexpand}
primitive, otherwise one would end up with more possible font inclusions
than would be comfortable.

\pdftexprimitive {\Syntax {\Tex {\pdfprotrudechars} \Whatever{integer}}}

\bookmark{\type{\pdfprotrudechars}}

Yet another way of optimizing paragraph breaking is to let certain
characters move into the margin (`character protrusion'). When
\type{\pdfprotrudechars=1}, the glyphs qualified as such will make
this move when applicable, without changing the line-breaking. When
\type{\pdfprotrudechars=2} (or greater), character protrusion will
be taken into account while considering breakpoints, so line-breaking
might be changed. This qualification and the amount of shift are set by
the primitives \type{\rpcode} and \type{\lpcode}. Character protrusion
is disabled when \type{\pdfprotrudechars=0} (or negative).

If you want to protrude some item other than a~character (\eg\
 a~\type{\hbox}), you can do so by padding the item with an invisible
zero||width character, for which protrusion is activated.

\pdftexprimitive {\Syntax {\Tex {\rpcode} \Something{font} \Whatever{integer}}}

\bookmark{\type{\rpcode}}

The amount that a~character from a~given font may shift into the right
margin (`character protrusion') is set by the primitive \type {\rpcode}.
The protrusion distance is the integer value given to \type {\rpcode},
multiplied with 0.001\,em from the current font. The given integer
value is clipped to the range $-$1000..1000, corresponding to a~usable
protrusion range of $-$1\,em..1\,em. Example:

\starttyping
\rpcode\somefont`,=200
\rpcode\somefont`-=150
\stoptyping

Here the comma may shift by 0.2\,em into the margin and the hyphen by
0.15\,em. All these small bits and pieces will help \PDFTEX\ to give
you better paragraphs (use \type{\rpcode} judiciously; don't overdo it).

Remark: old versions of \PDFTEX\ use the character width as measure. This
was changed to a~proportion of the em-width after \THANH\ finished his
master's thesis.

\pdftexprimitive {\Syntax {\Tex {\lpcode} \Something{font}
\Whatever{integer}}}

\bookmark{\type{\lpcode}}

This is similar to \type{\rpcode}, but affects the amount by which
characters may protrude into the left margin. Also here the given integer
value is clipped to the range $-$1000..1000.

\pdftexprimitive {\Syntax {\Tex {\leftmarginkern} \Something{box number} \Whatever{expandable}}}

\bookmark{\type{\leftmarginkern}}

The \Tex{\leftmarginkern} \Something{box number} primitive expands to the
width of the margin kern at the left side of the horizontal list stored
in \Tex{\box} \Something{box number}. The expansion string includes the
unit \type{pt}. \Eg, when the left margin kern of \type{\box0} amounts
to $-$10\,pt, \type{\leftmarginkern0} will expand to \type{-10pt}.
A~similar primitive \type{\rightmarginkern} exists for the right margin.
\introduced{1.30.0}

These are auxiliary primitives to make character protrusion
more versatile. When using the \TEX\ primitive \type{\unhbox} or
\type{\unhcopy}, the margin kerns at either end of the unpackaged
hbox will be removed (\eg\ to avoid weird effects if several
hboxes are unpackaged behind each other into the same horizontal
list). These \type{\unhbox} or \type{\unhcopy} are often used together
with \type{\vsplit} for dis- and re||assembling of paragraphs, \eg\ to
add line numbers. Paragraphs treated like this do not show character
protrusion by default, as the margin kerns have been removed during the
unpackaging process.

The \type{\leftmarginkern} and \type{\rightmarginkern} primitives allow
to access the margin kerns and store them away before unpackaging the
hbox. \Eg\, the following code snipplet restores margin kerning of
a~horizontal list stored in \type{\box\testline}, resulting in a~hbox with
proper margin kerning (which is then done by ordinary kerns).

\starttyping
\dimen0=\leftmarginkern\testline
\dimen1=\rightmarginkern\testline
\hbox to\hsize{\kern\dimen0\unhcopy\testline\kern\dimen1}
\stoptyping

\pdftexprimitive {\Syntax {\Tex {\rightmarginkern} \Something{box number} \Whatever{expandable}}}

\bookmark{\type{\rightmarginkern}}

The \Tex{\rightmarginkern} \Something{box number} primitive expands to
the width of the margin kern at the right side of the horizontal list
stored in \Tex{\box} \Something{box number}. See \type{\leftmarginkern}
for more details.
\introduced{1.30.0}

\pdftexprimitive {\Syntax {\Tex {\pdffontattr} \Something{font} \Something{general text}}}

\bookmark{\type{\pdffontattr}}

This primitive inserts the \Something{general text} to the \type {/Font}
dictionary. The text must conform to the specifications as laid down in the
\PDF\ Reference Manual, otherwise the document can be invalid.

\pdftexprimitive {\Syntax {\Tex {\pdffontname} \Something{font} \Whatever
{expandable}}}

\bookmark{\type{\pdffontname}}

In \PDF\ files produced by \PDFTEX\ one can recognize a~font resource
by the prefix~\type{/F} followed by a~number, for instance \type{/F12}
or \type{/F54}.  For a~given \TEX\ \Something{font}, this primitive
expands to the number from the corresponding font resource name.
\Eg, if \type{/F12} corresponds to some \TEX\ font \type{\foo}, the
\type{\pdffontname\foo} expands to the number~\type{12}.

In the current implementation, when \type{\pdfuniqueresname} (see below)
is set to a~positive value, the \type{\pdffontname} still returns only the
number from the font resource name, but not the appended random string.

\pdftexprimitive {\Syntax {\Tex {\pdffontobjnum} \Something{font} \Whatever
{expandable}}}

\bookmark{\type{\pdffontobjnum}}

This command is similar to \type{\pdffontname}, but it returns the
\PDF\ object number of the font dictionary instead of the number from
the font resource name.  \Eg, if the font dictionary (\type{/Type
/Font}) in \PDF\ object~3 corresponds to some \TEX\ font \type{\foo},
the \type{\pdffontobjnum\foo} gives the number~\type{3}.

Use of \type{\pdffontname} and \type {\pdffontobjnum} allows users full
access to all the font resources used in the document.

\pdftexprimitive {\Syntax {\Tex {\pdffontsize} \Something{font} \Whatever
{expandable}}}

\bookmark{\type{\pdffontsize}}

This primitive expands to the font size of the given font, with unit
\type{pt}.  \Eg, when using the plain \TeX\ macro package, the call
\type{\pdffontsize\tenrm} expands to \type{10.0pt}.

\pdftexprimitive {\Syntax {\Tex {\pdfincludechars} \Something{font}
\Something{general text}}}

\bookmark{\type{\pdfincludechars}}

This command causes \PDFTEX\ to treat the characters in \Something{general
text} as if they were used with \Something{font}\unkern, which means that the
corresponding glyphs will be embedded into the font resources in the \PDF\
output. Nothing is appended to the list being built.

\pdftexprimitive {\Syntax {\Tex {\pdfuniqueresname} \Whatever{integer}}}

\bookmark{\type{\pdfuniqueresname}}

When this primitive is assigned a~positive number, \PDF\ resource names
will be made reasonably unique by appending a~random string consisting
of six \ASCII\ characters.

\pdftexprimitive {\Syntax {\Tex {\pdfmapfile} \Something{map spec}}}

\bookmark{\type{\pdfmapfile}}

This primitive is used for managing font map files, to make them known
to \PDFTEX\ so that they can be read in.  If no \type{\pdfmapfile}
primitive is given, the default map file \filename{pdftex.map} will be
read in by \PDFTEX.

Normally there is no need for the \PDFTEX\ user to bother about the
\type{\pdfmapfile} primitive, as the main \TEX\ distributions provide
nice helper tools that automatically assemble the default font map file.
One prominent tool example is the script \type{updmap} coming with \TETEX.

The operation mode of the \type{\pdfmapfile} primitive is selected by
a~flag letter (\type{+}, \type{=}, \type{-}, or omitted). This flag defines
how a~map file is going to be handled, and how a~collision between an
existing map entry and a~newer one is resolved; either ignoring a~later
entry, or replacing or deleting an existing entry.  But in any case,
map entries of fonts already in use are kept untouched.  The companion
primitive \type{\pdfmapline} allows something similar, only that a~single
map line for one font (instead of a~map file name) is given as argument.
Here are two examples:

\starttyping
\pdfmapfile{+myfont.map}
\pdfmapline{+ptmri8r Times-Italic <8r.enc <ptmri8a.pfb}
\stoptyping

The general map handling function is that map items, which are either map
file names or single font map lines (in case of the \type{\pdfmapline}
primitive) are put into an auxiliary list of pending map items. During
the next page shipout, this list is processed and all pending map items
are sequentially scanned for their map entries, and an internal map
entry database is updated, using one of the modes described below. Then
the list of pending map items is cleared.  All \type{\pdfmapfile}
and \type{\pdfmapline} commands can also be given after shipout of the
first page.

If your map file isn't in the current directory (or a~standard system
directory), you will need to set the \type{TEXFONTMAPS} variable (in
\WEBC) or give an explicit path so that it will be found.

\type{\pdfmapfile{foo.map}} (\type{+}/\type{=}/\type{-} flags omitted)
clears the list of pending map items and starts a~new list with the only
pending file \type{foo.map}. When the file \filename{foo.map} is scanned,
duplicate map entries are ignored and a~warning is issued. When this
command is given at the beginning of a~\TEX\ run, the default map file
\filename{pdftex.map} will {\em not} be read in.  This is compatible
with the former behaviour.

If you want to add support for a~new font through an additional font map
file while keeping all the existing mappings, don't use this version
of command, but instead type either \type{\pdfmapfile{+myfont.map}}
or \type{\pdfmapfile{=myfont.map}}, as described below.

\type{\pdfmapfile {+foo.map}} puts the file \filename{foo.map} into the
list of pending map items. When the file \filename{foo.map} is scanned,
duplicate map entries are ignored and a~warning is issued. This is
compatible with the former behaviour.

\type{\pdfmapfile {=foo.map}} puts the file \filename{foo.map} into the
list of pending map items. When the file \filename{foo.map} is scanned,
matching map entries in the database are replaced by new entries from
\filename{foo.map}.

\type{\pdfmapfile {-foo.map}} puts the file \filename{foo.map} into the
list of pending map items. When the file \filename{foo.map} is scanned,
matching map entries are deleted from the database.

\type{\pdfmapfile {}} clears the list of pending map items. It does
not affect map entries already registered into the database. This is
compatible with the former behaviour.  When this command is given at the
beginning of a~\PDFTEX\ run, the default map file \filename{pdftex.map}
will {\em not} be read in.  This may help with quick program startup,
if no fonts are required.

If you want to use a~base map file name other than \type{pdftex.map},
or change its processing options through a~\PDFTEX\ format, you can do
this by appending the \type{\pdfmapfile} command to the
\type{\everyjob{}} token list for the \type{-ini} run, \eg:

\starttyping
\everyjob\expandafter{\the\everyjob\pdfmapfile{+myspecial.map}}
\dump
\stoptyping

\pdftexprimitive {\Syntax {\Tex {\pdfmapline} \Something{map spec}}}

\bookmark{\type{\pdfmapline}}

Similar to \type{\pdfmapfile}, but here you can give a~single
map line (like the ones in map files) as an argument. The modifiers
(\type{+-=}) have the same effect as with \type{\pdfmapfile}; see also
the description above.  Example:

\starttyping
\pdfmapline{+ptmri8r Times-Italic <8r.enc <ptmri8a.pfb}
\stoptyping

This primitive (especially the \type{\pdfmapline{=...}} variant) allows
quick checks of a~new font map entry, before writing it into
a~map file.

\type{\pdfmapline {}} clears the list of pending map items a~similar
way as \type{\pdfmapfile{}} does. \introduced{1.20a}

\pdftexprimitive {\Syntax {\Tex {\pdftracingfonts} \Whatever{integer}}}

\bookmark{\type{\pdftracingfonts}}

This integer parameter specifies the level of verbosity for info about
expanded fonts given in the log, \eg\ when \type{\tracingoutput=1}.
If \type{\pdftracingfonts=0}, which is the default, the log shows the
actual non-zero signed expansion value for each expanded letter within
brackets, \eg:

\starttyping
...\xivtt (+20) t
\stoptyping

If \type{\pdftracingfonts=1}, also the name of the \TFM\ file is listed,
together with the font size, \eg:

\starttyping
...\xivtt (cmtt10+20@14.0pt) t
\stoptyping

Setting \type{\pdftracingfonts} to a~value other than~0 or~1 is not
recommended, to allow future extensions.  \introduced{1.30.0}

\pdftexprimitive {\Syntax {\Tex {\pdfmovechars} \Whatever{integer}}}

\bookmark{\type{\pdfmovechars}}

Since \PDFTEX\ version 1.30.0 the primitive \type{\pdfmovechars} is obsolete,
and its use merely leads to a~warning. (This primitive specified whether
\PDFTEX\ should try to move characters in range 0..31 to higher slots;
its sole purpose was to remedy certain bugs of early \PDF\ viewers.)

\pdftexprimitive {\Syntax {\Tex {\pdfpkmode} \Whatever{tokens}}}

\bookmark{\type{\pdfpkmode}}

The \type{\pdfpkmode} is a~token register that sets the \METAFONT\ mode for
pixel font generation. The contents of this register is dumped into the
format, so one can (optionally) preset it \eg\ in \type{pdftexconfig.tex}.
\introduced{1.30.0}

\pdftexprimitive {\Syntax {\Tex {\pdfnoligatures} \Something{font}}}

\bookmark{\type{\pdfnoligatures}}

This diables all ligatures in the loaded font \Something{font}.
\introduced{1.30.0}

%***********************************************************************

\subsection{\PDF\ objects}

\pdftexprimitive {\Syntax {\Tex {\pdfobj} \Something{object type spec}}}

\bookmark{\type{\pdfobj}}

This command creates a~raw \PDF\ object that is written to the \PDF\ file as
\type{1 0 obj} \unknown\ \type{endobj}. The object is written to \PDF\ output as
provided by the user. When \Something{object type spec} is not given, \PDFTEX\
does not any longer create a~dictionary object with contents \Something{general text},
as it was in the past.

When however \Something{object type spec} is given as \Something{attr spec}
\Literal{stream}, the object will be created as a~stream with contents
\Something{general text} and additional attributes in \Something{attr
spec}\unkern.

When \Something{object type spec} is given as \Something{attr spec}
\Literal{file}, then the \Something{general text} will be treated as a~file
name and its contents will be copied into the stream contents.

When \Something{object type spec} is given as \type{reserveobjnum},
just a~new object number is reserved. The number of the reserved object
is accessible via \type{\pdflastobj}. The object can later be filled
with contents by \Syntax{\Tex{\pdfobj} \Literal{useobjnum}
\Something{number} \Lbrace\Something{balanced text}\Rbrace}.
But the reserved object number can already be be used before by other
objects, which provides a~forward||referencing mechanism.

The object is kept in memory and will be written to the \PDF\ output only
when its number is referred to by \type{\pdfrefobj} or when \type{\pdfobj}
is preceded by \type{\immediate}. Nothing is appended to the list being
built. The number of the most recently created object is accessible via
\type{\pdflastobj}.

\pdftexprimitive {\Syntax {\Tex {\pdflastobj} \Whatever{read||only integer}}}

\bookmark{\type{\pdflastobj}}

This command returns the object number of the last object created by \type
{\pdfobj}.

\pdftexprimitive {\Syntax {\Tex {\pdfrefobj} \Something{object number}}}

\bookmark{\type{\pdfrefobj}}

This command appends a~whatsit node to the list being built. When the whatsit
node is searched at shipout time, \PDFTEX\ will write the object
\Something{object number}
to the \PDF\ output if it has not been written yet.

%***********************************************************************

\subsection{Page and pages objects}

\pdftexprimitive {\Syntax {\Tex {\pdfpagesattr} \Whatever{tokens}}}

\bookmark{\type{\pdfpagesattr}}

\PDFTEX\ expands this token list when it finishes the \PDF\ output and adds
the resulting character stream to the root \type{Pages} object. When defined,
these are applied to all pages in the document. Some examples of attributes
are \type{/MediaBox}, the rectangle specifying the natural size of the page,
\type{/CropBox}, the rectangle specifying the region of the page being
displayed and printed, and \type{/Rotate}, the number of degrees (in
multiples of 90) the page should be rotated clockwise when it is displayed or
printed.

% /MediaBox is not a good example, since will never take an effect

\starttyping
\pdfpagesattr
  { /Rotate 90                % rotate all pages by 90 degrees
    /CropBox [0 0 612 792] }  % the crop size of all pages (in bp)
\stoptyping

\pdftexprimitive {\Syntax {\Tex {\pdfpageattr} \Whatever{tokens}}}

\bookmark{\type{\pdfpageattr}}

This is similar to \type{\pdfpagesattr}, but has priority over it.
It can be used to override any attribute given by \type{\pdfpagesattr}
for individual pages. The token list is expanded when \PDFTEX\ ships out
a~page. The contents are added to the attributes of the current page.

\pdftexprimitive {\Syntax {\Tex {\pdfpageref} \Something{page number} \Whatever{expandable}}}

\bookmark{\type{\pdfpageref}}

This primitive expands to the number of the page object that contains the
dictionary for page \Something{page number}. If the page \Something{page
number} does not exist, a~warning will be issued, a fresh unused \PDF\
object will be generated, and \type{\pdfpageref} will expand to that
object number.

\Eg, if the dictionary for page~5 of the \TEX\ document is contained in
\PDF\ object no.~18, \tex{pdfpageref5} expands to the number 18.

%***********************************************************************

\subsection{Form XObjects}

The next three primitives support a~\PDF\ feature called \quote {object
reuse} in \PDFTEX. The idea is first to create a~`form XObject' in
\PDF. The content of this object corresponds to the content of a~\TEX\
box; it can contain pictures and references to other form XObjects
as well. After creation, the form XObject can be used multiple times
by simply referring to its object number. This feature can be useful
for large documents with many similar elements, as it can reduce the
duplication of identical objects.

These commands behave similarly to \type{\pdfobj}, \type{\pdfrefobj}
and \type{\pdflastobj}, but instead of taking raw \PDF\ code, they handle
text typeset by \TEX.

\pdftexprimitive {\Syntax {\Tex {\pdfxform} \Optional{\Something{attr
spec}} \Optional{\Something{resources spec}} \Something{box number}}}

\bookmark{\type{\pdfxform}}

This command creates a~form XObject corresponding to the contents of the
box \Something{box number}. The box can contain other raw objects, form
XObjects, or images as well. It can however {\em not} contain annotations
because they are laid out on a~separate layer, are positioned absolutely,
and have dedicated housekeeping. \type{\pdfxform} makes the box void,
as \type{\box} does.

When \Something{attr spec} is given, the text will be written
as additional attribute into the form XObject dictionary. The
\Something{resources spec} is similar, but the text will be added
to the resources dictionary of the form XObject. The text given by
\Something{attr spec} or \Something{resources spec} is written before
other entries of the form dictionary and|/|or the resources dictionary
and takes priority over later ones.

\pdftexprimitive {\Syntax {\Tex {\pdfrefxform} \Something{object number}}}

\bookmark{\type{\pdfrefxform}}

The form XObject is kept in memory and will be written to the \PDF\
output only when its object number is referred to by \type{\pdfrefxform}
or when \type{\pdfxform} is preceded by \type{\immediate}. Nothing is
appended to the list being built. The number of the most recently created
form XObject is accessible via \type {\pdflastxform}.

When issued, \type{\pdfrefxform} appends a~whatsit node to the list being
built. When the whatsit node is searched at shipout time, \PDFTEX\ will
write the form \Something{object number} to the \PDF\ output if it is
not written yet.

\pdftexprimitive {\Syntax {\Tex {\pdflastxform}} \Whatever{read||only integer}}

\bookmark{\type{\pdflastxform}}

The object number of the most recently created form XObject is accessible
via \type {\pdflastxform}.

As said, this feature can be used for reusing information. This mechanism
also plays a~role in typesetting fill||in forms. Such widgets sometimes
depends on visuals that show up on user request, but are hidden otherwise.

\pdftexprimitive {\Syntax {\Tex {\pdfxformname} \Something{object number} \Whatever{expandable}}}

\bookmark{\type{\pdfxformname}}

In \PDF\ files produced by \PDFTEX\ one can recognize a~form Xobject by
the prefix~\type{/Fm} followed by a~number, for instance \type{/Fm2}.
For a~given form XObject number, this primitive expands to the number
in the corresponding form XObject name. \Eg, if \type{/Fm2} corresponds
to some form XObject with object number 7, the \type{\pdfxformname7}
expands to the number~\type{2}. \introduced{1.30.0}

%***********************************************************************

\subsection{Graphics inclusion}

\PDF\ provides a~mechanism for embedding graphic and textual objects: form
XObjects.  In \PDFTEX\ this mechanism is accessed by means of \type{\pdfxform},
\type{\pdflastxform} and \type{\pdfrefxform}. A~special kind of XObjects
are bitmap graphics and for manipulating them similar commands are provided.

\pdftexprimitive {\Syntax {
  \Tex {\pdfximage}
  \Optional{\Something{rule spec}}
  \Optional{\Something{attr spec}}
  \Optional{\Something{page spec}}
  \Optional{\Something{colorspace spec}}
  \Optional{\Something{pdf box spec}}
  \Something{general text}
}}

\bookmark{\type{\pdfximage}}

This command creates an image object. The dimensions of the image can be
controlled via \Something{rule spec}\unkern. The default values are zero for depth
and \quote {running} for height and width. If all of them are given, the
image will be scaled to fit the specified values. If some (but not
all) are given, the rest will be set to a~value corresponding to the
remaining ones so as to make the image size to yield the same proportion of
$width : (height+depth)$ as the original image size, where depth is treated
as zero. If none are given then the image will take its natural size.

An image inserted at its natural size often has a~resolution of \type
{\pdfimageresolution} (see below) given in dots per inch in the output file,
but some images may contain data specifying the image resolution, and in such
a~case the image will be scaled to the correct resolution. The dimensions of
an image can be accessed by enclosing the \type{\pdfrefximage} command to
a~box and checking the dimensions of the box:

\starttyping
\setbox0=\hbox{\pdfximage{somefile.png}\pdfrefximage\pdflastximage}
\stoptyping

Now we can use \type{\wd0} and \type{\ht0} to question the natural size of
the image as determined by \PDFTEX. When dimensions are specified before the
\type{{somefile.png}}, the graphic is scaled to fit these. Note that, unlike
the \eg\ \type{\input} primitive, the filename is supplied between
braces.

The image type is specified by the extension of the given file name: \type
{.png} stands for \PNG\ image, \type{.jpg} for \JPEG, and \type{.pdf} for \PDF\
file. But once \PDFTEX\ has opened the file, it checks the file type first by
looking to the magic number at the file start, which gets precedence over the
file name extension.  This gives a~certain degree of fault tolerance, if the
file name extension is stated wrongly.

Similarly to \type{\pdfxform}, the optional text given by \Something{attr spec}
will be written as additional attributes of the image before other keys of the
image dictionary. One should be aware, that slightly different type of PDF
object is created while including \PNG\ or \JPEG\ bitmaps and \PDF\ images.

While working with \PDF\ image, \Something{page spec} allows to decide which
page of the document is to be included. Starting from \PDFTEX\ 1.11 one may
also decide which \PDF\ page box of the image is to be treated as a~final
bounding box. If \Something{pdf box spec} is present, overrides default
behaviour specified by \type{\pdfforcepagebox} parameter. Both options are
irrelevant for non||\PDF\ inclusions.

Starting from \PDFTEX\ 1.21, \type{\pdfximage} command supports
\type{colorspace} keyword followed by an object number (user||defined
colorspace for the image being included). This feature works for \JPEG\ images
only. \PNG s are \RGB\ palettes and \PDF\ images have always self||contained
color space information.

\pdftexprimitive {\Syntax {\Tex {\pdfrefximage} \Something{integer}}}

\bookmark{\type{\pdfrefximage}}

The image is kept in memory and will be written to the \PDF\ output only when
its number is referred to by \type{\pdfrefximage} or \type{\pdfximage} is
preceded by \type{\immediate}. Nothing is appended to the list being built.

\type{\pdfrefximage} appends a~whatsit node to the list being built. When
the whatsit node is searched at shipout time, \PDFTEX\ will write the image
with number \Something{integer} to the \PDF\ output if it has not been
written yet.

\pdftexprimitive {\Syntax {\Tex {\pdflastximage} \Whatever{read||only
integer}}}

\bookmark{\type{\pdflastximage}}

The number of the most recently created XObject image is accessible via \type
{\pdflastximage}.

\pdftexprimitive {\Syntax {\Tex {\pdflastximagepages} \Whatever{read||only
integer}}}

\bookmark{\type{\pdflastximagepages}}

This read||only register returns the highest page number from a~file
previously accessed via the \type{\pdfximage} command.
This is useful only for \PDF\ files; it always returns \type{1}
for \PNG\ or \JPEG\ files.

\pdftexprimitive {\Syntax {\Tex {\pdfimageresolution} \Whatever{integer}}}

\bookmark{\type{\pdfimageresolution}}

The integer \type{\pdfimageresolution} parameter (unit: dots per
inch, dpi) is a~last resort value, used only for bitmap (\JPEG,
\PNG) images, but not for \PDF{}s. The priorities are as follows:
Often one image dimension (\type{width} or \type{height}) is stated
explicitely in the \TEX\ file. Then the image is properly scaled so
that the aspect ratio is kept. If both image dimensions are given,
the image will be stretched accordingly, whereby the aspect ratio might
get distorted. Only if no image dimension is given in the \TEX\ file,
the image size will be calculated from its width and height in pixels,
using the $x$ and $y$ resolution values normally contained in the image
file. If one of these resolution values is missing or weird (either
$<$~0 or $>$~65535), the \type{\pdfimageresolution} value will be used
for both $x$ and $y$ resolution, when calculating the image size. And
if the \type{\pdfimageresolution} is zero, finally a~default resolution
of 72\,dpi would be taken. The \type{\pdfimageresolution} is read when
\PDFTEX\ creates an image via \type{\pdfximage}. The given value is
clipped to the range 0..65535\,[dpi].

Currently this parameter is used particularily for calculating the
dimensions of \JPEG\ images in \EXIF\ format (unless at least one
dimension is stated explicitely); the resolution values coming with \EXIF\
files are currently ignored.

\pdftexprimitive {\Syntax {\Tex {\pdfforcepagebox} \Whatever{integer}}}

\bookmark{\type{\pdfforcepagebox}}

%- It is now possible to specify the pdf page box to use when including pdfs.
%  The syntax has been extended:
%    \pdfximage [<image attr spec>] <general text>           (h, v, m)
%    <image attr spec> --> [<rule spec>] [<attr spec>] [<page spec>] [<pdf box spec>]
%    <pdf box spec> --> mediabox|cropbox|bleedbox|trimbox|artbox
%  The default is cropbox (which defaults to mediabox), to be compatible with
%  previous versions of pdfTeX.
%    The page box specified at \pdfximage can be globally overridden with the
%  config parameter always_use_pdfpagebox and the command
%  \pdfoptionalwaysusepdfpagebox <integer>, where 0 is the default (use whatever
%  is specified with \pdfximage), 1 is media, 2 is crop, 3 is bleed, 4 is trim
%  and 5 is artbox. This can only be set once per object, i.\,e.\ the value used at
%  \pdfximage is remembered.
%    See the pdf reference for an explanation of the boxes.

When \PDF\ files are included, the command \type{\pdfximage} allows the selection
of which \PDF\ page box to use in the optional field \Something{image attr
spec}\unkern.  The integer primitive \type{\pdfforcepagebox} allows to globally
override this choice by giving them one of the following values: (1) media box,
(2) crop box, (3) bleed box, (4) trim box, and (5) artbox. The command is
available starting from \PDFTEX\ 1.30.0, as a~shortened synonym of obsolete
\type{\pdfoptionalwaysusepdfpagebox} instruction.

\pdftexprimitive {\Syntax {\Tex {\pdfinclusionerrorlevel} \Whatever{integer}}}

\bookmark{\type{\pdfinclusionerrorlevel}}

This controls the behaviour of \PDFTEX\ when a~\PDF\ file is included
that has a~newer version than the one specified by this primitive:
If it is set to~0, \pdfTeX\ gives only a~warning; if it's~1, \pdfTeX\
raises an error. The command has been introduced in \PDFTEX\ 1.30.0 as
a~shortened synonym of \type{\pdfoptionpdfinclusionerrorlevel}, that is
now obsolete.

\pdftexprimitive {\Syntax {\Tex {\pdfimagehicolor} \Whatever{integer}}}

\bookmark{\type{\pdfimagehicolor}}

This primitive, when set to~1, enables embedding of \PNG\ images with
16~bit wide color channels at their full color resolution. As such
an embedding mode is defined only from \PDF\ version~1.5 onwards,
the \type{\pdfimagehicolor} functionality is automatically disabled
in \PDFTEX\ if \type{\pdfminorversion}~$<$~5; then each 16~bit
color channel is reduced to a~width of 8~bit by stripping the lower 8~bits
before embedding. The same stripping happens when \type{\pdfimagehicolor}
is set to~0. For \type{\pdfminorversion}~$\ge$~5 the default
value of \type{\pdfimagehicolor} is~1. If specified, the parameter
must appear before any data is written to the \PDF\ output.
\introduced{1.30.0}

\pdftexprimitive {\Syntax {\Tex {\pdfimageapplygamma} \Whatever{integer}}}

\bookmark{\type{\pdfimageapplygamma}}

This primitive, when set to~1, enables gamma correction while embedding
\PNG\ images, taking the values of the primitives \type {\pdfgamma} as
well as the gamma value embedded in the \PNG\ image into account. When
\type{\pdfimageapplygamma} is set to~0, no gamma correction is performed.
If specified, the parameter must appear before any data is written to the
\PDF\ output.
\introduced{1.30.0}

\pdftexprimitive {\Syntax {\Tex {\pdfgamma} \Whatever{integer}}}

\bookmark{\type{\pdfgamma}}

This primitive defines the `device gamma' for \PDFTEX. Values are in
promilles (same as for \type{\mag}). The default value of this primitive
is~1000, defining a~device gamma value of~1.0.

When \type{\pdfimageapplygamma} is set to~1, then whenever a~\PNG\ image
is included, \PDFTEX\ applies a~gamma correction. This correction is
based on the  value of the \type{\pdfgamma} primitive and the `assumed
device gamma' that is derived from the value embedded in the actual
image. If no embedded value can be found in the \PNG\ image, then the
value of \type{\pdfimagegamma} is used instead.
If specified, the parameter must appear before any data is written to the
\PDF\ output.
\introduced{1.30.0}

\pdftexprimitive {\Syntax {\Tex {\pdfimagegamma} \Whatever{integer}}}

\bookmark{\type{\pdfimagegamma}}

This primitive gives a~default `assumed gamma' value for \PNG\ images.
Values are in promilles (same as for \type{\pdfamma}). The default value
of this primitive is~2200, implying an assumed gamma value of~2.2.

When \PDFTEX\ is applying gamma corrections, images that do not have an
embedded `assumed gamma' value are assumed to have been created for
a~device with a~gamma of 2.2. Experiments show that this default setting
is correct for a~large number of images; however, if your images come
out too dark, you probably want to set \type{\pdfimagegamma} to a~lower
value, like~1000.
If specified, the parameter must appear before any data is written to the
\PDF\ output.
\introduced{1.30.0}

%***********************************************************************

\subsection{Annotations}

\PDF\ 1.4 provides four basic kinds of annotations:

\startitemize[packed]
\item hyperlinks, general navigation
\item text clips (notes)
\item movies
\item sound fragments
\stopitemize

The first type differs from the other three in that there is a~designated
area involved on which one can click, or when moved over some action occurs.
\PDFTEX\ is able to calculate this area, as we will see later. All
annotations can be supported using the next two general annotation
primitives.

\pdftexprimitive {\Syntax {\Tex {\pdfannot} \Something{annot type spec}}}

\bookmark{\type{\pdfannot}}

This command appends a~whatsit node corresponding to an annotation to the
list being built. The dimensions of the annotation can be controlled via the
\Something{rule spec}. The default values are running for all width, height
and depth. When an annotation is written out, running dimensions will take
the corresponding values from the box containing the whatsit node
representing the annotation. The \Something{general text} is inserted as raw
\PDF\ code to the contents of annotation. The annotation is written out only
if the corresponding whatsit node is searched at shipout time.

\pdftexprimitive {\Syntax {\Tex {\pdflastannot} \Whatever{read||only
integer}}}

\bookmark{\type{\pdflastannot}}

This primitive returns the object number of the last annotation created by
\type{\pdfannot}. These two primitives allow users to create any annotation
that cannot be created by \type{\pdfstartlink} (see below).

%***********************************************************************

\subsection[linking]{Destinations and links}

The first type of annotation, mentioned above, is implemented by three
primitives. The first one is used to define a~specific location as being
referred to. This location is tied to the page, not the exact location on the
page. The main reason for this is that \PDF\ maintains a~dedicated list of
these annotations |<|and some more when optimized|>| for the sole purpose of
speed.

\pdftexprimitive {\Syntax {\Tex {\pdfdest} \Something{dest spec}}}

\bookmark{\type{\pdfdest}}

This primitive appends a~whatsit node which establishes a~destination for
links and bookmark outlines; the link is identified by either a~number or
a~symbolic name, and the way the viewer is to display the page must be
specified in \Something{dest type}\unkern, which must be one of those mentioned in
\in{table}[appearance].

\startbuffer
\starttabulate[|l|l|]
\HL
\NC \bf keyword \NC \bf meaning                                       \NC\NR
\HL
\NC \tt fit    \NC fit the page in the window                         \NC\NR
\NC \tt fith   \NC fit the width of the page                          \NC\NR
\NC \tt fitv   \NC fit the height of the page                         \NC\NR
\NC \tt fitb   \NC fit the \quote{Bounding Box} of the page           \NC\NR
\NC \tt fitbh  \NC fit the width of \quote{Bounding Box} of the page  \NC\NR
\NC \tt fitbv  \NC fit the height of \quote{Bounding Box} of the page \NC\NR
\NC \tt xyz    \NC goto the current position (see below)              \NC\NR
\HL
\stoptabulate
\stopbuffer

\placetable
  [here][appearance]
  {Options for display of outline and destinations.}
  {\getbuffer}

The specification \Literal {xyz} can optionally be followed by \Literal
{zoom} \Something{integer} to provide a~fixed zoom||in. The \Something
{integer} is processed like \TEX\ magnification, i.\,e.\ 1000 is the
normal page view. When \Literal {zoom} \Something{integer} is given,
the zoom factor changes to 0.001 of the \Something{integer} value,
otherwise the current zoom factor is kept unchanged.

The destination is written out only if the corresponding whatsit node is
searched at shipout time.

\pdftexprimitive {\Syntax {\Tex {\pdfstartlink} \Optional{\Something{rule
spec}} \Optional{\Something{attr spec}} \Something{action spec}}}

\bookmark{\type{\pdfstartlink}}

This primitive is used along with \type{\pdfendlink} and appends a~whatsit
node corresponding to the start of a~hyperlink. The whatsit node representing
the end of the hyperlink is created by \type{\pdfendlink}. The dimensions of
the link are handled in the similar way as in \type{\pdfannot}. Both \type
{\pdfstartlink} and \type{\pdfendlink} must be in the same level of box
nesting. A~hyperlink with running width can be multi||line or even
multi||page, in which case all horizontal boxes with the same nesting level
as the boxes containing \type{\pdfstartlink} and \type{\pdfendlink} will be
treated as part of the hyperlink. The hyperlink is written out only if the
corresponding whatsit node is searched at shipout time.

Additional attributes, which are explained in great detail in the \PDF\
Reference Manual, can be given via \Something{attr spec}\unkern. Typically, the
attributes specify the color and thickness of any border around the link.
Thus \typ {/C [0.9 0 0] /Border [0 0 2]} specifies a~color (in \RGB) of dark
red, and a~border thickness of 2~points.

While all graphics and text in a~\PDF\ document have relative positions,
annotations have internally hard||coded absolute positions. Again this
is for the sake of speed optimization. The main disadvantage is that these
annotations do {\em not} obey transformations issued by \type
{\pdfliteral}'s.

The \Something{action spec} specifies the action that should be performed
when the hyperlink is activated while the \Something{user-action spec}
performs a~user||defined action. A~typical use of the latter is to specify
a~\URL, like \typ {/S /URI /URI (http://www.tug.org/)}, or a~named action like
\typ {/S /Named /N /NextPage}.

 A~\Something{goto-action spec} performs a~GoTo action. Here \Something
{numid} and \Something{nameid} specify the destination identifier (see
below). The \Something{page spec} specifies the page number of the
destination, in this case the zoom factor is given by \Something{general
text}\unkern. A~destination can be performed in another \PDF\ file by specifying
\Something{file spec}\unkern, in which case \Something{newwindow spec} specifies
whether the file should be opened in a~new window. A~\Something{file spec}
can be either a~\type{(string)} or a~\type{<<dictionary>>}. The default
behaviour of the \Something{newwindow spec} depends on the browser setting.

 A~\Something{thread-action spec} performs an article thread reading. The
thread identifier is similar to the destination identifier. A~thread can be
performed in another \PDF\ file by specifying a~\Something{file spec}\unkern.

\pdftexprimitive {\Syntax {\Tex {\pdfendlink}}}

\bookmark{\type{\pdfendlink}}

This primitive ends a~link started with \type{\pdfstartlink}. All text
between \type{\pdfstartlink} and \type{\pdfendlink} will be treated as part
of this link. \PDFTEX\ may break the result across lines (or pages), in which
case it will make several links with the same content.

\pdftexprimitive {\Syntax {\Tex {\pdflinkmargin} \Whatever{dimension}}}

\bookmark{\type{\pdflinkmargin}}

This dimension parameter specifies the margin of the box representing
a~hyperlink and is read when a~page containing hyperlinks is shipped out.

\pdftexprimitive {\Syntax {\Tex {\pdfdestmargin} \Whatever{dimension}}}

\bookmark{\type{\pdfdestmargin}}

Margin added to the dimensions of the rectangle around the destinations.

%***********************************************************************

\subsection{Bookmarks}

\pdftexprimitive {\Syntax {\Tex {\pdfoutline}
\Optional{\Something{attr spec}} \Something{action spec}
\Optional{\Literal {count} \Something{integer}} \Something{general text}}}

\bookmark{\type{\pdfoutline}}

This primitive creates an outline (or bookmark) entry. The first parameter
specifies the action to be taken, and is the same as that allowed for \type
{\pdfstartlink}. The \Something{count} specifies the number of direct
subentries under this entry; specify~0 or omit it if this entry has no
subentries. If the number is negative, then all subentries will be closed and
the absolute value of this number specifies the number of subentries. The
\Something{text} is what will be shown in the outline window. Note that this
is limited to characters in the \PDF\ Document Encoding vector. The outline
is written to the \PDF\ output immediately.

%***********************************************************************

\subsection{Article threads}

\pdftexprimitive {\Syntax {\Tex {\pdfthread}
\Optional{\Something{rule spec}} \Optional
{\Something{attr spec}} \Something{id spec}}}

\bookmark{\type{\pdfthread}}

Defines a~bead within an article thread. Thread beads with same
identifiers (spread across the document) will be joined together.

\pdftexprimitive {\Syntax {\Tex {\pdftstarthread}
\Optional{\Something{rule spec}} \Optional
{\Something{attr spec}} \Something{id spec}}}

\bookmark{\type{\pdfstartthread}}

This uses the same syntax as \type{\pdfthread}, apart that it must be
followed by a~\type{\pdfendthread}.  \type{\pdfstartthread} and the
corresponding \type{\pdfendthread} must end up in vboxes with the same
nesting level; all vboxes between them will be added into the thread.
Note that during output runtime if there are other newly created
boxes which have the same nesting level as the vbox/vboxes containing
\type{\pdfstartthread} and \type{\pdfendthread}, they will be also added
into the thread, which is probably not what you want. To avoid such
unconsidered behaviour, it's often enough to wrap boxes that shouldn't
belong to the thread by a~box to change their box nesting level.

\pdftexprimitive {\Syntax {\Tex {\pdfendthread}}}

\bookmark{\type{\pdfendthread}}

This ends an article thread started before by \type{\pdfstartthread}.

\pdftexprimitive {\Syntax {\Tex {\pdfthreadmargin} \Whatever{dimension}}}

\bookmark{\type{\pdfthreadmargin}}

Specifies a~margin to be added to the dimensions of a~bead within
an article thread.

%***********************************************************************

\subsection{Literals and specials}

\pdftexprimitive {\Syntax {\Tex {\pdfliteral } \Optional{\Literal {direct}}
\Something{general text}}}

\bookmark{\type{\pdfliteral}}

Like \type{\special} in normal \TEX, this command inserts raw \PDF\ code
into the output. This allows support of color and text transformation. This
primitive is heavily used in the \METAPOST\ inclusion macros. Normally
\PDFTEX\ ends a~text section in the \PDF\ output and resets the
transformation matrix before inserting \Something{general text}\unkern, however this
can be turned off by giving the optional keyword \Literal {direct}. This
command appends a~whatsit node to the list being built. \Something{general
text} is expanded when the whatsit node is created and not when it is shipped
out, as with \type{\special}.

% HE: \unkern is a kludge here; wanted to have tight "{pdf:"
\pdftexprimitive {\Syntax {\Tex {\special} \Lbrace\unkern\type{pdf:} \Something{text} \Rbrace}}

\bookmark{\type{\special}}

This is equivalent to \Syntax{\Tex{\pdfliteral} \Lbrace \Something{text} \Rbrace}.

\pdftexprimitive {\Syntax {\Tex {\special} \Lbrace\unkern\type{pdf:direct:} \Something{text} \Rbrace}}

\bookmark{\type{\special\ direct}}

This is equivalent to \Syntax{\Tex{\pdfliteral} \Literal{direct} \Lbrace \Something{text} \Rbrace}.

%***********************************************************************

\subsection{Strings}

\pdftexprimitive {\Syntax {\Tex {\pdfescapestring} \Something {general text}}}

\bookmark{\type{\pdfescapestring}}

Starting from version 1.30.0, \PDFTEX\ provides a~mechanism for converting
a~general text into \PDF\ string. Many characters that may be needed inside
such a~text (especially parenthesis), have a~special meaning inside a~\PDF\
string object and thus, can't be used literally. The primitive replaces each
special \PDF\ character by its literal representation by inserting a~backslash
before that character. Some characters (\eg space) are also converted into
3||digit octal number. In example, \type{\pdfescapestring{Text (1)}} will be
expanded to \type{Text\040\(1\)}. This ensures a~literal interpretation of the
text by the \PDF\ viewer.
\introduced{1.30.0}

\pdftexprimitive {\Syntax {\Tex {\pdfescapename} \Something {general text}}}

\bookmark{\type{\pdfescapename}}

In analogy to \type{\pdfescapestring}, \type{\pdfescapename} replaces each
special \PDF\ character inside the general text by its hexadecimal
representation preceded by \type{#} character. This ensures the proper
interpretation of the text if used as a~\PDF\ name object. In example,
\type{Text (1)} will be replaced by \type{Text#20#281#29}.
\introduced{1.30.0}

\pdftexprimitive {\Syntax {\Tex {\pdfescapehex} \Something {general text}}}

\bookmark{\type{\pdfescapehex}}

This command converts each character of \Something{general text} into its
hexadecimal representation. Each character of the argument becomes a~pair of
hexadecimal digits. \introduced{1.30.0}

\pdftexprimitive {\Syntax {\Tex {\pdfunescapehex} \Something {general text}}}

\bookmark{\type{\pdfunescapehex}}

This command treats each character pair of \Something {general text} as
a~hexadecimal number and returns an \ASCII\ character of this code.
\introduced{1.30.0}

% Currently it's only experimental, so don't really document it
\iffalse
\pdftexprimitive {\Syntax {\Tex {\pdfstrcmp} \Something{general text}
\Something{general text} \Whatever{expandable}}}

\bookmark{\type{\pdfstrcmp}}

This command compares two strings and expands to 0 if the strings are
equal, to $-1$ if the first string is $<$ then the second, and to $+1$ if
the first is $>$ than the second.
\introduced{1.30.0}
\fi

\pdftexprimitive{\Syntax {\Tex {\pdfmdfivesum} \Something{general text}
\Whatever{expandable}}}

\bookmark{\type{\pdfmdfivesum}}

This command expands to the \MDFIVE\ of \Something{general text} in uppercase
hexadecimal format (same as \type{\pdfescapehex}).
\introduced{1.30.0}

%***********************************************************************

\subsection{Timekeeping}

\pdftexprimitive {\Syntax {\Tex {\pdfelapsedtime}} \Whatever{read||only
integer}}

\bookmark{\type{\pdfelapsedtime}}

The command expands to a~number that represents the time elapsed from the
moment of run start. The elapsed time is returned in `scaled seconds', that
means seconds divided by 65536, \eg\ \PDFTEX\ has run for \the\pdfelapsedtime\
`scaled seconds' when this paragraph was typeset. Obviously, the command will
never return a value greater than the highest number available in \TeX: if
the time exceeds 32767 seconds, the constant value $2^{31}-1$ will be returned.
\introduced{1.30.0}

\pdftexprimitive{\type{\pdfresettimer}}

\bookmark{\type{\pdfresettimer}}

The command resets the internal timer so that \type{\pdfelapsedtime}
starts returning micro||time from~0 again. \introduced{1.30.0}

%***********************************************************************

\subsection{Random numbers}

\pdftexprimitive{\Syntax {\Tex {\pdfuniformdeviate} \Something{number} \Whatever{expandable}}}

\bookmark{\type{\pdfuniformdeviate}}

The commands generates a uniformly distributed random integer value between
0 (inclusive) and \Something{number} (exclusive). This primitive expands to
a list of tokens. \introduced{1.30.0}

\pdftexprimitive{\Syntax {\Tex {\pdfnormaldeviate} \Whatever{expandable}}}

\bookmark{\type{\pdfnormaldeviate}}

%MS Hopefully the random number doesn't influce the length of this paragraph;
%MS otherwise texexec could loop :-)
The commands generates a random integer value with a mean of 0 and a unit
of 65\,536, \eg\ $\pdfnormaldeviate$. This primitive expands to a list of
tokens. \introduced{1.30.0}

\pdftexprimitive {\Syntax {\Tex {\pdfrandomseed}} \Whatever{read||only
integer}}

\bookmark{\type{\pdfrandomseed}}

You can use \type{\the\pdfrandomseed} to query the current seed value, so
you can \eg\ write the value to the log file.
The initial value of the seed is derived from the system time, and is not
more than 1\,000\,999\,999 (this ensures that the value can be used with
commands like \type{\count}). \introduced{1.30.0}

\pdftexprimitive{\Syntax {\Tex {\pdfsetrandomseed} \Something{number}}}

\bookmark{\type{\pdfsetrandomseed}}

This sets the random seed (\type{\pdfrandomseed}) to a specific value,
allowing you to re-play sequences of semi-randoms at a later moment.
\introduced{1.30.0}

%***********************************************************************

\subsection{Files}

\pdftexprimitive{\Syntax {\Tex {\pdffilemoddate} \Something{general text}
\Whatever{expandable}}}

\bookmark{\type{\pdffilemoddate}}

Expands to the modification date of file \Something{general text} in the same
format as for \type{\pdfcreationdate}, \eg\ it's {\tt \pdffilemoddate
{./pdftex-t.tex}} for the source of this manual.
\introduced{1.30.0}

\pdftexprimitive{\Syntax {\Tex {\pdffilesize} \Something{general text}
\Whatever{expandable}}}

\bookmark{\type{\pdffilesize}}

Expands to the size of file \Something{general text}, \eg\ it's {\tt
\pdffilesize {./pdftex-t.tex}} for the source of this manual.
\introduced{1.30.0}

\pdftexprimitive{\Syntax {\Tex {\pdfmdfivesum} file \Something{general text}
\Whatever{expandable}}}

\bookmark{\type{\pdfmdfivesum\ file}}

Expands to the \MDFIVE\ of file \Something{general text} in uppercase
hexadecimal format (same as \type{\pdfescapehex}), \eg\ it's {\tt
\pdfmdfivesum file {./pdftex-t.tex}} for the source of this manual.
\introduced{1.30.0}

\pdftexprimitive{\Syntax {\Tex {\pdffiledump} \Optional{offset
\Something{number}} \Optional{length \Something{number}} \Something{general
text} \Whatever{expandable}}}

\bookmark{\type{\pdffiledump}}

Expands to the dump of the file \Something{general text} in uppercase
hexadecimal format (same as \type{\pdfescapehex}), starting at offset
\Something{number} or 0 with length \Something{number}, if given. The first ten
bytes of the source of this manual are {\tt \pdffiledump length 10
{./pdftex-t.tex}}. \introduced{1.30.0}

%***********************************************************************

\subsection{Miscellaneous}

\pdftexprimitive {\Syntax {\Tex {\vadjust}
  \Optional{\Something{pre spec}} \Something{filler}
  \Lbrace \Something{vertical mode material} \Rbrace }}

\bookmark{\type{\vadjust}}

The \type{\vadjust} implementation of \PDFTEX\ adds an optional
qualifier \Something{pre spec} (which is the string \type{pre}) to
the original \TEX\ primitive with the same name. As long as there is
no \type{pre} given, \type{\vadjust} behaves exactly as the original
(see the \TEX book, p.~281); it appends an adjustment item created
from \Something{vertical mode material} to the current list {\em after}
the line in which \type{\vadjust} appears. However with the qualifier
\type{pre}, the adjustment item is put {\em before} the line in which
\type{\vadjust pre} appears.

\pdftexprimitive {\Syntax {\Tex {\pdfsavepos}}}

\bookmark{\type{\pdfsavepos}}

This primitive marks the current absolute ($x,y$) position on the
media, with the reference point in the lower left corner. It is active
only during page shipout, when the page is finally assembled. The
position coordinates can then be retrieved by the \type{\pdflastxpos}
and \type{\pdflastypos} primitives, and \eg\ written out to some
auxiliary file. The coordinates can be used only after the current
\type{\shipout} has been finalized, therefore normally two \PDFTEX\
runs are required to utilize these primitives.

\pdftexprimitive {\Syntax {\Tex {\pdflastxpos}} \Whatever{read||only
integer}}

\bookmark{\type{\pdflastxpos}}

This primitive returns an integer number representing the absolute $x$
coordinate of the last point marked by \type{\pdfsavepos}. The unit is
`scaled points' (sp).

\pdftexprimitive {\Syntax {\Tex {\pdflastypos}} \Whatever{read||only
integer}}

\bookmark{\type{\pdflastypos}}

This primitive works similar to \type{\pdflastxpos}, only it returns
the $y$ coordinate.

\pdftexprimitive {\Syntax {\Tex {\pdftexversion} \Whatever{read||only integer}}}

\bookmark{\type{\pdftexversion}}
\def\versplit#1#2#3{#1.#2#3}

Returns the version of \PDFTEX\ multiplied by 100, \eg\ for \PDFTEX\
version \expandafter\versplit\the\pdftexversion.\pdftexrevision\ used to
produce this document, it returns {\tt \number\pdftexversion}.

\pdftexprimitive {\Syntax {\Tex {\pdftexrevision} \Whatever{expandable}}}

\bookmark{\type{\pdftexrevision}}

Returns the revision letter of \PDFTEX, \eg\ for \PDFTEX\ version
\expandafter\versplit\the\pdftexversion.\pdftexrevision\ used to produce
this document, it returns the letter {\tt \pdftexrevision}.

\pdftexprimitive {\Syntax {\Tex {\pdftexbanner} \Whatever{expandable}}}

\bookmark{\type{\pdftexbanner}}

Returns the \PDFTEX\ banner message, \eg\ for the version used here:
{\tt \pdftexbanner}. \introduced{1.20a}

\pdftexprimitive {\Syntax {\Tex {\pdfcreationdate} \Whatever{expandable}}}

\bookmark{\type{\pdfcreationdate}}

Expands to the date string \PDFTEX\ uses in the info dictionary of the
document, \eg\ for this file {\tt\pdfcreationdate}. \introduced{1.30.0}

\pdftexprimitive {\Syntax {\Tex {\pdfshellescape} \Whatever{read||only
integer}}}

\bookmark{\type{\pdfshellescape}}

This primitive is~1 if \type{\write18} is enabled, 0 otherwise.
\type{\write18} was \ifnum\pdfshellescape=0\relax not \fi enabled when this
manual was typeset.
\introduced{1.30.0}

%***********************************************************************

\section{Graphics and color}

\PDFTEX\ supports inclusion of pictures in \PNG, \JPEG, and \PDF\ format;
 a~few differences between these are discussed below. The most common
technique with \TEX\ |<|the inclusion of \EPS\ figures|>| is replaced by
\PDF\ inclusion. \EPS\ files can be converted to \PDF\ by \GHOSTSCRIPT,
Adobe Distiller or other \POSTSCRIPT||to||\PDF\ converters.

The \PDF\ format is currently the most versatile source format for
graphics embedding. \PDFTEX\ allows to insert arbitrary pages from
\PDF\ files with their own fonts, graphics, and pixel images into
a~document. The cover page of this manual is an example of such an insert,
being a~one page document generated by \PDFTEX.

By default \PDFTEX\ takes the BoundingBox of a~\PDF\ file from its CropBox
if available, otherwise from its MediaBox. This can be influenced by
the \Something{pdf box spec} option to the \type{\pdfximage} primitive,
or by setting the \type{\pdfforcepagebox} primitive to
a~value corresponding to the wanted box type.

To get the right BoundingBox from a~\EPS\ file, before converting to \PDF,
it is necessary to transform the \EPS\ file so that the start point is
at the (0,0)~coordinate and the page size is set exactly corresponding
to the BoundingBox. A~\PERL\ script (\EPSTOPDF) for this purpose has been
written. The \TEXUTIL\ utility script and the \PSTOPDF\ program that comes
with \GHOSTSCRIPT\ can so a~similar job. (Concerning this conversion,
they can handle complete directories, remove some garbage from files,
takes precautions against duplicate conversion, etc.)

The lossless compressing \PNG\ format is great for embedding crisp pixel
graphics (\eg\ line scans, screen shots). Since \PDFTEX\ 1.30.0 also the
alpha-channel of \PNG\ images is processed if available; this allows
embedding of images with simple transparency. The \PNG\ format does not
support the CMYK color model, which is sometimes required for print media
(this often can be replaced by four component \JPEG\ in high quality or
lossless compression mode). Photos in \PNG\ format have a~rather weak
compression; here the \JPEG\ format is preferable.

Embedding \PNG\ images in the general case requires \PDFTEX\ to uncompress
the pixel array and to re||compress it to the \PDF\ requirements; this
process often takes a~noticeable amount of time. Since \PDFTEX\ 1.30.0
there is now a~fast \PNG\ embedding mode that goes without uncompressing;
the image data are directly copied into the \PDF\ stream, resulting in
 a~much higher embedding speed. However this mode is only activated, if
the image array structure of the \PNG\ file is compatible with the \PDF\
image structure (\eg\ an interlaced \PNG\ image requires uncompressing to
re||arrange the image lines). Luckily it seems that the most common \PNG\
files also allow fast copying. The use of gamma correction disables fast
copying, as it requires calculations with individual pixels. Whether the
fast copy mode is used for a~\PNG\ image can be seen from the log file,
which then shows the string `(PNG copy)' after the \PNG\ file name.

The \JPEG\ format is normally used in lossy mode; then it's ideal for
embedding photos; it's not recommended for crisp images from synthetic
sources with a~limited amount of colors.

Other options for graphics in \PDFTEX\ are:

\description {\LATEX\ picture mode} Since this is implemented simply in terms
of font characters, it works in exactly the same way as usual.

\description {Xy||pic} If the \POSTSCRIPT\ back||end is not requested, Xy||pic
uses its own Type~1 fonts, and needs no special attention.

\description {tpic} The \quote {tpic} \type{\special} commands (used in some
macro packages) can be redefined to produce literal \PDF, using some macros
written by Hans Hagen.

\description {\METAPOST} Although the output of \METAPOST\ is \POSTSCRIPT,
it is in a~highly simplified form, and a~\METAPOST\ to \PDF\ conversion
(\MPTOPDF, written by Hans Hagen and Tanmoy Bhattacharya) is implemented
as a~set of macros which reads \METAPOST\ output and supports all of
its features.

For new work, the \METAPOST\ route is highly recommended. For the future,
Adobe has announced that they will define a~specification for \quote
{encapsulated \PDF}.

The inclusion of raw \POSTSCRIPT\ commands |<|a technique utilized
by for instance the \type{pstricks} package|>| cannot directly be
supported. Although \PDF\ is direct a~descendant of \POSTSCRIPT, it
lacks any programming language commands, and cannot deal with arbitrary
\POSTSCRIPT.

%***********************************************************************

\section{Character translation}

Characters that are input to \PDFTEX\ are subject to optional
\TEX\ character translation (\TCX) under control of a~\TCX\ file.
The \TCX\ maps the input character codes (\eg\ from \type{\input} or
\type{\read}) to the character codes as seen by \PDFTEX.  This mapping
takes place before the characters enter \PDFTEX's `mouth'.  If no \TCX\
file is read, the input characters enter \PDFTEX\ directly; no mapping
is done.

\TCX\ files consist of lines each containing one or two integer numbers
in the range 0..255, either in decimal or hex notation.
% strtol() C-function
 A~comment sign \type{%} in a~\TCX\ line starts a~comment until the
end of line.  The first number in each line is for matching the input
character code, the second, optional number is the corresponding \TEX\
character code.  If a~line contains only one number, characters with
this code enter \PDFTEX\ unchanged; no mapping is done.

\TCX\ mapping also influences \PDFTEX\ output streams for \type{\message} and
\type{\write}.  Without \TCX\ mapping, only characters that are within
the range 32..126 are flagged as `printable', meaning that these
characters are output directly by \type{\message} and \type{\write}
primitives.  Characters outside the range 32..126 are instead output
in escaped form, \eg\ as \type{^^A} for a~character with code
\type{0x01}.  When a~character code is mentioned in the 2nd column of the
\TCX\ file, or as the only value in a~line, it is flagged as `printable'.
During \type{\message} and \type{\write}, output characters are mapped
in reverse direction: they are looked up in the 2nd column of the \TCX\
file and the corresponding values from the 1st column are output.  Again,
if a~\PDFTEX\ character code is found as the only number in a~line, no
mapping is done.  Mentioning a~character code as the only number on
a~line has the sole purpose to flag this code `printable'; remember that
character within the range 32..126 are `printable' anyway.

The characters output into the \PDF\ file, \eg\ by \type{\pdfliteral}
or \type{\special} primitives, are not subject to \TCX\ output remapping.

Beware: Character translation interferes with the \ENCTEX\ primitives; to
avoid surprises, don't use \ENCTEX\ and \TCX\ mapping at the same time.
Further details about \TCX\ file loading can be found in the \TETEX\
manual.

%***********************************************************************

\section{Limitations of \PDFTEX}

\PDFTEX\ currently lacks a~colorstack.
This can be overcome by the \type{pdfcolmk} package.

%***********************************************************************

\stopbodymatter

%D We did use some abbreviations. Only those really used will end up in the
%D following list.

\startbackmatter

\writebetweenlist[section]{\blank[line]}

%***********************************************************************

\section{Abbreviations}

In this document we used a~few abbreviations. For convenience we mention
their meaning here.

\placelistofabbreviations

%***********************************************************************

\start \setupalign[nothanging,nohz]

\section{Examples of HZ and protruding}

In the following sections we will demonstrate \PDFTEX's protruding and
\HZ\ features, using a~text from E.~Tufte. This sample text has a~lot
of punctuation and often needs hyphenation. Former \PDFTEX\ versions
had sometimes problems with combining these features, but from version
1.21a on it should be ok. If you still encounter problems, please try
to prepare a~small test file that demonstrates the problem and send it
to one of the maintainers.

\startbuffer
    \input tufte
    \blank
    \startcolumns
        \input tufte
    \stopcolumns
\stopbuffer

\subsection{Normal}     \start                         \getbuffer \stop
\subsection{HZ}         \start \setupalign[hz]         \getbuffer \stop
\subsection{Protruding} \start \setupalign[hanging]    \getbuffer \stop
\subsection{Both}       \start \setupalign[hz,hanging] \getbuffer \stop

\stop

%***********************************************************************

\section{Additional \PDF\ keys}

{\em This section is based on the manual on keys written by Martin
Schr\"oder, one of the maintainers of \PDFTEX.}

 A~\PDF\ document should contain only the structures and attributes defined
in the \PDF\ specification. However, the specification allows applications
to insert additional keys, provided they follow certain rules.

The most important rule is that developers have to register with Adobe
prefixes for the keys they want to insert.  Hans Hagen has registered the
prefix \type {PTEX} for \PDFTEX.

\PDFTEX\ generates an XObject for every included \PDF. The dictionary of
this object contains these additional keys:

\starttabulate[|lT|l|p|]
\HL
\NC \bf key         \NC \bf type   \NC meaning \NC \NR
\HL
\NC PTEX.FileName   \NC string     \NC The name of the included file as seen by
                                       \PDFTEX. \NC \NR
\NC PTEX.InfoDict   \NC dictionary \NC The document information dictionary of the included
                                       \PDF\ (an indirect object). \NC \NR
\NC PTEX.PageNumber \NC integer    \NC The page number of the included file. \NC \NR
\HL
\stoptabulate

The \PDF\ reference manual says: \quotation {Although viewer applications
can store custom metadata in the document information dictionary, it is
inappropriate to store private content or structural information there;
such information should be stored in the document catalog instead.}

Although it would seem more natural to put this infomation in the document
information dictionary, we have to obey the rules laid down in the \PDF\
reference. The following key ends up in the document catalog.

\starttabulate[|lT|l|p|]
\HL
\NC \bf key         \NC \bf type   \NC meaning \NC \NR
\HL
\NC PTEX.Fullbanner \NC string     \NC The full version of the \pt binary that
produced the file as displayed by \typ {pdftex --version}, \eg\
{\tt\pdftexbanner}.  This is necessary because the string in the
\type {Producer} key in the info dictionary is rather short,
\eg\ {\tt pdfeTeX-\currentpdftex}. \NC \NR
\HL
\stoptabulate

%***********************************************************************

\section{Colophon}

This manual is typeset in \CONTEXT. One can generate an A4 version from
the source code by typing:

\starttyping
texexec --result=pdftex-a.pdf pdftex-t
\stoptyping

Or in letter size:

\starttyping
texexec  --mode=letter --result=pdftex-l.pdf pdftex-t
\stoptyping

Given that the A4 version is typeset, one can generate an A5 booklet by
typing:

\starttyping
texexec --pdfarrange --paper=a5a4 --print=up --addempty=1,2
  --result=pdftex-b.pdf pdftex-a
\stoptyping

Odd and even page sets for non-duplex printers can be generated using
\type{--pages=odd} and \type{--pages=even} options
(which might require some disciplined shuffling of sheet).

This also demonstrates that \PDFTEX\ can be used for page imposition
purposes (given that \PDFTEX\ and the fonts are set up properly).

%***********************************************************************

\page

% The next section is sponsored by the paper producing industry -)
% I will make this proper \quotation's some day

\definehead[gnusection][subsection]
\definehead[gnusubject][subsubject]

\setuphead[gnusection,gnusubject][style=\bfa,before=\blank,after=\blank,align={right,broad,nothyphenated}]

\setuphead[gnusection][ownnumber=yes]

\section{GNU Free Documentation License}

\startnotmode[screen]
    \switchtobodyfont[7pt]
    \setuplayout[grid=yes]
    \setupcolumns[n=3]
\stopnotmode

\startmode[screen]
    \setupcolumns[n=2]
\stopmode

\startcolumns[tolerance=verytolerant,distance=12pt] \widowpenalty10000 \clubpenalty10000

\startlines
Version 1.2, November 2002
Copyright \copyright\ 2000, 2001, 2002
Free Software Foundation, Inc.
59 Temple Place, Suite 330,
Boston, MA  02111-1307  USA
\stoplines

Everyone is permitted to copy and distribute verbatim copies
of this license document, but changing it is not allowed.

\gnusubject{Preamble}

The purpose of this License is to make a~manual, textbook, or other
functional and useful document ``free'' in the sense of freedom: to
assure everyone the effective freedom to copy and redistribute it,
with or without modifying it, either commercially or noncommercially.
Secondarily, this License preserves for the author and publisher a~way
to get credit for their work, while not being considered responsible
for modifications made by others.

This License is a~kind of ``copyleft'', which means that derivative
works of the document must themselves be free in the same sense.  It
complements the GNU General Public License, which is a~copyleft
license designed for free software.

We have designed this License in order to use it for manuals for free
software, because free software needs free documentation: a~free
program should come with manuals providing the same freedoms that the
software does.  But this License is not limited to software manuals;
it can be used for any textual work, regardless of subject matter or
whether it is published as a~printed book.  We recommend this License
principally for works whose purpose is instruction or reference.

\gnusection{1}{APPLICABILITY AND DEFINITIONS}

This License applies to any manual or other work, in any medium, that
contains a~notice placed by the copyright holder saying it can be
distributed under the terms of this License.  Such a~notice grants
a~world-wide, royalty-free license, unlimited in duration, to use that
work under the conditions stated herein.  The {\bf ``Document''}, below,
refers to any such manual or work.  Any member of the public is
a~licensee, and is addressed as {\bf ``you''}.  You accept the license if you
copy, modify or distribute the work in a~way requiring permission
under copyright law.

 A~{\bf ``Modified Version''} of the Document means any work containing the
Document or a~portion of it, either copied verbatim, or with
modifications and/or translated into another language.

 A~{\bf ``Secondary Section''} is a~named appendix or a~front-matter section of
the Document that deals exclusively with the relationship of the
publishers or authors of the Document to the Document's overall subject
(or to related matters) and contains nothing that could fall directly
within that overall subject.  (Thus, if the Document is in part
a~textbook of mathematics, a~Secondary Section may not explain any
mathematics.)  The relationship could be a~matter of historical
connection with the subject or with related matters, or of legal,
commercial, philosophical, ethical or political position regarding
them.

The {\bf ``Invariant Sections''} are certain Secondary Sections whose titles
are designated, as being those of Invariant Sections, in the notice
that says that the Document is released under this License.  If
a~section does not fit the above definition of Secondary then it is not
allowed to be designated as Invariant.  The Document may contain zero
Invariant Sections.  If the Document does not identify any Invariant
Sections then there are none.

The {\bf ``Cover Texts''} are certain short passages of text that are listed,
as Front-Cover Texts or Back-Cover Texts, in the notice that says that
the Document is released under this License. A~Front-Cover Text may
be at most 5 words, and a~Back-Cover Text may be at most 25 words.

 A~{\bf ``Transparent''} copy of the Document means a~machine-readable copy,
represented in a~format whose specification is available to the
general public, that is suitable for revising the document
straightforwardly with generic text editors or (for images composed of
pixels) generic paint programs or (for drawings) some widely available
drawing editor, and that is suitable for input to text formatters or
for automatic translation to a~variety of formats suitable for input
to text formatters. A~copy made in an otherwise Transparent file
format whose markup, or absence of markup, has been arranged to thwart
or discourage subsequent modification by readers is not Transparent.
An image format is not Transparent if used for any substantial amount
of text. A~copy that is not ``Transparent'' is called {\bf ``Opaque''}.

Examples of suitable formats for Transparent copies include plain
ASCII without markup, Texinfo input format, LaTeX input format, SGML
or XML using a~publicly available DTD, and standard-conforming simple
HTML, \POSTSCRIPT\ or \PDF\ designed for human modification.  Examples of
transparent image formats include PNG, XCF and JPG.  Opaque formats
include proprietary formats that can be read and edited only by
proprietary word processors, SGML or XML for which the DTD and/or
processing tools are not generally available, and the
machine-generated HTML, \POSTSCRIPT\ or \PDF\ produced by some word
processors for output purposes only.

The {\bf ''Title Page''} means, for a~printed book, the title page itself,
plus such following pages as are needed to hold, legibly, the material
this License requires to appear in the title page.  For works in
formats which do not have any title page as such, ``Title Page'' means
the text near the most prominent appearance of the work's title,
preceding the beginning of the body of the text.

 A~section {\bf ``Entitled XYZ''} means a~named subunit of the Document whose
title either is precisely XYZ or contains XYZ in parentheses following
text that translates XYZ in another language.  (Here XYZ stands for
a~specific section name mentioned below, such as {\bf ``Acknowledgements''},
{\bf ``Dedications''}, {\bf ``Endorsements''}, or {\bf ``History''}.)
To {\bf ``Preserve the Title''}
of such a~section when you modify the Document means that it remains
a~section ``Entitled XYZ'' according to this definition.

The Document may include Warranty Disclaimers next to the notice which
states that this License applies to the Document.  These Warranty
Disclaimers are considered to be included by reference in this
License, but only as regards disclaiming warranties: any other
implication that these Warranty Disclaimers may have is void and has
no effect on the meaning of this License.


\gnusection{2}{VERBATIM COPYING}

You may copy and distribute the Document in any medium, either
commercially or noncommercially, provided that this License, the
copyright notices, and the license notice saying this License applies
to the Document are reproduced in all copies, and that you add no other
conditions whatsoever to those of this License.  You may not use
technical measures to obstruct or control the reading or further
copying of the copies you make or distribute.  However, you may accept
compensation in exchange for copies.  If you distribute a~large enough
number of copies you must also follow the conditions in section 3.

You may also lend copies, under the same conditions stated above, and
you may publicly display copies.


\gnusection{3}{COPYING IN QUANTITY}

If you publish printed copies (or copies in media that commonly have
printed covers) of the Document, numbering more than 100, and the
Document's license notice requires Cover Texts, you must enclose the
copies in covers that carry, clearly and legibly, all these Cover
Texts: Front-Cover Texts on the front cover, and Back-Cover Texts on
the back cover.  Both covers must also clearly and legibly identify
you as the publisher of these copies.  The front cover must present
the full title with all words of the title equally prominent and
visible.  You may add other material on the covers in addition.
Copying with changes limited to the covers, as long as they preserve
the title of the Document and satisfy these conditions, can be treated
as verbatim copying in other respects.

If the required texts for either cover are too voluminous to fit
legibly, you should put the first ones listed (as many as fit
reasonably) on the actual cover, and continue the rest onto adjacent
pages.

If you publish or distribute Opaque copies of the Document numbering
more than 100, you must either include a~machine-readable Transparent
copy along with each Opaque copy, or state in or with each Opaque copy
 a~computer-network location from which the general network-using
public has access to download using public-standard network protocols
 a~complete Transparent copy of the Document, free of added material.
If you use the latter option, you must take reasonably prudent steps,
when you begin distribution of Opaque copies in quantity, to ensure
that this Transparent copy will remain thus accessible at the stated
location until at least one year after the last time you distribute an
Opaque copy (directly or through your agents or retailers) of that
edition to the public.

It is requested, but not required, that you contact the authors of the
Document well before redistributing any large number of copies, to give
them a~chance to provide you with an updated version of the Document.


\gnusection{4}{MODIFICATIONS}

You may copy and distribute a~Modified Version of the Document under
the conditions of sections 2 and 3 above, provided that you release
the Modified Version under precisely this License, with the Modified
Version filling the role of the Document, thus licensing distribution
and modification of the Modified Version to whoever possesses a~copy
of it.  In addition, you must do these things in the Modified Version:

\startitemize[A,packed]
\item
   Use in the Title Page (and on the covers, if any) a~title distinct
   from that of the Document, and from those of previous versions
   (which should, if there were any, be listed in the History section
   of the Document).  You may use the same title as a~previous version
   if the original publisher of that version gives permission.

\item
   List on the Title Page, as authors, one or more persons or entities
   responsible for authorship of the modifications in the Modified
   Version, together with at least five of the principal authors of the
   Document (all of its principal authors, if it has fewer than five),
   unless they release you from this requirement.

\item
   State on the Title page the name of the publisher of the
   Modified Version, as the publisher.

\item
   Preserve all the copyright notices of the Document.

\item
   Add an appropriate copyright notice for your modifications
   adjacent to the other copyright notices.

\item
   Include, immediately after the copyright notices, a~license notice
   giving the public permission to use the Modified Version under the
   terms of this License, in the form shown in the Addendum below.

\item
   Preserve in that license notice the full lists of Invariant Sections
   and required Cover Texts given in the Document's license notice.

\item
   Include an unaltered copy of this License.

\item
   Preserve the section Entitled ``History'', Preserve its Title, and add
   to it an item stating at least the title, year, new authors, and
   publisher of the Modified Version as given on the Title Page.  If
   there is no section Entitled ``History'' in the Document, create one
   stating the title, year, authors, and publisher of the Document as
   given on its Title Page, then add an item describing the Modified
   Version as stated in the previous sentence.

\item
   Preserve the network location, if any, given in the Document for
   public access to a~Transparent copy of the Document, and likewise
   the network locations given in the Document for previous versions
   it was based on.  These may be placed in the ``History'' section.
   You may omit a~network location for a~work that was published at
   least four years before the Document itself, or if the original
   publisher of the version it refers to gives permission.

\item
   For any section Entitled ``Acknowledgements'' or ``Dedications'',
   Preserve the Title of the section, and preserve in the section all
   the substance and tone of each of the contributor acknowledgements
   and/or dedications given therein.

\item
   Preserve all the Invariant Sections of the Document,
   unaltered in their text and in their titles.  Section numbers
   or the equivalent are not considered part of the section titles.

\item
   Delete any section Entitled ``Endorsements''.  Such a~section
   may not be included in the Modified Version.

\item
   Do not retitle any existing section to be Entitled ``Endorsements''
   or to conflict in title with any Invariant Section.

\item
   Preserve any Warranty Disclaimers.
\stopitemize

If the Modified Version includes new front-matter sections or
appendices that qualify as Secondary Sections and contain no material
copied from the Document, you may at your option designate some or all
of these sections as invariant.  To do this, add their titles to the
list of Invariant Sections in the Modified Version's license notice.
These titles must be distinct from any other section titles.

You may add a~section Entitled ``Endorsements'', provided it contains
nothing but endorsements of your Modified Version by various
parties--for example, statements of peer review or that the text has
been approved by an organization as the authoritative definition of
a~standard.

You may add a~passage of up to five words as a~Front-Cover Text, and
a~passage of up to 25 words as a~Back-Cover Text, to the end of the list
of Cover Texts in the Modified Version.  Only one passage of
Front-Cover Text and one of Back-Cover Text may be added by (or
through arrangements made by) any one entity.  If the Document already
includes a~cover text for the same cover, previously added by you or
by arrangement made by the same entity you are acting on behalf of,
you may not add another; but you may replace the old one, on explicit
permission from the previous publisher that added the old one.

The author(s) and publisher(s) of the Document do not by this License
give permission to use their names for publicity for or to assert or
imply endorsement of any Modified Version.

\gnusection{5}{COMBINING DOCUMENTS}

You may combine the Document with other documents released under this
License, under the terms defined in section 4 above for modified
versions, provided that you include in the combination all of the
Invariant Sections of all of the original documents, unmodified, and
list them all as Invariant Sections of your combined work in its
license notice, and that you preserve all their Warranty Disclaimers.

The combined work need only contain one copy of this License, and
multiple identical Invariant Sections may be replaced with a~single
copy.  If there are multiple Invariant Sections with the same name but
different contents, make the title of each such section unique by
adding at the end of it, in parentheses, the name of the original
author or publisher of that section if known, or else a~unique number.
Make the same adjustment to the section titles in the list of
Invariant Sections in the license notice of the combined work.

In the combination, you must combine any sections Entitled ``History''
in the various original documents, forming one section Entitled
``History''; likewise combine any sections Entitled ``Acknowledgements'',
and any sections Entitled ``Dedications''.  You must delete all sections
Entitled ``Endorsements''.

\gnusection{6}{COLLECTIONS OF DOCUMENTS}

You may make a~collection consisting of the Document and other documents
released under this License, and replace the individual copies of this
License in the various documents with a~single copy that is included in
the collection, provided that you follow the rules of this License for
verbatim copying of each of the documents in all other respects.

You may extract a~single document from such a~collection, and distribute
it individually under this License, provided you insert a~copy of this
License into the extracted document, and follow this License in all
other respects regarding verbatim copying of that document.

\gnusection{7}{AGGREGATION WITH INDEPENDENT WORKS}

 A~compilation of the Document or its derivatives with other separate
and independent documents or works, in or on a~volume of a~storage or
distribution medium, is called an ``aggregate'' if the copyright
resulting from the compilation is not used to limit the legal rights
of the compilation's users beyond what the individual works permit.
When the Document is included in an aggregate, this License does not
apply to the other works in the aggregate which are not themselves
derivative works of the Document.

If the Cover Text requirement of section 3 is applicable to these
copies of the Document, then if the Document is less than one half of
the entire aggregate, the Document's Cover Texts may be placed on
covers that bracket the Document within the aggregate, or the
electronic equivalent of covers if the Document is in electronic form.
Otherwise they must appear on printed covers that bracket the whole
aggregate.

\gnusection{8}{TRANSLATION}

Translation is considered a~kind of modification, so you may
distribute translations of the Document under the terms of section 4.
Replacing Invariant Sections with translations requires special
permission from their copyright holders, but you may include
translations of some or all Invariant Sections in addition to the
original versions of these Invariant Sections.  You may include
a~translation of this License, and all the license notices in the
Document, and any Warranty Disclaimers, provided that you also include
the original English version of this License and the original versions
of those notices and disclaimers.  In case of a~disagreement between
the translation and the original version of this License or a~notice
or disclaimer, the original version will prevail.

If a~section in the Document is Entitled ``Acknowledgements'',
``Dedications'', or ``History'', the requirement (section 4) to Preserve
its Title (section~1) will typically require changing the actual
title.

\gnusection{9}{TERMINATION}

You may not copy, modify, sublicense, or distribute the Document except
as expressly provided for under this License.  Any other attempt to
copy, modify, sublicense or distribute the Document is void, and will
automatically terminate your rights under this License.  However,
parties who have received copies, or rights, from you under this
License will not have their licenses terminated so long as such
parties remain in full compliance.

\gnusection{10}{FUTURE REVISIONS OF THIS LICENSE}

The Free Software Foundation may publish new, revised versions
of the GNU Free Documentation License from time to time.  Such new
versions will be similar in spirit to the present version, but may
differ in detail to address new problems or concerns.  See
http://www.gnu.org/copyleft/.

Each version of the License is given a~distinguishing version number.
If the Document specifies that a~particular numbered version of this
License ``or any later version'' applies to it, you have the option of
following the terms and conditions either of that specified version or
of any later version that has been published (not as a~draft) by the
Free Software Foundation.  If the Document does not specify a~version
number of this License, you may choose any version ever published (not
as a~draft) by the Free Software Foundation.
\stopcolumns

%***********************************************************************

\stopbackmatter

%D And then we're done.

\stoptext
